\documentclass[11pt]{report}
\pagestyle{plain}
\usepackage[spanish]{babel}
\selectlanguage{spanish}
\usepackage[utf8]{inputenc}
\usepackage{listings}
\usepackage{color}
\usepackage[table]{xcolor}
\usepackage{graphicx}
\usepackage{amssymb}

% ------------------- Titulo y  Autor -----------------------------
\title{Samplers: Framework para construir aplicaciones Android para recolectar muestras  en proyectos de Ciencia Ciudadana}
\author{Laura Lus y Javier Ramírez}

\begin{document}

\maketitle

\begin{abstract}
%Este es un resumen de la tesis muy corto (media carilla). El que lo lee se tiene que quedar con la idea de: 1) en que tema trabajaron, 2) cual es el problema que intentaron resolver, 3) que técnica aplicaron para resolverlo y por qué es novedosa, 4) que descubrieron en el proceso (un adelanto de conclusión). 

La Ciencia Ciudadana es una forma de investigación que involucra a los ciudadanos (voluntarios) en proyectos de investigación. Los proyectos de investigación que pueden incluir Ciencia Ciudadana se clasifican en acción, conservación, recolección, virtual y educativos. El presente trabajo se concentra en los proyectos de recolección, que son los que requieren la recolección de muestras del medio físico. Samplers es un framework Android para construir aplicaciones que permitan la recolección de muestras utilizando las herramientas que brindan los dispositivos móviles.

\end{abstract}

\tableofcontents

%Cada capitulo será un archivo .tex el cual estará en su propia carpeta, con sus propias imágenes).
%El comando include hace que el capitulo respectivo se incluya en el documento. No es necesario indicar la extensión 
%del archivo dado que asume que es .tex
%Los números en el nombre de las carpetas son solo para tenerlas ordenadas.

\chapter{Introducción}

\label{introduccion}

\section{Ciencia  Ciudadana y Ciencia Abierta}

\begin{itemize}
   \item Ciencia Ciudadana
   \begin{itemize}
        \item Cómo participan los ciudadanos en la ciencia?        
        Esto de que hay que asignarles tareas acordes o darles una pequeña capacitación o ayuda en pantalla. Comunidades que hacen de soporte de voluntarios.
        \item De qué depende que un proyecto incluya ciencia ciudadana?
        Tipología de los proyectos de ciencia ciudadana, por ejemplo, que sea de educación, de investigación, que no cualquier proyecto puede tilizar ciencia ciudadana y no siempre se aplica en todo el proyecto. Muchas veces los ciudadanos colaboran con una parte.
   \end{itemize}   
   \item Ciencia Abierta   
   \begin{itemize}
        \item \item Por qué es importante la ciencia abierta? democracia y cuestiones políticas. Acceso público a la información de interés general. 
        \item Qué relación tiene con la ciencia ciudadana? Básicamente los participantes en proyectos de investigación de ciencia ciudadana lo hacen por interés en el tema de investigación. Es una buena práctica que una vez finalizada la investigación se haga una devolución de los resultados de la misma para que los ciudadanos participantes quienes estaban interesados en el tema de movida puedan ver los resultados de la investigación. Este tema está directamente relacionado con la ciencia abierta que básicamente es abrir los datos, resultados y procesos utilizados para conseguir resultados a el público general.
   \end{itemize}
\end{itemize}

\section{ Dispositivos Móviles y Android }
\begin{itemize}
	\item Distribución de dispositivos móviles entre la población
	cantidad de personas que tienen dispositivos móviles. Que porcentaje de la población representan. Zonas de concentración de dispositivos:cómo están distribuidos
	\item Características de los dispositivos móviles
	cámaras, micrófonos, conexiones a redes, posibilidad de transferencias de archivos, navegabilidad en la interfaz de aplicación. 
    \item Android 
    el sistema operativo. Licencia. Estructura. Versiones y lo que ello implica.
    \item Ejemplos de aplicaciones de ciencia ciudadana y dispositivos móviles 
    hablemos del ejemplo africano que no tenía palabras para que la población partipe sin necesidad de saber leer o escribir. AppEAR y Cazamosquitos. Ejemplo aplicado a salud Colombia
\end{itemize}

\section{ Frameworks }

\begin{itemize}
	\item Frameworks para construir aplicaciones
    \item Configuración de aplicaciones mediante archivos
\end{itemize}

\begin{figure}
\begin{center}
\includegraphics[width=0.8\textwidth]{00-introduccion/zobel-page-46}
\caption{Extracto del libro Writing for Computers Science de Justin Zobel}
\label{zobel-page-46}
\end{center}
\end{figure}


\chapter{Trabajo Relacionado}
Hacer una tesis implica encontrar una pregunta que valga la pena responder o un problema que valga la pena resolver y darle respuesta o solución. Es una tarea de investigación que tiene como aspecto muy importante conocer lo que ya existe alrededor de la pregunta o problema que se elige. 

Al llegar a este capitulo, el lector tiene una idea de cual es el problema. Seguramente se imagina problemas similares o soluciones al problema. El objetivo, en este momento, es convencerlo de que conocemos el problema y otros similares; que conocemos las formas en las que se lo ha intentado resolver (o a problemas similares); y que aún después de saber todo eso sigue siendo un problema importante, difícil y que nadie resolvió 8o nadie revolvió tan bien como nosotros).

Para escribir este capitulo hay que leer. Hay que buscar soluciones a problemas similares y compararlas con lo que nosotros queremos hacer. Si sabemos que la nuestra es mejor, ya podemos marcar cuales son los puntos débiles de las existentes. También se puede escribir un poco sobre otras investigaciones, que si bien no atacaron problemas parecidos, pueden ser aplicadas a resolver parte de este. 

Este capitulo es bueno ir escribiendolo en borrador cada vez que se lee algo (un articulo por ejemplo). Por lo menos hay que escribir un resumen de un párrafo de lo leído (registrando la referencia en el archivo bibliografia.bib y citando dede acá), y dar nuestra opinión al respecto en términos de su relación con el problema de nuestra tesis.
\include{20-estrategia/estrategia}

\include{30-especifico/especifico}

\include{70-evaluacion/evaluacion}
\include{80-conclusiones/conclusiones}


%Los artículos que se citan en la tesis, se incluyen en el archivo bibliografía.bib, en formato bibtex.
%En la sección bibliografía, van a aparecer automáticamente aquellos que se citan desde el texto 
%utilizando el comando \cite con la etiqueta correspondiente - en la introducción hay un par de ejemplos. 

\bibliographystyle{plain}
\bibliography{90-bibliografia}


\end{document}
\end

