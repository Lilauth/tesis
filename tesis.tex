\documentclass[11pt]{report}
\setcounter{tocdepth}{4}
\setcounter{secnumdepth}{4}
\pagestyle{plain}
\usepackage[spanish]{babel}
\selectlanguage{spanish}
\usepackage[utf8]{inputenc}
\usepackage{listings}
%\usepackage{color}
\usepackage{graphicx}
\usepackage{amssymb}
%\usepackage{cite}
%\usepackage{url}
\usepackage{float}
\usepackage{verbatim}
\usepackage{appendix}
\usepackage[table,svgnames]{xcolor}
\usepackage{pdfpages}


\lstset{literate=
  {á}{{\'a}}1 {é}{{\'e}}1 {í}{{\'i}}1 {ó}{{\'o}}1 {ú}{{\'u}}1
  {Á}{{\'A}}1 {É}{{\'E}}1 {Í}{{\'I}}1 {Ó}{{\'O}}1 {Ú}{{\'U}}1
  {à}{{\`a}}1 {è}{{\`e}}1 {ì}{{\`i}}1 {ò}{{\`o}}1 {ù}{{\`u}}1
  {À}{{\`A}}1 {È}{{\'E}}1 {Ì}{{\`I}}1 {Ò}{{\`O}}1 {Ù}{{\`U}}1
  {ä}{{\"a}}1 {ë}{{\"e}}1 {ï}{{\"i}}1 {ö}{{\"o}}1 {ü}{{\"u}}1
  {Ä}{{\"A}}1 {Ë}{{\"E}}1 {Ï}{{\"I}}1 {Ö}{{\"O}}1 {Ü}{{\"U}}1
  {â}{{\^a}}1 {ê}{{\^e}}1 {î}{{\^i}}1 {ô}{{\^o}}1 {û}{{\^u}}1
  {Â}{{\^A}}1 {Ê}{{\^E}}1 {Î}{{\^I}}1 {Ô}{{\^O}}1 {Û}{{\^U}}1
  {œ}{{\oe}}1 {Œ}{{\OE}}1 {æ}{{\ae}}1 {Æ}{{\AE}}1 {ß}{{\ss}}1
  {ű}{{\H{u}}}1 {Ű}{{\H{U}}}1 {ő}{{\H{o}}}1 {Ő}{{\H{O}}}1
  {ç}{{\c c}}1 {Ç}{{\c C}}1 {ø}{{\o}}1 {å}{{\r a}}1 {Å}{{\r A}}1
  {€}{{\EUR}}1 {£}{{\pounds}}1
}

\renewcommand{\appendixname}{Anexos}
\renewcommand{\appendixtocname}{Anexos}
\renewcommand{\appendixpagename}{Anexos}
\addto\captionsspanish{\renewcommand{\appendixname}{Anexo}}

% STYLE para el codigo de ejemplo ===============================================
\definecolor{codegray}{rgb}{0.5,0.5,0.5}

\lstdefinestyle{mystyle}{
    backgroundcolor=\color{White},   
    commentstyle=\color{DeepPink},
    keywordstyle=\color{DarkOrange},
    numberstyle=\tiny\color{codegray},
    stringstyle=\color{SteelBlue},
    basicstyle=\ttfamily\footnotesize,
    breakatwhitespace=false,         
    breaklines=true,                 
    captionpos=b,                    
    keepspaces=true,                 
    numbers=left,                    
    numbersep=5pt,                  
    showspaces=false,                
    showstringspaces=false,
    showtabs=false,                  
    tabsize=2,
    rulecolor=\color{Gray}
}


% TODO ESTO es para poder poner el signo de pregunta que abre adentro del codigo ----------------------------------------------

% some listings shenanigans to have access to the current listings style
% within an escape to LaTeX
\makeatletter
\newcommand\currentStyle@lstparam{}
\lst@AddToHook{Output}{\global\let\currentStyle@lstparam\lst@thestyle}
\lst@AddToHook{OutputOther}{\global\let\currentStyle@lstparam\lst@thestyle}
\makeatother

\makeatletter
% Usage: \highlightcode{color}{content}
\newcommand{\highlightcode}[2]{\currentStyle@lstparam \textcolor{#1}{#2}}
\makeatother

% HASTA ACA -----------------------------------------------------------------------------------------------------------------------------------------------

\lstset{
style=mystyle,
inputencoding=utf8,
extendedchars=true,
}

\renewcommand{\lstlistingname}{Código}

% END STYLE ===================================================================

\begin{document}

\includepdf[pages={1}]{PortadaFacultad.pdf}


% ------------------- Titulo y  Autor -----------------------------
\title{Samplers: Framework para construir aplicaciones Android para recolectar muestras en proyectos de Ciencia Ciudadana}
\author{Laura Lus y Javier Ramírez}

\maketitle

\begin{abstract}
%Este es un resumen de la tesis muy corto (media carilla). El que lo lee se tiene que quedar con la idea de: 1) en que tema trabajaron, 2) cual es el problema que intentaron resolver, 3) que técnica aplicaron para resolverlo y por qué es novedosa, 4) que descubrieron en el proceso (un adelanto de conclusión). 


La Ciencia Ciudadana involucra al público en proyectos de investigación científica. Las tareas de un voluntario o ciudadano científico pueden ser simples y no necesitar ningún conocimiento especial, o pueden ser más complejas y requerir capacitación previa. Ejemplos de tareas en proyectos de ciencia ciudadana pueden ser contar elementos que aparecen en una fotografía (si aparece un determinado animal o si puede reconocer una galaxia) o bien responder una serie de preguntas para recolectar información sobre un ambiente que el voluntario está observando (podría ser el ecosistema que rodea una laguna o un estuario). 

Hacer partícipes a los ciudadanos de proyectos de investigación científica persigue varios fines, entre de ellos poder realizar investigaciones a gran escala temporal y espacial, brindar la oportunidad de participar en proyectos reales e interactuar con científicos, o bien perseguir fines educativos. Los proyectos de investigación que incluyen Ciencia Ciudadana pueden clasificarse en acción, conservación, recolección, virtual y educativos. 

El presente trabajo presenta un framework Android para construir aplicaciones que permitan la recolección de muestras utilizando las herramientas que brindan los dispositivos móviles, como puede ser geolocalización o toma de fotografías. Está orientado a los proyectos de recolección, que son los que requieren recolectar muestras del medio físico, y más específicamente a los que requieren recolectar muestras haciendo uso de dispositivos móviles.  

\end{abstract}

\chapter*{Agradecimientos} % el * va si no queremos que añada la palabra "Capitulo"
%\addcontentsline{toc}{chapter}{Agradecimientos} % si queremos que aparezca en el índice

Quiero darle las gracias a mi madre y a mi padre por su gigantesca y necesaria presencia.

A mi hijo Camilo, a mis hermanes y amigues, no puedo pedir una hinchada mejor que ustedes.

A Damián, que el camino nos encuentre siempre compartiendo mates y estudio.

\begin{flushright}
Laura
\end{flushright}


Quiero agradecer a mis viejos, por los valores inculcados, por haberme dado siempre lo mejor de ellos y por haber hecho el gran esfuerzo para que pudiera estudiar en La Plata.

A mis hermanos y cuñadas, por el apoyo y acompañamiento cuando más lo necesitaba.

A mi novia, por apoyarme, acompañarme y estimularme para que terminara esta tesina.

A Sura, donde quiera que estés, por haberme hecho compañía  todo el tiempo mientras desarrollaba este trabajo.


\begin{flushright}
Javier
\end{flushright}


Queremos agradecer a nuestros directores, Diego y Alejandro, por habernos apoyado para terminar esta tesina.

También queremos agradecer a la educación pública porque si no fuera gratuita esto no habría sido posible,  y en especial a la Facultad de Informática de la UNLP por la calidad académica que brinda a sus alumnos.


\begin{flushright}
Laura y Javier
\end{flushright}

\tableofcontents

%Cada capitulo será un archivo .tex el cual estará en su propia carpeta, con sus propias imágenes).
%El comando include hace que el capitulo respectivo se incluya en el documento. No es necesario indicar la extensión 
%del archivo dado que asume que es .tex
%Los números en el nombre de las carpetas son solo para tenerlas ordenadas.

\chapter{Introducción}

\label{introduccion}

\section{Ciencia  Ciudadana y Ciencia Abierta}

\begin{itemize}
   \item Ciencia Ciudadana
   \begin{itemize}
        \item Cómo participan los ciudadanos en la ciencia?        
        Esto de que hay que asignarles tareas acordes o darles una pequeña capacitación o ayuda en pantalla. Comunidades que hacen de soporte de voluntarios.
        \item De qué depende que un proyecto incluya ciencia ciudadana?
        Tipología de los proyectos de ciencia ciudadana, por ejemplo, que sea de educación, de investigación, que no cualquier proyecto puede tilizar ciencia ciudadana y no siempre se aplica en todo el proyecto. Muchas veces los ciudadanos colaboran con una parte.
   \end{itemize}   
   \item Ciencia Abierta   
   \begin{itemize}
        \item \item Por qué es importante la ciencia abierta? democracia y cuestiones políticas. Acceso público a la información de interés general. 
        \item Qué relación tiene con la ciencia ciudadana? Básicamente los participantes en proyectos de investigación de ciencia ciudadana lo hacen por interés en el tema de investigación. Es una buena práctica que una vez finalizada la investigación se haga una devolución de los resultados de la misma para que los ciudadanos participantes quienes estaban interesados en el tema de movida puedan ver los resultados de la investigación. Este tema está directamente relacionado con la ciencia abierta que básicamente es abrir los datos, resultados y procesos utilizados para conseguir resultados a el público general.
   \end{itemize}
\end{itemize}

\section{ Dispositivos Móviles y Android }
\begin{itemize}
	\item Distribución de dispositivos móviles entre la población
	cantidad de personas que tienen dispositivos móviles. Que porcentaje de la población representan. Zonas de concentración de dispositivos:cómo están distribuidos
	\item Características de los dispositivos móviles
	cámaras, micrófonos, conexiones a redes, posibilidad de transferencias de archivos, navegabilidad en la interfaz de aplicación. 
    \item Android 
    el sistema operativo. Licencia. Estructura. Versiones y lo que ello implica.
    \item Ejemplos de aplicaciones de ciencia ciudadana y dispositivos móviles 
    hablemos del ejemplo africano que no tenía palabras para que la población partipe sin necesidad de saber leer o escribir. AppEAR y Cazamosquitos. Ejemplo aplicado a salud Colombia
\end{itemize}

\section{ Frameworks }

\begin{itemize}
	\item Frameworks para construir aplicaciones
    \item Configuración de aplicaciones mediante archivos
\end{itemize}

\begin{figure}
\begin{center}
\includegraphics[width=0.8\textwidth]{00-introduccion/zobel-page-46}
\caption{Extracto del libro Writing for Computers Science de Justin Zobel}
\label{zobel-page-46}
\end{center}
\end{figure}


\chapter{Marco Teórico}
		
	
\section{Introducción a la Ciencia Ciudadana}
    
    Ciencia Ciudadana es un termino que engloba las diferentes maneras en las que los ciudadanos participan en ciencia. Estas actividades pueden ser enviar información desde aplicaciones instaladas en sus dispositivos móviles acerca de especies invasivas, comportamiento de las aves, la cantidad de mariposas presentes al comienzo de la primavera o cualquier otro tema generalmente relacionado a la ecología y la conservación; como así también participar en debates locales de políticas que afectan directamente el ecosistema de una determinada ciudad o zona, como puede ser el fracking, la minería y los pesticidas en la agricultura y de esta manera influir en las políticas locales respecto de cómo regular esas actividades.\cite{envCitizenScience}
    
	Algunos de estos proyectos cuantan con una larga historia de investigación (como el CBC, Christmas Bird Count), otros con gran cantidad de participantes, quienes interactúan a través de foros y aplicaciones web desde la comodidad de sus hogares (como sucede en proyectos como Zooniverse) o bien abarcando grandes extensiones geográficas (como puede ser en The Big Butterfly Count). \cite{shirk2012public} 
		
	Aunque la ciencia ciudadana no es un concepto nuevo, su notoriedad es relativamente reciente. Los científicos ciudadanos o voluntarios ahora participan de proyectos relacionados con el cambio climático, las especies invasivas, la conservación de ecosistemas biológicos, el monitoreo de la calidad del agua y varios tópicos más. Esta popularidad se vio impulsada principalmente por tres factores:

\begin{itemize}
	\item {Disponibilidad Tecnológica}
	La disponibilidad de herramientas tecnológicas que permitan distribuir información acerca de los proyectos y también permitan recolectar la información generada por los voluntarios. De estas herramientas internet es la más representativa, pero la tecnología móvil está jugando un rol fundamental con la popularización de smartphones. \cite{silvertown2009new}
	\item {Reconocimiento del Aporte de los Voluntarios por parte de los Profesionales}
	La participación de voluntarios en un proyecto de investigación aporta diferentes atributos que pueden ser trabajo, poder de cómputo o habilidad \cite{cohn2008citizen}
	\item {Inclusión del Público General en Proyectos Científicos}
	La mejor manera para que el común de la gente entienda y se involucre en proyectos científicos es siendo parte de ellos. \cite{silvertown2009new}
\end{itemize} 

	En ciencias como pueden ser la arqueología, la astronomía o las ciencias naturales la capacidad de observación es a veces es más importante que el equipamiento costoso. Los voluntarios o científicos ciudadanos realizan actividades como parte de un proyecto científico que está especialmente diseñado o bien fue adaptado para que cumplan un rol, ya sea para fines educativos de los mismos voluntarios o para beneficio del proyecto. En los mejores ejemplos, se benefician ambos: los voluntarios y el proyecto.\cite{silvertown2009new}
	Para ello vamos a utilizar la clasificación provista por Wiggins and Crowston para tener un marco de lo que son los proyectos de recolección según esta clasificación, y por qué son los que mejor se relacionan con los proyectos de Ciencia Ciudadana en donde los colaboradores actúan como investigadores de campo recolectando muestras para introducirel tema de la toma y el envío de muestras recolectadas mediante aplicaciones móviles.	

\section{Clasificación de los Proyectos de Ciencia Ciudadana}	

\begin{itemize}
	\item {Acción}
		
		Los proyectos clasificados en esta categoría no son planificados o iniciados por científicos, sino más bien por los ciudadanos, y generalmente requieren un compromiso a lo largo del tiempo en los problemas ambientales locales por lo cual las actividades científicas están orientadas al ambiente físico. 
		
		Estos proyectos solicitan la colaboración de científicos como asesores o consultores, y no como organizadores. Los datos o resultados obtenidos no persiguen un fin académico, sino más bien buscan fundamentar con evidencia para poder tomar acciones sobre alguna situación. 
	\item {Conservación} 
	
	Al igual que los proyectos de Acción, los proyectos de Conservación son fuertemente regionales, y las actividades de los voluntarios están enfocadas mayormente en la recolección de información. La mayoría de los proyectos de investigación tienen contenido o fines educativos. También tienden a ser de alcance regional, como lo reflejan sus desafíos y metas.
	
	Estos proyectos buscan generar información principalmente como fuente para la toma de decisiones relacionadas al manejo de recursos, y también buscan la promoción de la administración y reconocimiento del voluntariado. También prestan especial atención a la generación de información científicamente válida. Son esfuerzos de monitoreo a largo plazo y generalmente no tienen problemas de sustentabilidad, ya que son subvencionadas por fondos públicos o reciben ingresos de agencias que son las que nuclean los proyectos o son las interesadas en sus resultados.
	
	\item {Investigación o Recolección} 
	
	Los proyectos de investigación concentran su atención en investigaciones científicas cuyos objetivos requieren la recolección de información del medio físico. Este tipo de proyectos es el que mejor encaja en la definición de Ciencia Ciudadana. Y aunque los objetivos de educación no son los principales de estos proyectos, forman parte de ellos como material de capacitación o incluyendo estructuras de recolección que alientan el aprendizaje al aire libre. El alcance varía de regional a internacional, y puede lograr participación masiva de hasta decenas de miles de voluntarios y obtener millones de observaciones (de voluntarios) anuales. La mayoría de estos proyectos están enfocados en la investigación biológica, medioambiental o meteorológica, por dar algunos ejemplos. 
	
	Una de las principales preocupaciones de este tipo de proyectos es generar resultados científicamente válidos, ya que son concebidos para generar conocimiento formal y son mayormente organizados por científicos. El cuidadoso diseño del proyecto y de las tareas son los que permiten lograr resultados válidos. Aparte utilizan toda una serie de metodologías que permiten la validación de la información generada. Los voluntarios están dispersos geográficamente y esto es un recurso valioso ya que este tipo de proyectos intenta muchas veces registrar la distribución geográfica de determinadas especies o la ocurrencia de fenómenos naturales. 
	
	\item {Virtual} 
	
	En los proyectos de ciencia ciudadana denominados Virtuales todas las actividades son mediante tecnologías de la información y la comunicación, sin la intervención de elementos del entorno físico.
	
	Los proyectos que cumplen las condiciones para ser clasificados como virtuales provienen de la astronomía, la paleontología y la proteómica, que es una rama de la microbiología que estudia la estructura de las proteínas. Algunos proyectos de psicología podrían clasificar, pero no lo hacen porque los voluntarios colaboran como sujetos de pruebas, y esto no es formalmente considerado como colaboración en la investigación. 
	
	Galaxy Zoo es un ejemplo de proyecto virtual de ciencia ciudadana. Desde hace más de diez años los voluntarios que colaboran con el proyecto clasifican galaxias en fotografías. Responden una serie de preguntas respecto de la foto que están observando, y de esta manera los científicos encargados del proyecto obtienen una primera clasificación, que se construye en base a las observaciones de varios voluntarios de manera independiente. \cite{GalaxyZoo} 
	
	Al igual que en los proyectos de recolección anteriormente mencionados, los proyectos virtuales encuentran dificultades a la hora de conseguir resultados válidos en términos científicos. Estos resultados se obtienen mediante el desarrollo cuidadoso de las actividades. Como la participación es mayormente virtual, poder mantener a los voluntarios comprometidos con el proyecto es un desafío. Por eso muchas veces estos proyectos incluyen técnicas de gamificación, de competencia amigable entre participantes o de valoración de la contribución hecha por el voluntario (feedback).
	 
	\item {Educación} 
	
	Los proyectos de ciencia ciudadana pertenecientes a esta categoría son aquellos cuyo principal objetivo es educar. Los participantes de estos proyectos tienen como objetivo educar, y aportan recursos educativos informales; mientras los proyectos ofrecen material educativo formal. También, las actividades están pensadas para que el participante vaya acumulando conocimientos.
	
	Un ejemplo de este tipo de proyectos es Fossil Finders, que centra la investigación en el análisis de fósiles del Devónico (período de la era Paleozóica) proveyendo de materiales de estudio a estudiantes y profesores de escuelas secundarias. Los estudiantes van a identificar y medir fósiles en muestras de rocas enviadas a sus aulas. Luego ingresarán los datos obtenidos en una base de datos online, y podrán comparar sus datos con los de otras escuelas participantes. Con ello van a tener la oportunidad de involucrarse con métodos de investigación reales y de asistir a los investigadores del Instituto de Investigación Paleontológica a reconstruir el pasado geológico de Nueva York. \cite{FossilFinders}

	La mayoría de estos proyectos proyectos persiguen un fin educativo, el aprendizaje y el desarrollo de habilidades científicas. Por ello incluyen actividades de análisis de datos o muestras, brindando la posibilidad de desarrollar pensamiento crítico. 
	\end{itemize} 
	
	Los proyectos de Ciencia Ciudadana buscan resultados que generalmente caen en tres grandes categorías: resultados que sirven a la investigación; resultados que le sirven a los participantes como pueden ser adquirir nuevas habilidades o conocimientos y/o resultados que tienen que ver con sistemas socio-ecológicos, como es influenciar políticas, construir bases para la toma de decisiones en una comunidad o participar de acciones para la conservación del medio ambiente. \cite{shirk2012public}

\section{Proyectos de Recolección}	 
	En los proyectos de recolección, según la clasificación antes descripta, es en donde Samplers  brindaría su aporte, ya que está pensado para crear aplicaciones Android que sirvan para recolectar muestras en proyectos de Ciencia Ciudadana.
	
	Hay tres puntos en los que se debe enfatizar para que los voluntarios pueden recolectar y enviar información confiable: proveer información clara acerca de los protocolos de recolección, proveer formularios para el ingreso de datos que sean lo más lógicos y simples posibles y por último brindar soporte para que los participantes entiendan cómo seguir los protocolos y cómo enviar la información.\cite{bonney2009citizen}

\begin{itemize}
	\item {Protocolos}	
			Los datos que se obtienen en proyectos de Ciencia Ciudadana son recolectados mediante protocolos que especifican dónde, cuándo y cómo esos datos deben ser recolectados. Los protocolos deben definir un diseño formal o un plan de acción que permita combinar las muestras que fueron tomadas por múltiples participantes en diferentes ubicaciones para su posterior análisis. Los protocolos utilizados en proyectos de ciencia ciudadana deben ser fáciles de ejecutar, deben poder ser explicados de manera simple y directa, y deben ser desafiantes para los voluntarios.\cite{bonney2009citizen}
		Como se explica más adelante, la implementación del protocolo de recolección de muestras es el workflow de Samplers.
		
	\item {Formularios de ingreso de Datos}	
			En conjunto con un protocolo de recolección bien diseñado están los formularios de ingreso de datos. Los formularios en los que los usuarios ingresan sus observaciones deben ser fáciles de entender y completar. En cada paso de su wokflow, Samplers provee formularios de ingreso de datos que pueden ser preguntas de respuesta simple o compuesta (como componentes radio o checkbox), permite sacar fotografías o informar una ubicación entre otras funcionalidad provistas. 
			Es recomendable definir límites en el rango de datos que los formularios recogen para simplificar su posterior análisis. Por ejemplo, si la respuesta esperada a la pregunta 'Cuántos árboles cuenta en una cuadra' debería ser un número entre cero y 15 una respuesta como 150 podría indicar un error de tipeo. Entonces, es aconsejable pedirle al usuario que reporta que chequee si la información ingresada es correcta. De esta manera, los usuarios pueden revisar sus respuestas cvuando están fuera de los rangos establecidos. Aun así, suponiendo que el usuario indique un respuesta de esas características, es decir, una respuesta que se salga de los límites esperados; ese formulario debería guardarse con alguna marca que permita identificarlo para que los responsables del proyecto puedan analizarlo con más detalle y ver si se debe a un cambio en el entorno que está siendo observado o si es realmente un error de ingreso de datos del usuario.
			
	\item {Material Educativo}
		Los participantes deben ser provistos de material educativo para entender y seguir de manera satisfactoria los protocolos del proyecto. El material educativo puede incluir guías de identificación, posters, manuales, videos, podcasts, listas de correo y FAQ para que los participantes puedan consultar e incluso participar en foros y discusiones acerca del relevamiento que están haciendo o de cómo se espera que completen los formularios provistos. 
		Samplers ofrece ayuda en cada formulario o ventana de ingreso de datos que puede ser configurada para brindar información acerca de cómo se espera que el voluntario la complete.
		
\end{itemize} 

	
\section{Ciencia Ciudadana y Dispositivos Móviles}	

Para hablar de esto tengo que agarrar el paper de Sensr y el de The Future of Citizen Science



%\chapter{Características de los Proyectos de Recolección}

\section{Proyectos de Recolección}

	
%Los proyectos de recolección generalmente tienen algunas complejidades particulares. Se necesita mucha gente para tomar las muestras, cubren grandes extensiones de territorio. Los voluntarios son imprescindibles en estos proyectos, pero también la capacitación previa o conocimiento en el área aportan. La toma de muestras sigue determinado conjunto de pasos cuyo orden debe respetarse para que la muestra sea considerada una muestra. %

\subsection{El Método Científico}
No podemos hablar mucho acá, sino más que nada dar una introducción al método científico, que es el que determina por qué una muestra es una muestra, explicar un poco cómo sería el protocolo de la toma de muestras de campo y eso. Esto podemos consultarlo, para ver de donde sacara bibliografía.

\subsection{Toma de Muestras}

	El artículo Citizen Science: Can volunteers do real research? describe el caso de un voluntario que colabora \textit{con un proyecto de ciencia ciudadana en el Appalachian Trail}. Luego de caminar varios kilómetros, se detiene en un lugar predeterminado y saca su GPS para estar seguro de que es el lugar correcto. Se aparta del sendero peatonal y busca un sendero menos transitado en el bosque más denso. Encuentra un camino lleno de huellas de ciervo y excremento que le indica que es el camino que está buscando.
	
	Este voluntario preside el Natural Bridge Appalachian Trail en Lynchburg, y camina de un lado a otro buscando una cámara digital que dejó en un árbol un mes atrás. Luego de buscar durante varios minutos la encuentra. Apaga la cámara, reemplaza la tarjeta de memoria y las pilas. Vuelve al sendero con huellas de animales, lo recorre dos tercios de milla y lo vuelve a dejar en otro árbol cuidando que el objetivo apunte al sendero. Luego se pone en la mira de la cámara y se mueve hasta escuchar el obturador activado por el sensor de movimiento. 
	
	De esta manera el y otros cientos colaboran con un proyecto de censado de mamíferos del Appalachian Trail desde el sur de Virginia hasta Pennsylvania, y manejan equipamiento, recogen información y anotan observaciones como uno de los varios proyectos que manejan en conjunto agencias gubernamentales, universidades, grupos de conservación ecológica y científicos para supervisar las tendencias ambientales en las cerca de 2175 millas del Appalachian Trail. Tanto el censo de mamíferos como el Appalachian Trail MEGA Transect dependen de voluntarios. Los científicos ciudadanos ayudan a supervisar animales salvajes, plantas u otros objetivos medioambientales y no reciben un pago por su colaboración e incluso a veces no son científicos profesionales. Son aficionados que colaboran con estos proyectos porque disfrutan de estar afuera o porque están comprometidos con los problemas ecológicos y quieren hacer algo al respecto. Generalmente no analizan la información o escriben artículos científicos, pero son esenciales para recolectar la información en la que luego se basan los estudios. 
	
	La Ciencia Ciudadana no es nueva. Lo que es nuevo es la cantidad de voluntarios que se enlistan para colaborar en los estudios, y la amplitud de la información que se les pide recoger. Los investigadores a menudo les piden a los colaboradores que utilicen equipamiento y técnicas sofisticadas para monitorear la calidad del aire y el agua; que documenten el crecimiento, la floración y la muerte de las plantas o que observen cuando las aves y otros animales migran atravesando un área o de qué manera se comportan mientras están en la misma. 
	
%En esta sección podemos ampliar el tema de las características de los proyectos de recolección. Hablar de CBC que abarca una parte muy grande de América del Norte, podemos hablar del proyecto de investigación por el cual AppEAr existe, que es el relevamiento de estuarios en Argentina. Podemos hablar de la app africana para relevar determinadas cosas que hay en terrenos súper inaccesibles, esta app es súper interesante porque es sólo pictográfica, lo que la vuelve amigable para mucha de la gente que no sabe leer ni escribir. %

\subsection{Protocolos de Recolección de Muestras}

	Los protocolos para recolección de muestras en proyectos que incluyen a científicos ciudadanos deben ser simples. Es más sencillo pedir que se identifique, documente o cuente 5 o 10 especies diferentes de plantas que sean fácilmente reconocibles y que serviría para indicar que la especie está presente en el área, en vez de pedir que se identifiquen todas las especies presentes en un área determinada. Una manera de ayudar a los voluntarios es darle libros con guías o material impreso que los ayude.
	
	Como hacemos referencia en el marco teórico, en los primeros estudios la información recolectada por los científicos ciudadanos era imprecisa como para ser utilizada. El principal problema es que la información generada por los voluntarios a veces representan valores en un rango en vez de números específicos, lo que dificulta la detección de cambios en los valores o fundamentar conclusiones. Ahora, los científicos ciudadanos son entrenados para leer instrumentos y recolectar números específicos. Esto debe seguir siendo compatible con protocolos de recolección simples. De esta manera se evitaría caer en el error de solicitarle a los voluntarios la recolección de información muy compleja o detallada. Pero algunas veces se puede solicitar este tipo de complejidad o precisión en la recolección de información. Pero esto no siempre es así. Muchos de los voluntarios que participan en los estudios tienen algún tipo de conocimiento acerca del método científico. De todas maneras debe esperarse información de calidad variada. En esos casos los científicos que dirigen los proyectos deben estar preparados para escrutar cuidadosamente la información obtenida y deben estar dispuestos a descartar información sospechosa o poco confiable.\cite{cohn2008citizen} 
		
\subsection{Diseño de Proyectos de Ciencia Ciudadana}

	El diseño y la implementación de cada proyecto requiere que se tomen decisiones acerca de los intereses de qué público se podría y  debería ser tenido en cuenta a la hora de perseguir intereses, y cómo los objetivos finales, o resultados esperados son definidos. En algunos campos donde puede participar la ciencia ciudadana, las decisiones de diseño son guiadas por teorías de participación, experiencia, o democracia.
	
	A la hora de diseñar un proyecto, una de las primeras preguntas que hay que responder es 'a los intereses de quién/es sirve?'. De esta manera van a quedar un conjunto de opciones resultantes a tomar para implementarlo. Es la negociación entre los intereses científicos y los intereses públicos lo que puede influenciar un rango de resultados potenciales. Public Participation in Scientific Research \citep{shirk2012public} propone un framework para diseñar proyectos de ciencia ciudadana. Los elementos de este framework  son entradas, actividades, salidas, resultados e impacto. 
	
\begin{itemize}
	\item {Entradas}
		Los proyectos de ciencia ciudadana son, por la forma de definirse, un esfuerzo colaborativo, y es por ello que su diseño debe permitir entradas de múltiples constituyentes. Es decir, sus participantes tiene múltiples capacidades y múltiples maneras de brindar su colaboración. Por dar un ejemplo, hay varias maneras de describir una misma imagen. Algunas pueden ser más extensas y detalladas y otras descripciones pueden serlo menos. Estas entradas son los intereses (las esperanzas, deseos, objetivos y expectativas) tanto del público como de la comunidad científica en conjunto a la hora de determinar el objetivo de un proyecto. Podría haber más intereses, pero se considerarán estos dos, las entradas del framework.
		
		Para los voluntarios, sus intereses pueden ser contribuir a generar conocimiento científico, recolectar y diseminar información con respecto a peligros medioambientales, afectar la administración de recursos, proteger [livelihoods], o para satisfacer necesidades que tienen que ver con su identidad personal u objetivos de aprendizaje. Y aunque sería fácil asumir que los intereses de los científicos son principalmente conseguir resultados científicos, bien podrían estar interesados en afectar la educación, la conservación, en manejar la información surgida de sus propias observaciones,o cualquiera de los intereses atribuidos a los voluntarios. Además, estos intereses no son homogéneos incluso dentro del mismo grupo de investigadores o comunidad científica. Y, para tener en cuenta, la línea que divide a los individuos considerados 'científicos' de los que son 'el público' suele no estar bien definida.
		
\end{itemize}	


\section{Ciencia Ciudadana y Tecnología}

	Los teléfonos celulares son dispositivos que están presentes en casi todos los ámbitos y su capacidad de capturar, clasificar y transmitir imágenes, acústica, ubicación y otra información de manera autónoma o interactiva está en crecimiento.
Planteando la arquitectura adecuada, pueden actuar como una red de sensores e instrumentos de recolección de información de localización. 
Esta forma de red de sensores distribuidos puede tener aplicaciones científicas, industriales y militares. Se sabe menos acerca de su función y utilidad en la esfera pública, es decir cuando los que los operan y poseen son usuarios regulares.   Estos sensores en vez de estar en manos de un coordinador central, están siempre bajo el control de sus dueños.

	Solicitar que los dispositivos móviles que ya están [deployados] en el campo, que armen redes de sensores de manera interactiva, participativa y que le permitan a los usuarios del público general y a los profesionales reunir, analizar y compartir información regional. Los micrófonos y cámaras que están presentes en el [handset] del dispositivo pueden registrar información del entorno, mientras se siguen integrando otros sensores de manera inalámbrica. La localización brindada por las antenas de telefonía, el GPS y otras tecnologías proveen información de ubicación y [time-synchronization]. [Las radios wireless] La conexión y el procesamiento que brinda el dispositivo permiten la interacción con la información procesada tanto de manera local como en servidores remotos. 
	
	Los legisladores (creadores de políticas públicas), investigadores y la comunidad utilizan información para comprender y convencer; a mejor calidad de información se consigue una mejor comprensión y políticas significativas. Un ejemplo de ello es la preocupación ciudadana de la ciudad de Los Angeles, cuyos ciudadanos pudieron establecer una relación entre la contaminación del aire y la salud pública. El área contaminación del aire y salud pública es ampliamente estudiada en todo el país utilizando métodos de recolección de datos de manera top-down y bottom-up, y se estima que una red de recolección de datos aportaría una contribución valiosa. El artículo "Elemental Carbon and PM2.5 Levels in an Urban Community Heavily Impacted by Truck Traffic" documenta un estudio hecho de manera conjunta entre la universidad y la comunidad acerca de circulación desproporcionada de tránsito pesado y las tasas de asma registradas. De esta manera, los investigadores de la universidad local llevaron adelante el monitoreo de partículas con equipamiento especializado y con la colaboración de la comunidad para documentar el tráfico comercial de camiones, y eventualmente relacionar la densidad del tráfico con los niveles de partículas saturados de diesel y evidenciar el uso ilegal de rutas no comerciales; información que pueden influir sobre políticas públicas y de salud.
	
	Una arquitectura que permita el participatory sensing puede mejorar y sistematizar la metodología existente incrementando la cantidad, calidad y credibilidad de la información reunida por la comunidad. Implementando protocolos de recolección de información adaptativos basados en estadísticas locales o globales, el participatory sensing facilitaría datos confiables mediante geolocalización, o habilitando la subida automática de información desde equipamiento especializado que todavía no esté conectado a una red. En proyectos con diseño profesional de recolección de datos, se puede incrementar la información recabada distribuyendo observaciones previas hechas con aplicaciones que estén en distribuidas entre los participantes; por ejemplo, conteos previos de cantidad de camionetas en el tránsito del lugar.
	
	Entonces el protocolo adaptado de recolección de información, la ayuda brindada desde la aplicación, la geolocalización y la hora de la toma de la muestra incrementa la confiabilidad de la información generada. Con esta información también se podría detectar en qué lugares o momentos falta recolectar información, y podría pedirle a los participantes que toma la muestra, de ser posible, a una hora determinada o en un lugar en particular. La utilización de auriculares y micrófonos (handset) se abren las posibilidades de capturar información relativa a la exposición individual y la actividad. Además de la recolección de datos interactiva, muestras de audio tomadas en forma periódica del medio que rodea al usuario pueden ser analizadas para detectar si el dispositivo está en medio de un embotellamiento, uno de los motivos principales de la exposición a partículas de diesel. Los dispositivos móviles pueden ser utilizados para detectar patrones de actividad de las personas para establecer la correlación entre los datos recolectados por agencias gubernamentales y obras sociales (healthcare providers); dicha información podría ayudar a los médicos a analizar patrones de exposición a partículas de pacientes, y también en el análisis de actividades y exposición de comunidades. \cite{burke2006participatory}
	
	
	[Pasa a explicar privacidad a la hora de compartir información]%
\chapter{Framework para Proyectos Android de Ciencia Ciudadana}

\section{Frameworks}

	Un framework es un diseño abstracto para un tipo particular de aplicación,y generalmente consiste de un conjunto de clases. Estas clases pueden pertenecer a una librería o pueden ser específicas de la aplicación. Lo frameworks se pueden construir sobre otros frameworks compartiendo clases abstractas.
	
	Brindan una manera de reutilizar código que es resistente frente a los intentos más comunes de reutilización. Los componentes independientes de una aplicación pueden ser reutilizados fácilmente, pero poder reutilizar la estructura que mantiene a los componentes juntos generalmente es posible copiando y editando. A diferencia de los programas esqueletos, que es el enfoque más convencional para reutilizar este tipo de código, los frameworks facilitan la tarea de asegurar que bajo requerimientos cambiantes la consistencia de todos sus componentes se va a mantener.
	
	Como hacen posible la reutilización en la granularidad más alta, no es ninguna sorpresa que diseñar un buen framework es mucho más difícil que diseñar una buena clase abstracta. También, tienden a ser específicos de la aplicación, a integrarse a otros frameworks mediante compartir clases abstractas, y a tener algunas clases abstractas especializadas para el framework. Diseñar un framework requiere experiencia y experimentación al igual que lo requiere el diseño de las clases abstractas de sus componentes.

\subsection{Frameworks de Caja Blanca y de Caja Negra}
\subsubsection{Frameworks de Caja Blanca}

	Una de las características importantes de un framework es que los métodos definidos por el usuario para extender el comportamiento van a ser invocados desde el interior del framework más que del código de la aplicación del usuario. A menudo hace las veces de programa principal coordinando y secuenciando las actividades de la aplicación. Esa inversión de control le permite al framework servir como esqueleto extensible. El código brindado por los usuarios en los métodos extienden el algoritmo genérico del framework para una aplicación en particular. 
	
	El comportamiento específico de una aplicación que utiliza un framework usualmente se define agregando métodos a las subclases o a una o más de sus clases. Cada método que se agrega a una subclase debe continuar con las convenciones internas que adoptan las superclases. Este tipo de framework se denomina de caja blanca (white-box) porque debe comprenderse cómo está implementado para poder utilizarlo.
	
	El principal problema de los frameworks de caja blanca es que cada aplicación requiere la creación de una numerosa cantidad  de subclases. Y aunque muchas de estas subclases creadas son simples, es su número lo que para un desarrollador con poca experiencia puede volver difícil comprender el diseño de una aplicación los suficiente como para modificarla.
	
	Un segundo problema es que un framework de caja blanca puede ser difícil de aprender a utilizar, ya que entender cómo se utliza es lo mismo que entender cómo está construido.
	
\subsubsection{Frameworks de Caja Negra}

	Otra manera de especializar un framework es incluir en él un conjunto de componentes que sean los que proveen el comportamiento específico de la aplicación. Cada uno de estos componentes debe entender un protocolo en particular. Todos o la mayoría de los componentes pueden tomarse de una librería de componentes. La interfaz entre componentes pueden se definidas con un protocolo, de esta manera el usuario sólo necesita entender la interfaz externa de los mismos. Este tipo de framework se denomina de caja negra.
	
	Los frameworks de caja negra son más fáciles de aprender a utilizar que los de caja blanca, pero son menos flexibles. 
	
	Una manera de caracterizar la diferencia entre un framework de caja blanca y uno de caja negra es observar que en el de caja blanca el estado de cada instancia está disponible de manera implícita en todos los métodos del framework, casi como las variables globales de Pascal. En un framework de caja negra, cualquier información que se pase a las partes constituyentes del framework debe pasarse de manera explícita. Un framework de caja blanca utiliza las reglas de alcance intra-objeto para evolucionar sin forzarlo a subscribirse a un protocolo explícito, rígido que podría restringir de manera prematura el proceso de diseño.
	
 \cite{johnson1988designing}

\subsection{Jerarquías y Composición}
De qué manera los frameworks permiten que los usuarios los configuren y les pongan comportamiento

\subsection{El Entorno de Android}
Herencia de las clases Activity y Fragment como principal herramienta para implementar código nativo Android.

\subsection{Estado del Arte}
\subsubsection{Sensr: un framework flexible para crear herramientas de recolección de datos con dispositivos móviles para ciencia ciudadana}

	Los dispositivos móviles son ideales para que las personas puedan de manera espontánea recolectar información. Sin embargo, esa simplicidad yace sobre una base que requiere fuertes conocimientos técnicos una infraestructura compleja. Por lo tanto, construir aplicaciones móviles implican una inversión que puede ser limitante para organizaciones pequeñas. Sensr es una herramienta que permite que personas que no son desarrolladoras tengan la posibilidad de crear herramientas que permitan la recolección de información para proyectos de ciencia ciudadana con dispositivos móviles. Esta herramienta aprovecha que el proceso y la estructura de la información en las actividades de recolección de datos de los proyectos de ciencia ciudadana son similares independientemente del dominio o la diversidad de los mismos. Sensr combina un ambiente de programación gráfico con una aplicación móvil para que las personas que no necesariamente poseen conocimientos técnicos puedan construir herramientas de recolección de información para dispositivos móviles y administrar la información recabada de manera colectiva.
	
	De esta manera, una persona que necesitan reunir información puede ser el autor de una campaña de ciencia ciudadana en el sitio de Sensr. La campaña es desplegada en la aplicación móvil de Sensr, y sus usuarios se pueden suscribir y contribuir a la campaña con los datos recolectados. Esta herramienta pretende simplificar de manera radical el proceso de crear una herramienta para dispositivos móviles que permita recolectar información y que sea de utilidad en una amplio conjunto de dominios de ciencia ciudadana. Los autores sólo necesitarían completar la descripción del proyecto y diseñar las plantillas o formularios que permitan el ingreso de los datos antes de incluir su proyecto en Sensr y ser distribuido de forma masiva. De esta manera, los autores se liberarían de las preocupaciones acerca de los requerimientos técnicos y las restricciones de la infraestructura.
	
	La falta de expertos técnicos y de recursos son a menudo los mayores obstáculos a la hora de desarrollar una aplicación móvil. Los grupos que quieren desarrollar una aplicación móvil de ciencia ciudadana a menudo son organizaciones sin fines de lucro o pequeñas organizaciones regionales que no poseen ni los recursos económicos ni los expertos técnicos que necesitan para desarrollar o mantener ese tipo de aplicaciones. Y además de la programación en si, la administración de los datos recolectados también representan un desafío, ya que estas mismas organizaciones tampoco poseen los servidores para almacenar o analizar el volumen de datos que puedan ser recolectados. El monitoreo participativo es un paradigma computacional que permite la recolección por parte de los voluntarios de información que se encuentra diseminada. Permite que el creciente número de usuarios de teléfonos móviles puedan compartir la información adquirida mediante los sensores de sus dispositivos en variados dominios. 
	
	Los investigadores han explorado la utilización de plataformas existentes como una alternativa para Y aunque varias de ellas son robustas y flexibles, la mayoría necesita de habilidad para programar y/o conocimiento de infraestructura en mayor o menor medida. Aunque por ejemplo el Proyecto Noah y EpiCollect son dos ejemplos claros de plataformas que soportan autoría de aplicaciones sin necesidad de programación. \cite{kim2013sensr}
\chapter{Samplers: Framework Android}

\section{Propuesta general}

\subsection{Descripción del problema}
Nativo en vez de web porque permite la toma de muestras sin conexion a internet y aprovecha mejor los recursos del celular (camara, gps, etc.).

En un principio se pensó para que un científico pudiera crear su propia aplicación móvil de ciencia ciudadana sin tener conocimientos de programación. La idea inicial era que mediante una aplicación web el científico pudiera armar el protocolo de recolección de las muestras (el Workflow en Samplers) de manera visual e intuitiva, y se generara un archivo de configuración para Samplers. Con el archivo de configuración se pasaría a Samplers y se generaría la aplicación móvil. Pero de esta forma el alcance de la tesis era muy grande, por lo que se decidió quitar la parte de la aplicación web y suponer que que el archivo de configuración ya viene armado.

\subsection{Alcance de la solución propuesta}
En un principio se pensó para que un científico pudiera crear su propia aplicación móvil de ciencia ciudadana sin tener conocimientos de programación. La idea inicial era que mediante una aplicación web el científico pudiera armar el protocolo de recolección de las muestras (el Workflow en Samplers) de manera visual e intuitiva, y se generara un archivo de configuración para Samplers. Con el archivo de configuración se pasaría a Samplers y se generaría la aplicación móvil. Pero de esta forma el alcance de la tesis era muy grande, por lo que se decidió quitar la parte de la aplicación web y suponer que que el archivo de configuración ya viene armado.

Como se mencionó antes, se requerían aplicaciones nativas pero abarcar los 3 sistemas operativos móviles más usados en ese momento (Android, iOS y Windows Phone) era mucho para el alcance de esta tesis, por lo que se decidió optar por Android, que era el sistema operativo móvil más usado en ese momento. Según una estadística de Gartner sobre las ventas de smartphones a nivel mundial en el último trimestre de 2016\cite{gartner}, mas del 80\% de las mismas fueron de celulares con Android.


\subsection{Descripción de la propuesta}
Samplers es un framework que permite construir, de manera sencilla, aplicaciones Android para recolectar muestras en proyectos de Ciencia Ciudadana. Brinda una solución simple al problema de la recolección de la muestra aprovechando las funcionalidades de los dispositivos móviles.


Para desarrollar aplicaciones móviles para la plataforma Android, el entorno de desarrollo integrado (IDE) oficial es Android Studio\cite{androidStudio}, por lo que Samplers se apoya sobre el mismo. Android Studio ha sido publicado de forma gratuita bajo Licencia Apache 2.0 y está disponible para las plataformas Microsoft Windows, MacOS y GNU/Linux.

De este modo, Samplers sería un framework que funcionaría sobre Android Studio, que recibiría un archivo de configuración y generaría una aplicación lista para ejecutarse en un dispositivo móvil con Android.

Para el archivo de configuración se eligió el formato JSON porque nos pareció más fácil de escribir/leer a mano (que XML por ejemplo, que también fue evaluado), suponiendo que el científico tuviese que armarlo a mano. En el mismo vendría el workflow, que representa el protocolo de recolección de las muestras, y las demás configuraciones necesarias para armar la aplicación.

El workflow estaría compuesto por los diferentes pasos (Steps en Samplers) necesarios para tomar la muestra con la aplicación. Estos pasos podrían ser:
\begin{itemize}
\item capturar una foto, un video o un audio
\item tomar la posición del GPS o grabar un recorrido con el GPS
\item contestar una pregunta con respecto a la muestra. Esta pregunta puede tener una o múltiples respuestas posibles.
\item introducir anotaciones (texto)
\item seleccionar una fecha o una hora
\item mostrar información (texto).
\end{itemize}

Además de éstos pasos, se debería contemplar algún mecanismo para poder mostrar ayuda para el usuario final de la aplicación (el científico ciudadano) que sirviera de orientación para tomar la muestra. Se pensó en poder mostrar ayuda asociada a un paso (Step) en particular y una ayuda general en la pantalla principal, y se optó por usar archivos HTML para mostrar las mismas por la variedad de opciones y posibilidades que conllevan, y por su simpleza para mostrarlos.

La aplicación generada servirá para tomar muestras siguiendo el protocolo de recolección especificado y las almacenará y empaquetará en el dispositivo móvil hasta que pueda ser enviada a un servidor web, el cual vendrá establecido en el archivo de configuración.

También se pensó en proveer algún mecanismo de autenticación, para poder identificar al usuario que toma las muestras, y de esta forma poder darle una devolución mostrándole información, o incentivarlo para que siga participando del proyecto a través de gamificación por ejemplo.
Se pensó en poder iniciar sesión con alguna red social, como por ejemplo Facebook, Twitter, Instagram, Google+, etc. y como en los dispositivos móviles con sistema operativo Android se requiere de una cuenta de Google para acceder a muchas de las funcionalidades, como por ejemplo la Play Store (la tienda de Android para descargar aplicaciones), se decidió incluir autenticación con Google y dejar preparado para que el usuario pueda incluir su propio sistema de inicio de sesión (por ejemplo con usuario y contraseña). Samplers también debería quedar preparado para agregar autenticación con otras redes sociales en un futuro.

Como Samplers se pensó como un proyecto dentro de Cientópolis\cite{cientopolis} el mismo debería ser open source, por lo que se decidió usar un repositorio público en GitHub\cite{github} para dejar el código fuente disponible para todo el mundo.

Si bien, como se mencionó antes, Samplers se pensó para que un científico pudiera crear su propia aplicación móvil de ciencia ciudadana de manera sencilla y sin tener conocimientos de programación, también se pensó para que un usuario programador pudiese modificar la aplicación generada, personalizando la pantalla principal, modificando la vista de un Step, o agregando uno propio que no esté en los que provee Samplers. También para que pudiese ser incluido en una aplicación ya comenzada o una que recién arranca.

Por lo tanto, Samplers funciona de dos formas:
\begin{itemize}
\item Instanciando y/o extendiendo las clases necesarias para armar la aplicación: es decir, instanciando un Workflow con instancias de los Steps que provee Samplers (o extendiendo los propios) para pasárselo a la una instancia de la clase TakeSampleActivity que es la encargada de ejecutarlo y generar las muestras (Samples).
\item Usando el generador de clases de Gradle: que básicamente lo que hace es extender e instanciar las clases necesarias para generar la aplicación en base al archivo de configuración JSON pasado. 
\end{itemize}



**FALTA DESARROLLAR**
** Configurable o Extensible nosotros creo que hicimos los dos... ------------------------------------- **
** Segun Spotters: **

Al momento de comenzar a trasladar los conceptos de Spotters a un prototipo, se presentó  como  interrogante  cuál  sería  la  forma  que  tendría  el  programador  de  utilizar el  framework  para  poder  incluir  las  cuestiones  particulares  de  su  dominio.  Analizando diversas alternativas, se encontraron dos caminos posibles: Configuración y Extensión. 
La  alternativa  de  Configuración  consiste  en  proveer  un  mecanismo  que  permita derivar el comportamiento de la instancia de Spotters a partir de información contenida en  un  archivo  con  directivas  o  una  base  de  datos.  La  alternativa  de Extensión,  consiste en  proveer  un  conjunto  de  clases  abstractas  que  el  programador  deba  sub  clasificar para crear las clases concretas necesarias para el funcionamiento de su instancia, tales como las preguntas que se realizan para la clasificación y los pasos del tutorial.

** ------------------------------------------------------- **


\subsection{FrozenSpots y HotSpots}
** OJO: ROBADO DE SPOTTERS!! ------------------------------------- **

Podemos definir un framework como una aplicación que por sí misma no tiene una funcionalidad concreta y no se encuentra finalizada, pero que fue pensada y desarrollada para que pueda ser ampliada fácilmente, y de esta manera, resuelva problemas determinados de un dominio específico; es decir, poder construir múltiples aplicaciones de un dominio específico a partir de una aplicación ya existente que sirve como punto de inicio, y que ya resuelve aspectos y requerimientos comunes al dominio donde se está trabajando.

El framework está compuesto por dos conceptos muy importantes que definen qué aspectos pueden editarse del framework (HotSpots) y qué aspectos son estáticos y por lo tanto el programador no podría modificar, ya que sirven de base común a todas las aplicaciones (FrozenSpots).

** ------------------------------------------------------- **

\subsubsection{FrozenSpots}

\begin{itemize}
	\item Solo sirve para dispositivos móviles con sistema operativo Android.
	\item El uso de la activity TakeSampleActivity es casi obligatorio... si bien se podría usar otra, hay que entender más como funciona el framework para poder hacer una propia.
	\item La muestra se envía en un archivo comprimido, en formato ZIP. Dentro esta el objeto muestra en formato JSON y los archivos multimedia.
	\item Las muestras se envían a través de internet, mediante un mensaje HTTP POST.
	\item Para definir una nueva vista (fragment) para un step, se debe heredar de StepFragment.
\end{itemize}

\subsubsection{HotSpots}

\begin{itemize}
	\item Se puede configurar un workflow con la cantidad de steps que se quiera. Se pueden armar bifurcaciones usando el step SelectOneStep o creando uno propio.
	\item Se puede definir una nueva vista (fragment) personalizada para los steps provistos por Samplers.
	\item Se puede definir nuevos steps personalizados, solo tienen que implementar la interfaz Step.
	\item Se puede definir un archivo para mostrar ayuda (en HTML) para cada step y una general.
	\item Se puede definir el servidor al cual se envían las muestras.
	\item Se puede configurar para que se use autenticación, requerida u opcional para el usuario final de la aplicación, con Google y también se pueden agregar otros métodos propios de autenticación.
	\item Se puede configurar para soportar múltiples idiomas.
\end{itemize}





\section{Modelo del Framework}
En esta sección se explica el modelo de las clases que permiten que se cree una aplicación móvil con Samplers y su funcionamiento.

\subsection{Workflow, Step, StepFragment, StepResult y Sample: el core del framework}
Estas clases conforman el corazón del framework, y son las necesarias para poder definir la muestra y el protocolo de recolección de las mismas.

A continuación se muestra el diagrama de clases (Figura \ref{fig:umlFrameworkCore}).

\begin{figure}[H]
  \centering
    \includegraphics[scale=0.4]{05-implementacion/FrameworkCore.png} 
   \caption{Diagrama de clases del core del framework}
   \label{fig:umlFrameworkCore}
\end{figure}


\subsubsection{Workflow: el Protocolo de recolección de las muestras}
La clase Workflow representa el protocolo de recolección de las muestras. Está formado por una colección de Steps, que representan los pasos a seguir para completar dicho protocolo, y un Step inicial que indica el inicio del mismo (el primer Step a ejecutar).

El Workflow es el encargado de llevar el estado del paso en el que se encuentra, y posee operaciones para obtener el siguiente Step (\textit{nextStep()}  que depende del Step actual) y el Step anterior (\textit{previuosStep()}  para lo que maneja una lista a modo de stack).

El Workflow es secuencial, puede ir hacia adelante o volver hacia atrás, pero esto no impide que pueda tener bifurcaciones, y así formar diferentes caminos para tomar la muestra. Por ejemplo, se le podría preguntar al científico ciudadano si observa una característica particular al momento de tomar la muestra, y si la observa solicitarle que tome una foto de la misma, pero en caso contrario se puede omitir el paso de la foto.

\subsubsection{Step: el Paso}
Un Step representa un paso dentro del protocolo de recolección de las muestras (Workflow).

Se definió como una interfaz, ya que su comportamiento depende de la implementación de cada tipo de Step que se defina. Con esto también se evita que los Steps creados por los usuarios programadores tengan que heredar de una clase en particular, y solo necesiten implementar la interfaz.

El Step tiene asociado un StepFragment (la vista y controlador) que es el encargado de ejecutar el Step para generar un resultado (StepResult). El Step básicamente tiene los parámetros o información para que pueda ser ejecutado por el StepFragment, y no tiene mucho comportamiento.

Por ejemplo, para el caso en que el paso sea contestar una pregunta, el Step tendrá la pregunta en sí (un String) y una colección con las posibles respuestas (id y descripción) .

El Step conoce cuál es el siguiente Step (su id) a ejecutar aunque a veces éste depende del StepResult generado, como es el caso de SelectOneStep en el que el siguiente paso depende de la opción seleccionada, y con esto se pueden crear bifurcaciones en el Workflow.

Samplers provee los siguientes Steps:
\begin{itemize}
	\item InformationStep: Muestra una información (texto) al usuario.
	\item PhotoStep: Permite tomar una foto con la cámara del dispositivo móvil.
	\item SoundRecordStep: Permite grabar un sonido con el micrófono del dispositivo móvil.
	\item SelectOneStep: Muestra una pregunta con varias opciones como respuesta, de las cuales solo se puede seleccionar una sola.
	\item MultipleSelectStep: Muestra una pregunta con varias opciones como respuesta, de las cuales solo se pueden seleccionar varias opciones.
	\item LocationStep: Permite tomar la geo posición del dispositivo móvil.
	\item RouteStep: Permite grabar un recorrido usando el GPS del dispositivo móvil.
	\item InsertTextStep: Permite ingresar un texto.
	\item InsertDateStep: Permite seleccionar una fecha.
	\item InsertTimeStep: Permite seleccionar una hora.
\end{itemize}

El detalle de estos Steps y su funcionamiento se explican en la sección \ref{sec:steps_detallados}.

Si bien estos son los Steps que provee Samplers de manera predeterminada, un usuario programador podría definir y agregar sus propios Steps, junto con sus StepFragments y StepResults como se explica en la sección \ref{sec:definir_steps}.

\subsubsection{StepFragment: la vista y controlador del Step}
El StepFragment es el encargado de ejecutar un Step y generar un StepResult con la interacción del usuario final (el científico ciudadano). Un StepFragment recibe un Step, que generalmente contiene los parámetros necesarios para ejecutarlo, y en base a eso muestra un fragment para que, interactuando con el científico ciudadano, poder obtener un resultado (StepResult) para la muestra (Sample). 

Por ejemplo, para el caso en que el paso sea contestar una pregunta, se mostrará la pregunta y se listarán las posibles respuestas en forma de radio-buttons, si solo se admite seleccionar una única respuesta, o en forma de check-buttons, si se permite seleccionar más de una.

Por cada tipo de Step, debe haber una clase StepFragment que sepa ejecutar ese Step y generar el StepResult asociado.

Es una clase abstracta, ya que su comportamiento depende de la implementación de cada tipo de StepFragment que se defina. Se definió así y no como una interfaz por la necesidad de que heredara de Fragment porque así lo necesita la activity TakeSampleActivity, que es la encargada de ejecutar el Workflow para obtener la muestra.

Por cada Step que provee Samplers de manera predeterminada, también se provee un StepFragment que ejecuta cada Step. Asimismo, un usuario programador también puede desarrollar un StepFragment propio para alguno de los Steps provistos por Samplers, como se explica en la sección \ref{sec:definir_steps}



\subsubsection{StepResult: el Resultado de la ejecución de un Paso}
El StepResult representa el resultado que se obtiene de ejecutar un Step. Contiene los datos obtenidos de ejecutar el Step asociado. Cada ejecución de un mismo Step, puede generar un StepResult diferente.

Por ejemplo, para el caso en que el paso (Step) sea contestar una pregunta, el StepResult contendrá la respuesta seleccionada, si solo se admite seleccionar una única respuesta, o una colección de respuestas, si se permite seleccionar más de una.

Se definió como una interfaz, ya que su comportamiento depende de la implementación de cada tipo de Step que se defina.

Un StepResult tiene asociado el Id del Step en base al cual se generó. También puede tener asociado un archivo multimedia, como por ejemplo en los casos de PhotoStep y SoundRecordStep que guardan una foto y un archivo de sonido respectivamente.


\subsubsection{Sample: la Muestra}
La clase Sample representa una muestra tomada a partir de seguir los pasos (Steps) del protocolo del recolección (Workflow). Contiene los resultados (StepResult) de la ejecución de cada paso. Cada ejecución del workflow puede generar una colección de StepResults diferente.

También guarda fecha y hora de inicio y finalización. Esto es útil para sacar una estadística de cuanto tarda un científico ciudadano en recolectar una muestra y así poder analizar optimizaciones para la aplicación final.

Una vez recolectadas, las muestras se guardan el dispositivo móvil y son enviadas a través de Internet a un servidor web previamente configurado.

\subsection{TakeSampleActivity: Toma de la Muestra}
La clase TakeSampleActivity es la encargada de ejecutar el Worflow para tomar la muestra. Es una activity que recibe un objeto Workflow como parámetro y va iterando sobre los Steps del mismo. A cada Step le pide su StepFragment y lo muestra en pantalla para que, interactuando con el científico ciudadano, se genere el StepResult para la muestra (Sample). Una vez finalizado el Workflow, guarda la muestra, controla si se puede enviar la mima y finaliza.



\subsection{Persistencia local}
Las muestras se guardan en el dispositivo móvil en un archivo JSON, junto con los archivos multimedia que pudiera tener, dentro de un directorio por cada muestra. Las mismas se guardan dentro del directorio samples en el almacenamiento interno del dispositivo.

Se eligió el almacenamiento interno para guardar las muestras porque no se necesitan permisos especiales para acceder al mismo y, de forma predeterminada, los archivos que se guardan en el mismo son privados para la aplicación y otras aplicaciones no pueden tener acceso a ellos (tampoco el usuario). Cuando el usuario desinstala la aplicación, estos archivos se quitan\cite{androidInternalStorage}.

Por ejemplo, una muestra con id 123456 y 2 fotos se guarda de la siguiente manera:
\begin{lstlisting}[language=Java, frame=tlb]
/samples/		// Directorio de muestras
  sample_123456/	// Directorio de la muestra con id:123456 
    sample_123456.json	// Objeto Sample en formato JSON
    1524437599776.jpg	// Archivo de foto 1
    1524441664170.jpg	// Archivo de foto 2
\end{lstlisting}

Para pasar la muestra a un archivo JSON se usó Gson\cite{gson}, una librería de Google distribuida bajo licencia Apache 2.0 que convierte objetos Java a JSON y viceversa de manera muy sencilla, con métodos \textit{toJson()} y \textit{fromJson()} para convertir hacia y desde JSON respectivamente.

\subsection{Envío de Muestras a Servidor Web}

Una vez guardadas localmente, las muestras se envían al servidor web previamente configurado, mediante un mensaje HTTP POST. Para ello se usó la librería OkHttp\cite{okhttp}, distribuida por Square Inc. bajo licencia Apache 2.0, que resuelve de manera sencilla y muy eficiente el envío de mensajes HTTP en Android.

Por cada muestra se envía el objeto Sample en formato JSON junto con los archivos multimedia que pudiera tener, todo comprimido en un solo archivo ZIP. Para comprimir las mismas se usaron las librerías estándares de Java (java.util.zip) que proporciona clases para leer y escribir archivos ZIP y GZIP estándares.

Las muestras se envían automáticamente cuando se detecta conexión wi-fi o a petición del usuario.

Para el envío automático, se controla al momento de guardar la muestra si hay conexión wi-fi y se intenta enviar la muestra; caso contrario queda pendiente de envío y se intenta enviar cuando se detecta conexión wi-fi. Para el caso que es a petición del usuario, se listan las muestras tomadas en la SamplesListActivity con un botón que permite el envío de la misma usando cualquier tipo de conexión a Internet.

\subsection{Autenticación}
Samplers provee un sistema de autenticación (login), el cual se puede habilitar o no de acuerdo a las necesidades del proyecto. Si se habilita el mismo puede ser opcional, permitiendo al científico ciudadano tomar y enviar muestras habiéndose o no autenticado, o puede ser obligatorio, requiriendo que el científico ciudadano se autentifique para poder tomar y enviar las muestras.

De manera predeterminada Samplers incluye autenticación con la API de Google, es decir que el científico ciudadano puede usar su cuenta de Google que tiene configurada en el dispositivo móvil Android para autenticarse en la aplicación.

Además, un usuario programador puede definir su propio método de autenticación, ya sea con usuario y contraseña o usando alguna otra API de redes sociales por ejemplo, como son Facebook, Twitter, etc. como se explica en la sección \ref{sec:usar_auth_propia}.

Cuando se envía la muestra también se incluyen dos parámetros que son el id del usuario autenticado y el método usado para autenticar (que puede ser google o uno personalizado). Se envían de esta forma porque al momento de tomar la muestra el científico ciudadano puede no estar autenticado, y autenticarse antes de enviarlas.

\subsection{Resto de la UI?? ANALIZAR}

SamplersMainActivity

SamplesListActivity

HelpActivity


\section{Instalación y uso del framework} \label{sec:instalacion_uso}
En esta sección se describe como instalar el framework en Android Studio y como instanciarlo para su uso, brindando algunos ejemplos concretos.

\subsection{Instalación}
A continuación se detallan los requisitos y los pasos a seguir para la instalación del framework en Android Studio.

\subsubsection{Requerimientos mínimos:}

\begin{itemize}
\item Android Studio (Java): Si bien esta pensado para y probado en Android Studio, podria funcionar bien en otro entorno que use Java y deje importar archivos Android Archive (.aar).
\item Android SDK API17: Android 4.2 (Jelly Bean) o superior.
\end{itemize}

\subsubsection{Pasos para la Instalación:}

\begin{enumerate}
	\item Crear en Android Studio un nuevo proyecto vacío (sin ninguna Activity).
		\begin{itemize}
		\item Seleccionar API17 o superior como versión mínima de Android SDK
		\end{itemize}
	\item Importar la librería del framework en el proyecto creado
		\begin{itemize}
		\item Descargar la última versión de \textbf{samplersFramework.aar} desde https://github.com/cientopolis/samplers/releases/
		\item Importar la librería al proyecto: File -$>$ New -$>$ New Module -$>$ Import .JAR/.AAR Package
		\end{itemize}
	\item Agregar el repositorio de Google
		\begin{itemize}
			\item En el archivo build.gradle del proyecto agregar: 
			\begin{lstlisting}[language=XML, frame=single]
allprojects {
    repositories {
        jcenter()
        google()
    }
}
			\end{lstlisting}	
		\end{itemize}
	\item Agregar las dependencias necesarias:
		\begin{itemize}
			\item En el archivo build.gradle de la aplicación agregar: 
			\begin{lstlisting}[language=XML, frame=tlb]
dependencies {
  // acá estarían las dependencias predeterminadas creadas por Android Studio:
  // ...

  // si no son agregadas automáticamente, agregar las siguientes dependencias:
  // (se deberían agregar las de la ultima versión disponible)
  compile 'com.android.support:design:24.2.1' 
  compile 'com.android.support.constraint:constraint-layout:1.0.2'

  // para usar mapas y los servicios de geolocalización se deben agregar las 
  // siguientes dependencias (usar la ultima versión disponible)
  compile ('com.google.android.gms:play-services-location:12.0.1')
  compile ('com.google.android.gms:play-services-maps:12.0.1')
  
  // para usar autenticación con Google se deben agregar las 
  // siguientes dependencias (usar la ultima versión disponible)  
  compile ('com.google.android.gms:play-services-auth:12.0.1')

  // agregar la dependencia del framework Samplers
  compile project(":samplersFramework")
}
			\end{lstlisting}	
		\end{itemize}	

	\item Instanciar:
		\begin{itemize}
		\item La instanciación puede ser manual o usando el generador de clases en Gradle como se explica en la siguiente sección.
		\end{itemize}
\end{enumerate}	

\subsection{Instanciación}
Una vez instalado el framework el siguiente paso es instanciarlo para usarlo. La instanciación puede ser manual o usando el generador de clases de Gradle.

\subsubsection{Instanciación manual}
Básicamente se tiene que crear un objeto Workflow, agregarle los objetos Steps, y llamar a la activity TakeSampleActivity pasándole el workflow como parámetro.
También es necesario establecer la configuración general en el método onCreate de la activity principal (main activity).
Se puede usar una activity principal propia o se puede heredar de SamplersMainActivity. En ambos casos se debe hacer lo siguiente:
\begin{itemize}
	\item Establecer la configuración general en el método onCreate de la activity principal:
		\begin{lstlisting}[language=Java, frame=tlb]
NetworkConfiguration.setURL("http://192.168.1.10/samplers/upload.php");
NetworkConfiguration.setPARAM_NAME_SAMPLE("sample");

// Opcional, si se desea usar autenticación (*)
NetworkConfiguration.setPARAM_NAME_USER_ID("user_id");
NetworkConfiguration.setPARAM_NAME_AUTHENTICATION_TYPE("authentication_type");

AuthenticationManager.setAuthenticationEnabled(true);
AuthenticationManager.setAuthenticationOptional(true);
		\end{lstlisting}
(*) Ver la sección \ref{sec:usando_autenticacion} para más detalles sobre como usar Autenticación.


	\item Crear un Workflow. Si se está heredando de SamplersMainActivity se debe hacer sobreescribiendo el método getWorkflow.
		\begin{lstlisting}[language=Java, frame=tlb]
@Override
protected Workflow getWorkflow() {
  Workflow workflow = new Workflow();
    	
  Step step = new InformationStep(2,"Por favor tome una foto de su gato", null);
  workflow.add(step);
    	
  step = new PhotoStep(1,"Bienvenido a la app de prueba", 2);
  workflow.add(step);
    	
  return workflow;
}		
		\end{lstlisting}
Nota: en el ejemplo anterior se muestra un Workflow que tiene dos Steps. El primero muestra un mensaje de bienvenida y el segundo pide para tomar una foto. Para ver los distintos Steps que se pueden usar vea la sección \ref{sec:steps_detallados}.

	\item Iniciar la activity TakeSampleActivity. Si se está heredando de SamplersMainActivity esto se hace automáticamente en el método onClick del botón "tomar muestra". De lo contrario, deberá iniciarla de la siguiente forma, en el método onClick de un botón por ejemplo:
		\begin{lstlisting}[language=Java, frame=tlb]
@Override
public void takeSampleClick(View view) {
  Workflow workflow = getWorkflow();

  Intent intent = new Intent(this, TakeSampleActivity.class);        
  intent.putExtra(TakeSampleActivity.EXTRA_WORKFLOW, workflow);
  startActivity(intent);
}		
		\end{lstlisting}

\end{itemize}


\subsubsection{Instanciación usando el  generador de clases de Gradle}

Básicamente, el generador de clases de Gradle se encarga de hacer una instanciación manual a partir de un archivo de configuración (JSON). Está pensado para desarrolladores que no tienen muchos conocimientos en Android, o para servir de interfaz entre una aplicación que genere apps a través de Samplers (***esto hay que escribirlo mejor***).

Los pasos para usar el generador de clases de Gradle son:
\begin{enumerate}
	\item Crear un archivo JSON con el nombre SamplersConfig.json
		\begin{itemize}
			\item El formato y las opciones están explicadas sección \ref{sec:archivo_config_detallado}.
			\item Para validar sintaxis se puede usar por ejemplo el validador online https://jsonformatter.curiousconcept.com que es gratuito.
			\item Al final de esta sección se provee un archivo de ejemplo.
		\end{itemize}
		
	\item Copiar el archivo creado en el item anterior al directorio raíz del proyecto Android creado.
	
	\item Descargar la última versión de los archivos \textbf{samplers.gradle} y \textbf{samplersclassgenerator.jar} del repositorio de Samplers (https://github.com/cientopolis/samplers/releases/) y copiarlos también al directorio raíz del proyecto Android.
	
	\item Enlazar el archivo samplers.gradle en el archivo \textbf{build.gradle} de la aplicación:
		\begin{itemize}
			\item Android Studio crea por defecto dos archivos build.gradle, uno a nivel de aplicación y otro a nivel de proyecto. Debe usarse el de aplicación.
			\item Al final del archivo build.gradle de aplicación agregar:
\begin{lstlisting}[language=XML, frame=tlb]
apply from: '../samplers.gradle'
\end{lstlisting}
			\item Al guardar los cambios, Android Studio sugerirá hacer una sincronización del proyecto, hacerla. Esto generará en la aplicación una activity llamada MyMainSamplersActivity en base a las opciones configuradas en el archivo SamplersConfig.json.
			\item Si se necesita volver a generar esta activity (si se quieren modificar algunas opciones por ejemplo ) se puede eliminar la misma, hacer las modificaciones en el archivo SamplersConfig.json y volver a generar el proyecto (en el menu Build -$>$ Make Project)
		\end{itemize}
		
	\item Eliminar o personalizar el archivo \textbf{style.xml} que está en \textbf{res/values} en la aplicación
	
	\item Ejecutar la aplicación y listo.

\end{enumerate}


\subsection{Secciones del Archivo} \label{sec:archivo_config_detallado}

El archivo SamplersConfig.json es un archivo JSON con 3 objetos:
\begin{itemize}
	\item El objeto \textbf{project}
		
	El objeto project tiene dos campos
	\begin{itemize}
		\item \textbf{app\_path}: Un String con la ubicación del directorio de los fuentes de la aplicación, relativo al directorio del proyecto. Es donde están los archivos -java de la aplicación y donde se creará el archivo MyMainSamplersActivity.java
		\item \textbf{package\_name}: Un String con el nombre del package usado para las activities de la aplicación. Es el package donde la activity MyMainSamplersActivity será agregada.
	\end{itemize}
	
Ejemplo:
\begin{lstlisting}[language=XML, frame=tlb]	
{
  "project" : {
    "app_path" = "app/src/main/java/com/example/myApplication/"
    "package_name" : "com.example.myApplication"
  }
}
\end{lstlisting}	
	
	\item El objeto \textbf{application}
	El objeto application tiene 7 campos, de los cuales 3 son requeridos y los otros 4 opcionales (para habilitar características especiales)
	\begin{itemize}
		\item \textbf{title}: Un String con el nombre de la aplicación.
		
		 \item \textbf{welcomeMessage}: Un String con el mensaje de bienvenida que se mostrará en la activity principal (MyMainSamplersActivity)
		 
		 \item \textbf{networkConfiguration}: Un objeto con la configuración de red que se usará para enviar las muestras al servidor web. Ver mas abajo la configuración de este objeto.
		 
		 \item \textbf{googleMaps\_API\_KEY}: [Opcional] Un String con la API Key de Google. Este campo es necesario si se van a usar los servicios de ubicación y mapas (Location Step y Route Step). La API Key de google se puede obtener desde la página de google developers (https://developers.google.com/maps/documentation/android-api/signup)
		 
		 \item \textbf{mainHelpFileName}: [Opcional] Un String con el nombre del archivo HTML que contiene la ayuda principal de la aplicación. Este archivo debe estar junto con el archivo SamplersConfig.json. Ver la sección Mostrando Ayuda para mas detalles.
		 
		 \item \textbf{authenticationEnabled}: [Opcional] Un boolean que indica si se usará autenticación (true) o no (false). Si se omite este campo se asume false. Ver la sección Usando Autenticación para mas detalles.
		 
		 \item \textbf{authenticationOptional}: [Opcional] Un boolean que indica si la autenticación será opcional (true) o requerida (false). Si se omite este campo se asume true (autenticación opcional). Este campo solo tiene sentido si se usa autenticación. Ver la sección Usando Autenticación para mas detalles.
		 
	\end{itemize}
	
	
	El objeto \textbf{networkConfiguration}:
	El objeto networkConfiguration contiene la configuración de red que se usará para enviar las muestras al servidor web. Tiene 4 campos, de los cuales 2 son requeridos y los otros 2 opcionales.
	
	\begin{itemize}
	
		\item \textbf{url}: Un String con la URL del servidor web al cual se le enviaran las muestra con un mensaje HTTP POST.
		
		\item \textbf{paramName}: Un String con el nombre del parámetro dentro del mensaje HTTP POST en el que se enviará la muestra.
	
		\item \textbf{paramNameUserId}: (Opcional) Un String con el nombre del parámetro dentro del mensaje HTTP POST en el que se enviará el id del usuario que envía la muestra. Este campo solo es necesario si se usa autenticación. Ver la sección Usando Autenticación para mas detalles.
		
		\item \textbf{paramNameAuthenticationType}: (Opcional) Un String con el nombre del parámetro dentro del mensaje HTTP POST en el que se enviará el tipo de autenticación que usó el usuario que envía la muestra. Este campo solo es necesario si se usa autenticación. Ver la sección Usando Autenticación para mas detalles.
	
	\end{itemize}
	
	
Ejemplo:
\begin{lstlisting}[language=XML, frame=tlb]
{
  "application": {
    "title" : "Samplers Hello World App",
    "welcomeMessage" : "Welcome to your first Samplers App!",
    "networkConfiguration" : {
      "url" : "http://192.168.1.10/samplers/upload.php",
      "paramName" : "sample",
      "paramNameUserId" : "user_id",
      "paramNameAuthenticationType" : "authentication_type"
    },
    "authenticationEnabled" : true,
    "authenticationOptional" : true,
    "googleMaps_API_KEY" : "your_google_maps_API_KEY",
    "mainHelpFileName" : "mainhelp.html"
  } 
}
\end{lstlisting}	
	
	
	\item El objeto \textbf{workflow}
	El objeto workflow representa el protocolo para la toma de la muestra. Son los pasos que se ejecutarán para tomar la misma.
	El objeto cuenta con dos campos:
		
	\begin{itemize}
	
		\item \textbf{actionLabel}: Un String con el título que se usará para el botón que inicia la activity TakeSampleActivity, que es la encargada de tomar la muestra.
		
		\item \textbf{steps}: Un Array de Objetos Step los cuales forman el workflow. El primer objeto del array se considera como el inicio del mismo. Ver la sección Steps para mas detalles.
	
	
	\end{itemize}	
	
	
Ejemplo:
\begin{lstlisting}[language=XML, frame=tlb]	
{
  "workflow": {
    "actionLabel" : "Tomar muestra",
    "steps": [
      {
        "id" : 1,
        "type" : "Information",
        "text" : "Por favor, siga las instrucciones",
        "nextStepId" : 2
      },
      {
        "id" : 2,
        "type" : "Location",
        "text" : "Por favor posicione la muestra en el mapa",
        "nextStepId" : 3,
        "helpFileName" : "locationhelp.html"
      },
      {
        "id" : 3,
        "type": "MultipleSelect",
        "title" : "Seleecione las cosas que ve",
        "helpFileName" : "selecthelp.html",
        "options" : [
          {
            "id":1,
            "text":"Arboles"
          },
          {
            "id":2,
            "text":"Basura"
          },
          {
            "id":3,
            "text":"Agua"
          }
        ]
      }
    ]
  }
}

\end{lstlisting}	
	
\end{itemize}

**FALTA DESARROLLAR**

Aca podriamos poner un ejemplo completo de un archivo JSON de los que estan en la wiki.

**FALTA DESARROLLAR**

\subsection{Configuración de los Servicios de Google?? (esto quedó viejo, no se si va acá o en otro lado)}
**FALTA DESARROLLAR**

\section{Los diferentes Steps y sus resultados (StepResult)} \label{sec:steps_detallados}
**FALTA DESARROLLAR**

Esto me parece que ya esta explicado arriba... hay que ver donde lo explicamos en forma generica y aca el detalle de cada uno.

Los steps representan un paso en el protocolo de recolección de la muestra (workflow).
Los StepResult son el resultado obtenido de ejecutar un Step.
Una muestra esta formada por la colección de StepResult generada luego de ejecutar cada Step del Workflow. Cada ejecución del workflow puede generar una colección de StepResults diferente.

\subsection{InformationStep: Mostrar información}
InformationStep se usa para mostrar una información (texto) al científico ciudadano. 

En Android Studio (Java):
\begin{lstlisting}[language=Java, frame=tlb]	
InformationStep step = new InformationStep(1,"Texto para mostrar",2);
\end{lstlisting}

Usando el generador de clases:
\begin{lstlisting}[language=XML, frame=tlb]	
{
	"id":1,
	"type" : "Information",
	"text" : "Texto para mostrar",
	"nextStepId": 2
}
\end{lstlisting}

\begin{figure}[H]
  \centering
    \includegraphics[scale=0.4]{05-implementacion/InformationStep.png} 
   \caption{Ejemplo de un InformationStep en ejecución}
   \label{fig:imgInformationStep}
\end{figure}

\subsubsection{InformationStepResult: El resultado de Mostrar información}
El resultado de mostrar información es nulo, es una clase vacía. Solo está para cerrar el circuito.

\subsection{PhotoStep: Tomar una foto}
PhotoStep se usa para solicitarle al científico ciudadano que tome una foto. Tiene un texto (instructionsToShow) que se muestran a modo de instrucciones o mensaje cuando la cámara esta encendida. Luego de tomada la foto, se muestra una vista preliminar de la misma, con un botón que da la opción de volver a tomar la foto.

En Android Studio (Java):
\begin{lstlisting}[language=Java, frame=tlb]	
PhotoStep step = new PhotoStep(1,"Por favor tome una foto de su gato",2);
\end{lstlisting}

Usando el generador de clases:
\begin{lstlisting}[language=XML, frame=tlb]	
{
  "id" : 1,
  "type" : "Photo",
  "text" : "Por favor tome una foto de su gato",
  "nextStepId" : 2
}
\end{lstlisting}

\begin{figure}[H]
  \centering
    \includegraphics[scale=0.4]{05-implementacion/PhotoStep1.png} 
    \includegraphics[scale=0.4]{05-implementacion/PhotoStep2.png}     
   \caption{Ejemplo de un PhotoStep en ejecución. A la izquierda al momento de tomar la foto, y a la derecha al momento de mostrar la vista preliminar.}
   \label{fig:imgPhotoStep}
\end{figure}

\subsubsection{PhotoStepResult: El resultado de Tomar una foto}
guarda el imageFileName de la foto. 
La foto va como archivo jpg en la carpeta de la muestra.
Puede haber varias fotos si hay varios PhotoSteps en el workflow

\subsection{SoundRecordStep: Grabar sonido}
Tiene instructionsToShow a modo de instrucciones que se muestran.

En Android Studio (Java):
\begin{lstlisting}[language=Java, frame=tlb]	
SoundRecordStep step = new SoundRecordStep(1,"Grabe el sonido de su auto",2); 
\end{lstlisting}

Usando el generador de clases:
\begin{lstlisting}[language=XML, frame=tlb]	
{
  "id" : 1,
  "type" : "Sound",
  "text" : "Grabe el sonido de su auto",
  "nextStepId" : 2
}
\end{lstlisting}

\subsubsection{SoundRecordStepResult: El resultado de Grabar sonido}
guarda el soundFileName del sonido.
El sonido va como archivo mp4 en la carpeta de la muestra.
Puede haber varios sonidos si hay varios SoundRecordSteps en el workflow


\subsection{SelectOneStep: Seleccionar una opción de un grupo de opciones}
Tiene un title
Las opciones son una lista de objetos SelectOneOption
Cada objeto SelectOneOption tiene un id, textToShow y nextStepId
Explicar la bifurcación de caminos en el workflow a partir de este Step
En forma de radio buttons

En Android Studio (Java):
\begin{lstlisting}[language=Java, frame=tlb]	
ArrayList<SelectOneOption> optionsToSelectOne = new ArrayList<SelectOneOption>();
optionsToSelectOne.add(new SelectOneOption(1,"Opcion 1", 2));
optionsToSelectOne.add(new SelectOneOption(2,"Opcion 2", 2));
optionsToSelectOne.add(new SelectOneOption(3,"Opcion 3", 3));
SelectOneStep step = new SelectOneStep(1,optionsToSelectOne,"Seleccione una opcion");

\end{lstlisting}

Usando el generador de clases:
\begin{lstlisting}[language=XML, frame=tlb]	
{
  "id" : 1,
  "type" : "SelectOne",
  "title" : "Seleccione una opcion",
  "options" : [
    {
      "id":1,
      "text":"Opcion 1",
      "nextStepId" : 2
    },
    {
      "id":2,
      "text":"Opcion 2",
      "nextStepId" : 2
    },
    {
      "id":3,
      "text":"Opcion 3",
      "nextStepId" : 3
    }
  ]
}
\end{lstlisting}

\subsubsection{SelectOneStepResult: El resultado de Seleccionar una opción de un grupo de opciones}
El resultado tiene la opción seleccionada (un objeto SelectOneOption)

\subsection{MultipleSelectStep: Seleccionar varias opciones de un grupo de opciones}
Tiene un title
Las opciones son una lista de objetos MultipleSelectOption
Cada objeto MultipleSelectOption tiene un id y textToShow
en forma de checkboxes

En Android Studio (Java):
\begin{lstlisting}[language=Java, frame=tlb]	
ArrayList<MultipleSelectOption> optionsToSelect = new ArrayList<MultipleSelectOption>();
optionsToSelect.add(new MultipleSelectOption(1,"Arboles"));
optionsToSelect.add(new MultipleSelectOption(2,"Basura"));
optionsToSelect.add(new MultipleSelectOption(3,"Agua"));
optionsToSelect.add(new MultipleSelectOption(4,"Animales"));
MultipleSelectStep step = new MultipleSelectStep(1,optionsToSelect,"Seleccione lo que ve",2); 
\end{lstlisting}

Usando el generador de clases:
\begin{lstlisting}[language=XML, frame=tlb]	
{
  "id" : 1,
  "type" : "MultipleSelect",
  "title" : "Seleccione lo que ve",
  "options" : [
    {
      "id":1,
      "text":"Arboles"
    },
    {
      "id":2,
      "text":"Basura"
    },
    {
      "id":3,
      "text":"Agua"
    },
    {
      "id":4,
      "text":"Animales"
    }
  ],
  "nextStepId" : 2
}
\end{lstlisting}

\subsubsection{MultipleSelectStepResult: El resultado de Seleccionar varias opciones de un grupo de opciones}
El resultado tiene una lista de las opciones seleccionadas (una lista de objetos MultipleSelectOption)



\subsection{LocationStep: Posicionar la muestra en el mapa con el GPS}
Tiene un textToShow a modo de instrucciones
Permite usar el GPS o posicionar la muestra manualmente en el mapa

En Android Studio (Java):
\begin{lstlisting}[language=Java, frame=tlb]	
LocationStep step = new LocationStep(1,"Por favor posicione la muestra en el mapa",2); 
\end{lstlisting}

Usando el generador de clases:
\begin{lstlisting}[language=XML, frame=tlb]	
{
  "id" : 1,
  "type" : "Location",
  "text" : "Por favor posicione la muestra en el mapa",
  "nextStepId" : 2
}
\end{lstlisting}

\subsubsection{LocationStepResult: El resultado de Posicionar la muestra en el mapa con el GPS}
Guarda latitude y longitude

\subsection{RouteStep: Grabar un recorrido en el mapa usando el GPS}
Tiene un textToShow a modo de instrucciones
Intervalo y mapZoom opcionales. Poner los valores por defecto

En Android Studio (Java):
\begin{lstlisting}[language=Java, frame=tlb]	
RouteStep step = new RouteStep(1,"Registre la ruta que corre",2); 
step.setInterval(10000);
step.setMapZoom(18);
\end{lstlisting}

Usando el generador de clases:
\begin{lstlisting}[language=XML, frame=tlb]	
{
  "id" : 1,
  "type" : "Route",
  "text" : "Registre la ruta que corre",
  "interval" : 10000,
  "mapZoom" : 18,
  "nextStepId" : 2
}
\end{lstlisting}

\subsubsection{RouteStepResult: El resultado de Grabar un recorrido en el mapa usando el GPS}
guarda una lista de objetos Location

\subsection{InsertTextStep: Ingresar texto}
textToShow a modo de instrucciones
sampleText a modo de ejmplo
maxLength cantidad máxima de caracteres permitida
Type Values allowed are: text, number or decimal
optional indicando si se puede dejar vacío y no ingresar ningún texto (true) o si se requiere que ingrese algo si o si (false)


En Android Studio (Java):
\begin{lstlisting}[language=Java, frame=tlb]	
InsertTextStep step = new InsertTextStep(1,"Por favor, ingrese el nombre del lago","Nombre del lago",50,InsertTextStep.InputType.TYPE_TEXT,true,2);
\end{lstlisting}

Usando el generador de clases:
\begin{lstlisting}[language=XML, frame=tlb]	
{
  "id" : 1,
  "type" : "InsertText",
  "text" : "Por favor, ingrese el nombre del lago",
  "sampleText" : "Nombre del lago",
  "inputType" : "text",
  "maxLength" : 50,
  "optional" : true,
  "nextStepId" : 2
}
\end{lstlisting}

\subsubsection{InsertTextStepResult: El resultado de Ingresar texto}
guarda el texto ingresado en insertedText

\subsection{InsertDateStep y InsertTimeStep: Ingresar fecha y hora}
ambos tienen textToShow a modo de instrucciones

En Android Studio (Java):
\begin{lstlisting}[language=Java, frame=tlb]	
InsertDateStep step = new InsertDateStep(1,"Por favor indique la fecha de la muestra",2); 
\end{lstlisting}

Usando el generador de clases:
\begin{lstlisting}[language=XML, frame=tlb]	
{
  "id" : 1,
  "type" : "InsertDate",
  "text" : "Por favor indique la fecha de la muestra",
  "nextStepId" : 2
}
\end{lstlisting}

En Android Studio (Java):
\begin{lstlisting}[language=Java, frame=tlb]	
InsertTimeStep step6 = new InsertTimeStep(1,"Por favor indique la hora de la muestra",2); 
\end{lstlisting}

Usando el generador de clases:
\begin{lstlisting}[language=XML, frame=tlb]	
{
  "id" : 1,
  "type" : "InsertTime",
  "text" : "Por favor indique la hora de la muestra",
  "nextStepId" : 2
}
\end{lstlisting}

\subsubsection{InsertDateStepResult y InsertTimeStep: El resultado de Ingresar fecha y hora}
un objeto Date que tiene la fecha o la hora según corresponda

\section{Mostrar Ayuda}

\section{Usando autenticación} \label{sec:usando_autenticacion}
Por defecto provee autenticación con Google, porque al tener Android tiene una cuenta de Google si o si.
Esta abierto a poder agregar autenticación con otras plataformas/APIs.

Samplers provee autenticación con Google, pero ese necesario registrar la aplicación en la pagina de desarrolladores de google (https://developers.google.com/identity/sign-in/android/start-integrating). Ahí hay que seguir los pasos para [to Configure a Google API Console project]. Es necesario [proveer] el nombre de la aplicación, package name, y también el SHA-1 hash del certificado con el que se firma la aplicación.

Una vez registrada la aplicación en Google, hay que configurar Samplers para habilitar la autenticación.

Una vez configurado los valores de los parámetros de autenticación, Samplers mostrará un fragment de inicio de sesión la primera vez que el usuario intente tomar una muestra. Si la autentición es opcional, se mostrará un botón para omitir el inicio de sesión y continuar con la toma de la muestra. También se muestra un botón para iniciar sesión en la activity principal (si se está usando la que provee Samplers).

Cuando la muestra es enviada, el id de usuario y el método de autenticación (por defecto 'google') se envían junto con esta.


\subsection{Configurar autenticación con el generador de clases de Gradle}

Para usar autenticación usando el generador de clases de Gradle, es necesario configurar los siguientes parámetros en el objeto \textbf{applicaction}:

\begin{itemize}

	\item \textbf{authenticationEnabled}: poner en true para habilitar la autenticación
		
	\item \textbf{authenticationOptional}: poner en true si se desea que la autenticación sea opcional, o en false si se desea que la autenticación sea obligatoria.
	
	\item Dentro del parámetro \textbf{networkConfiguration} es necesario establecer los parámetros \textbf{paramNameUserId} y \textbf{paramNameAuthenticationType} con los nombres de los parámetros con los que irán el id de usuario y el tipo de autenticación usada respectivamente dentro del mensaje HTTP POST.
	

\end{itemize}

Ejemplo:

\begin{lstlisting}[language=XML, frame=tlb]	
{
  "application": {
    "title" : "Samplers Hello World App",
    "welcomeMessage" : "Welcome to your first Samplers App!",
    "networkConfiguration" : {
      "url" : "http://192.168.1.10/samplers/upload.php",
      "paramName" : "sample",
      "paramNameUserId" : "user_id",
      "paramNameAuthenticationType" : "authentication_type"
    },
    "authenticationEnabled" : true,
    "authenticationOptional" : true
  } 
}
\end{lstlisting}

\subsection{Configurar autenticación [con instanciación manual]}

Para usar autenticación [con instanciación manual], es necesario definir la configuración de red y de autenticación en el método \textbf{onCreate()} de la activity principal:

\begin{lstlisting}[language=Java, frame=tlb]	
@Override
protected void onCreate(Bundle savedInstanceState) {
  super.onCreate(savedInstanceState);
	
  NetworkConfiguration.setURL("http://192.168.1.10/samplers/upload.php");
  NetworkConfiguration.setPARAM_NAME_SAMPLE("sample");
  // Set the authentication params of the Network Configuration
  NetworkConfiguration.setPARAM_NAME_USER_ID("user_id");
  NetworkConfiguration.setPARAM_NAME_AUTHENTICATION_TYPE("authentication_type");

  // Set the authenticationconfiguration
  AuthenticationManager.setAuthenticationEnabled(true);
  AuthenticationManager.setAuthenticationOptional(true);
}
\end{lstlisting}

\subsection{Usando un método de autenticación propio} \label{sec:usar_auth_propia}

Con Samplers también se puede usar un método de autenticación propio, definiendo un LoginFragment y una clase User (o varias clases si se desea proporcionar autenticación con diferentes APIs, como Facebook, Tweeter, Yahoo, etc.) y Samplers enviará junto con la muestra el id de usuario y el método de autenticación usado.


\subsubsection{Definiendo un Login Fragment [personalizado]}

Es necesario crear un fragment que herede de LoginFragment, y configurar la clase AuthenticationManager para que lo use, llamando al método setLoginFragmentClass() dentro del método onCreate() de la ativity principal:

\begin{lstlisting}[language=Java, frame=tlb]	
@Override
protected void onCreate(Bundle savedInstanceState) {
  super.onCreate(savedInstanceState);
	
  AuthenticationManager.setLoginFragmentClass(MyCustomLoginFragment.class);
}
\end{lstlisting}

El proceso de login y la interacción con las APIs responsabilidad del desarrollador, pero después de que el usuario inicia sesión en la API seleccionada, es necesario llamar al método login() en la clase AuthenticationManager y al método onLogin() en el objeto mListener heredado:

\begin{lstlisting}[language=Java, frame=tlb]	
if (loginOK) {
  AuthenticationManager.login(user, getActivity().getApplicationContext());
  mListener.onLogin(user);
}
\end{lstlisting}


\subsubsection{Definiendo una clase User [personalizada]}

Es necesario crear una clase usuario propia que implemente la interfaz User por cada método de autenticación que se use. Los objetos de dichas clases se usarán para llamar al método login() de la clase AuthenticationManager.

Ejemplo de una clase usuario propia:
\begin{lstlisting}[language=Java, frame=tlb]	
public class EMailUser implements User {

    public static final String AUTHENTICATION_TYPE = "email";

    private String userName;
    private String email;

    public GoogleUser(String userName, String email) {
        this.userName = userName;
        this.email = email;
    }

    @Override
    public String getAuthenticationType() {
        return AUTHENTICATION_TYPE;
    }

    @Override
    public String getUserName() {
        return userName;
    }

    @Override
    public String getUserId() {
        return email;
    }

}
\end{lstlisting}



\section{Personalización}

\subsection{Temas y colores}

\subsection{Idiomas}

\section{Definir un nuevo Step, StepFragment y StepResult ??} \label{sec:definir_steps}

** Ver Auth0


\chapter{Instanciación  y uso del framework}
En esta sección se explica como instanciar el framework para su uso, brindando algunos ejemplos concretos.

Una vez instalado el framework el siguiente paso es instanciarlo para usarlo. La instanciación puede ser manual o usando el generador de clases de Gradle.


\section{Instanciación manual} \label{sec:instanciacion_manual}
Una vez agregada la librería de Samplers al proyecto, se debe crear un objeto Workflow, agregarle los objetos Steps, y llamar a la activity TakeSampleActivity pasándole el objeto Workflow como parámetro.
También es necesario establecer la configuración general en el método onCreate de la activity principal (main activity) para indicar a donde se enviarán las muestras tomadas con la app.
Se puede usar una activity principal propia o se puede heredar de SamplersMainActivity. En ambos casos se debe hacer lo siguiente:
\begin{itemize}
	\item Establecer la configuración general en el método onCreate de la activity principal:
\end{itemize}
	
\begin{comment}		
\begin{figure}[H]
  \centering
    \includegraphics[scale=0.6]{50-anexos/B-uso/configuracion_general.png} 
   \caption{Ejemplo de configuración general}
\end{figure}		
\end{comment}

\begin{lstlisting}[language=Java, frame=tlbr, caption=Ejemplo de configuración general]
@Override
protected void onCreate(Bundle savedInstanceState) {
	super.onCreate(savedInstanceState);
	
	// Establecer la configuración de red
	NetworkConfiguration.setURL("http://192.168.1.10/samplers/upload.php");
	NetworkConfiguration.setPARAM_NAME_SAMPLE("sample");

	// Opcional, si se desea usar identificación (*)
	NetworkConfiguration.setPARAM_NAME_USER_ID("user_id");
	NetworkConfiguration.setPARAM_NAME_AUTHENTICATION_TYPE("authentication_type");

	AuthenticationManager.setAuthenticationEnabled(true);
	AuthenticationManager.setAuthenticationOptional(true);
}
\end{lstlisting}
		
		
(*) Ver la sección \ref{sec:usando_autenticacion} para más detalles sobre como usar identificación.

\begin{itemize}
	\item Crear un Workflow. Si se está heredando de SamplersMainActivity se debe hacer sobreescribiendo el método getWorkflow.
\end{itemize}	

\begin{comment}
\begin{figure}[H]
  \centering
    \includegraphics[scale=0.6]{50-anexos/B-uso/get_workflow.png} 
   \caption{Ejemplo de un Workflow que tiene dos Steps. El primero muestra un mensaje de bienvenida y el segundo pide para tomar una foto.}
\end{figure}		
\end{comment}

\begin{lstlisting}[language=Java, frame=tlbr, caption=Ejemplo de un Workflow que tiene dos Steps. El primero muestra un mensaje de bienvenida y el segundo pide para tomar una foto.]
@Override
protected Workflow getWorkflow() {
  Workflow workflow = new Workflow();

  Step step = new PhotoStep(2,"Por favor tome una foto de su gato", null);
  workflow.add(step);

  step = new InformationStep(1,"Bienvenido a la app de prueba", 2);
  workflow.add(step);

  return workflow;
}		
\end{lstlisting}
	

Nota: Para ver los distintos Steps que se pueden usar vea la sección \ref{sec:steps_detallados}.

\begin{itemize}
	\item Iniciar la activity TakeSampleActivity. Si se está heredando de SamplersMainActivity esto se hace automáticamente en el método onClick del botón "tomar muestra". De lo contrario, deberá iniciarla de la siguiente forma, en el método onClick de un botón por ejemplo:
\end{itemize}

\begin{comment}
\begin{figure}[H]
  \centering
    \includegraphics[scale=0.6]{50-anexos/B-uso/take_sample_click.png} 
   \caption{Ejemplo de como iniciar la activity TakeSampleActivity.}
\end{figure}		
\end{comment}

\begin{lstlisting}[language=Java, frame=tlbr, caption=Ejemplo de como iniciar la activity TakeSampleActivity.]
public void takeSampleClick(View view) {
  Workflow workflow = getWorkflow();

  Intent intent = new Intent(this, TakeSampleActivity.class);        
  intent.putExtra(TakeSampleActivity.EXTRA_WORKFLOW, workflow);
  startActivity(intent);
}		
\end{lstlisting}




\section{Instanciación usando el  generador de clases de Gradle}

Básicamente, el generador de clases de Gradle se encarga de hacer una instanciación manual a partir de un archivo de configuración (JSON). Está pensado para desarrolladores que no tienen muchos conocimientos en Android, o para servir de interfaz entre una aplicación que genere apps a través de Samplers.

Los pasos para usar el generador de clases de Gradle son:
\begin{enumerate}
	\item Crear un archivo JSON con el nombre SamplersConfig.json
		\begin{itemize}
			\item El formato y las opciones están explicadas sección \ref{sec:archivo_config_detallado}.
			\item Para validar sintaxis se puede usar por ejemplo el validador online https://jsonformatter.curiousconcept.com que es gratuito.
			\item Al final de esta sección se provee un archivo de ejemplo.
		\end{itemize}
		
	\item Copiar el archivo creado en el item anterior al directorio raíz del proyecto Android creado.
	
	\item Descargar la última versión de los archivos \textbf{samplers.gradle} y \textbf{samplersclassgenerator.jar} del repositorio de Samplers\footnote{https://github.com/cientopolis/samplers/releases/} y copiarlos también al directorio raíz del proyecto Android.
	
	\item Enlazar el archivo samplers.gradle en el archivo \textbf{build.gradle} de la aplicación:
		\begin{itemize}
			\item Android Studio crea por defecto dos archivos build.gradle, uno a nivel de aplicación y otro a nivel de proyecto. Debe usarse el de aplicación.
			\item Al final del archivo build.gradle de aplicación agregar:
\begin{lstlisting}[language=XML, frame=tlbr]
apply from: '../samplers.gradle'
\end{lstlisting}
			\item Al guardar los cambios, Android Studio sugerirá hacer una sincronización del proyecto, hacerla. Esto generará en la aplicación una activity llamada MyMainSamplersActivity en base a las opciones configuradas en el archivo SamplersConfig.json.
			\item Si se necesita volver a generar esta activity (si se quieren modificar algunas opciones por ejemplo ) se puede eliminar la misma, hacer las modificaciones en el archivo SamplersConfig.json y volver a generar el proyecto (en el menu Build -$>$ Make Project)
		\end{itemize}
		
	\item Eliminar o personalizar el archivo \textbf{style.xml} que está en \textbf{res/values} en la aplicación
	
	\item Ejecutar la aplicación y listo.

\end{enumerate}


\section{Secciones del archivo} \label{sec:archivo_config_detallado}

El archivo SamplersConfig.json es un archivo JSON con 3 objetos: project, application y workflow. A continuación se explican en detalle cada uno de ellos.

\subsection{El objeto project}
		
	El objeto project tiene dos campos
	\begin{itemize}
		\item \textbf{app\_path}: Un String con la ubicación del directorio de los fuentes de la aplicación, relativo al directorio del proyecto. Es donde están los archivos -java de la aplicación y donde se creará el archivo MyMainSamplersActivity.java
		\item \textbf{package\_name}: Un String con el nombre del package usado para las activities de la aplicación. Es el package donde la activity MyMainSamplersActivity será agregada.
	\end{itemize}
	
\begin{comment}
\begin{figure}[H]
  \centering
    \includegraphics[scale=0.6]{50-anexos/B-uso/json_project.png} 
    \caption{Ejemplo del objeto project.}
\end{figure}	
\end{comment}

\begin{lstlisting}[language=XML, frame=tlbr, caption=Ejemplo del objeto project.]	
"project" : {
	"app_path" = "app/src/main/java/com/example/myApplication/"
	"package_name" : "com.example.myApplication"
}
\end{lstlisting}

	
\subsection{El objeto application}
	El objeto application tiene 7 campos, de los cuales 3 son requeridos y los otros 4 opcionales (para habilitar características especiales)
	\begin{itemize}
		\item \textbf{title}: Un String con el nombre de la aplicación.
		
		 \item \textbf{welcomeMessage}: Un String con el mensaje de bienvenida que se mostrará en la activity principal (MyMainSamplersActivity)
		 
		 \item \textbf{networkConfiguration}: Un objeto con la configuración de red que se usará para enviar las muestras al servidor web. Ver mas abajo la configuración de este objeto.
		 
		 \item \textbf{googleMaps\_API\_KEY}: [Opcional] Un String con la API Key de Google. Este campo es necesario si se van a usar los servicios de ubicación y mapas (Location Step y Route Step). La API Key de google se puede obtener desde la página de google developers \footnote{https://developers.google.com/maps/documentation/android-api/signup}.
		 
		 \item \textbf{mainHelpFileName}: [Opcional] Un String con el nombre del archivo HTML que contiene la ayuda principal de la aplicación. Este archivo debe estar junto con el archivo SamplersConfig.json. Ver la sección \ref{sec:mostrar_ayuda} para mas detalles.
		 
		 \item \textbf{authenticationEnabled}: [Opcional] Un boolean que indica si se usará autenticación (true) o no (false). Si se omite este campo se asume false. Ver la sección Usando Autenticación para mas detalles.
		 
		 \item \textbf{authenticationOptional}: [Opcional] Un boolean que indica si la autenticación será opcional (true) o requerida (false). Si se omite este campo se asume true (autenticación opcional). Este campo solo tiene sentido si se usa autenticación. Ver la sección Usando Autenticación para mas detalles.
		 
	\end{itemize}
	
	
	El objeto \textbf{networkConfiguration}:
	El objeto networkConfiguration contiene la configuración de red que se usará para enviar las muestras al servidor web. Tiene 4 campos, de los cuales 2 son requeridos y los otros 2 opcionales.
	
	\begin{itemize}
	
		\item \textbf{url}: Un String con la URL del servidor web al cual se le enviaran las muestra con un mensaje HTTP POST.
		
		\item \textbf{paramName}: Un String con el nombre del parámetro dentro del mensaje HTTP POST en el que se enviará la muestra.
	
		\item \textbf{paramNameUserId}: [Opcional] Un String con el nombre del parámetro dentro del mensaje HTTP POST en el que se enviará el id del usuario que envía la muestra. Este campo solo es necesario si se usa autenticación. Ver la sección Usando Autenticación para mas detalles.
		
		\item \textbf{paramNameAuthenticationType}: [Opcional] Un String con el nombre del parámetro dentro del mensaje HTTP POST en el que se enviará el tipo de autenticación que usó el usuario que envía la muestra. Este campo solo es necesario si se usa autenticación. Ver la sección Usando Autenticación para mas detalles.
	
	\end{itemize}
	
\begin{comment}
\begin{figure}[H]
  \centering
    \includegraphics[scale=0.6]{50-anexos/B-uso/json_application.png} 
    \caption{Ejemplo del objeto application.}
\end{figure}	
\end{comment}

\begin{lstlisting}[language=XML, frame=tlbr, caption=Ejemplo del objeto application.]
"application": {
	"title" : "App de prueba",
	"welcomeMessage" : "Bienvenido a la app de prueba!",
	"networkConfiguration" : {
		"url" : "http://192.168.1.10/samplers/upload.php",
		"paramName" : "sample",
		"paramNameUserId" : "user_id",
		"paramNameAuthenticationType" : "authentication_type"
	},
	"authenticationEnabled" : true,
	"authenticationOptional" : true,
	"googleMaps_API_KEY" : "your_google_maps_API_KEY",
	"mainHelpFileName" : "mainhelp.html"
}
\end{lstlisting}
	
	
\subsection{El objeto workflow}
	El objeto workflow representa el protocolo para la toma de la muestra. Son los pasos que se ejecutarán para tomar la misma.
	El objeto cuenta con dos campos:
		
	\begin{itemize}
	
		\item \textbf{actionLabel}: Un String con el título que se usará para el botón que inicia la activity TakeSampleActivity, que es la encargada de tomar la muestra.
		
		\item \textbf{steps}: Un Array de Objetos Step los cuales forman el workflow. El primer objeto del array se considera como el inicio del mismo. Ver la sección Steps para mas detalles.
	
	
	\end{itemize}	
	
\begin{comment}
\begin{figure}[H]
  \centering
    \includegraphics[scale=0.6]{50-anexos/B-uso/json_workflow.png} 
    \caption{Ejemplo del objeto workflow.}
\end{figure}	
\end{comment}
		

\begin{lstlisting}[language=XML, frame=tlbr, caption=Ejemplo del objeto workflow.]	
"workflow": {
	"actionLabel" : "Tomar muestra",
	"steps": [
		{
			"id" : 1,
			"type" : "Information",
			"text" : "Por favor, siga las instrucciones",
			"nextStepId" : 2
		},
		{
			"id" : 2,
			"type" : "Location",
			"text" : "Posicione la muestra en el mapa",
			"nextStepId" : 3
		},
		{
			"id" : 3,
			"type" : "Photo",
			"text" : "Tome una foto del lugar",
			"nextStepId" : 4,
			"helpFileName" : "photohelp.html"
		},      
		{
			"id" : 4,
			"type": "MultipleSelect",
			"title" : "Seleecione las cosas que ve",
			"options" : [
				{
					"id":1,
					"text":"Arboles"
				},
				{
					"id":2,
					"text":"Basura"
				},
				{
					"id":3,
					"text":"Agua"
				},
				{
					"id":4,
					"text":"Animales"
				}          
			]
		}
	]
}
\end{lstlisting}



\section{Mostrar Ayuda} \label{sec:mostrar_ayuda}
Samplers permite mostrar ayuda para el usuario final de la app a través de archivos HTML. Estos archivos deben estar ubicados junto con el archivo SamplersConfig.json

Samplers permite dos formas de mostrar ayuda:

\begin{itemize}

	\item \textbf{Ayuda general}: que se accede desde el menú de la Activity principal cuando se configura un archivo de ayuda para la app.
		
	\item \textbf{Ayuda puntual para cada Step}: que se accede desde un botón en los StepFragments que aparece cuando el Step tiene configurado un archivo de ayuda.

\end{itemize}


\begin{figure}[H]
  \centering
    \includegraphics[scale=0.3]{50-anexos/B-uso/ayuda_ejemplo.png} 
    \caption{Ejemplo de como se muestra la ayuda.}
\end{figure}	
		

\textbf{JavaScript no esta habilitado:} Debido a las limitaciones del componente WebView de Android no esta habilitado el uso de JavaScript. Para más información ver la documentación oficial de Android sobre WebView.\footnote{https://developer.android.com/reference/android/webkit/WebView}

\subsection{Ayuda general}
Para configurar la ayuda general se debe completar el atributo \textit{mainHelpFileName} del objeto \textit{application} con el nombre del archivo HTML que contiene la ayuda que se quiere mostrar. El archivo HTML debe estar ubicado junto con el archivo SamplersConfig.json.

\begin{comment}
\begin{figure}[H]
  \centering
    \includegraphics[scale=0.6]{50-anexos/B-uso/json_application_ayuda.png} 
    \caption{Ejemplo de configuración de ayuda general.}
\end{figure}	
\end{comment}

\begin{lstlisting}[language=XML, frame=tlbr, caption=Ejemplo de configuración de ayuda general (línea 13).]
"application": {
	"title" : "App de prueba",
	"welcomeMessage" : "Bienvenido a la app de prueba!",
	"networkConfiguration" : {
		"url" : "http://192.168.1.10/samplers/upload.php",
		"paramName" : "sample",
		"paramNameUserId" : "user_id",
		"paramNameAuthenticationType" : "authentication_type"
	},
	"authenticationEnabled" : true,
	"authenticationOptional" : true,
	"googleMaps_API_KEY" : "your_google_maps_API_KEY",
	"mainHelpFileName" : "mainhelp.html"
}
\end{lstlisting}

\subsection{Ayuda puntual para cada Step}
Para configurar la ayuda para un Step se debe completar el atributo \textit{helpFileName} del objeto \textit{step} con el nombre del archivo HTML que contiene la ayuda que se quiere mostrar. El archivo HTML debe estar ubicado junto con el archivo SamplersConfig.json.

\begin{comment}
\begin{figure}[H]
  \centering
    \includegraphics[scale=0.6]{50-anexos/B-uso/json_step_ayuda.png} 
    \caption{Ejemplo de configuración de ayuda en un Step.}
\end{figure}	
\end{comment}

\begin{lstlisting}[language=XML, frame=tlbr, caption=Ejemplo de configuración de ayuda en un Step (línea 9).]	
"workflow": {
	"actionLabel" : "Tomar muestra",
	"steps": [
		{
			"id" : 1,
			"type" : "Photo",
			"text" : "Tome una foto del lugar",
			"nextStepId" : 2,
			"helpFileName" : "photohelp.html"
		},      
...		
\end{lstlisting}


\section{Usando identificación} \label{sec:usando_autenticacion}
Por defecto Samplers provee identificación con Google, porque en Android es necesario tener una cuenta de Google para usar el dispositivo móvil, aunque también está preparado para poder integrar identificación con otras plataformas/APIs.

Para usar la identificación con Google, es necesario registrar la aplicación en la pagina de desarrolladores de google\footnote{https://developers.google.com/identity/sign-in/android/start-integrating}. Ahí hay que seguir los pasos para configurar la API en el proyecto dentro de la consola de Google. Es necesario proveer el nombre de la aplicación, package name, y también el SHA-1 hash del certificado con el que se firma la aplicación.

Una vez registrada la aplicación en Google, hay que configurar Samplers para habilitar la identificación como se explican en las secciones \ref{sec:identificacion_gradle} o \ref{sec:identificacion_manual}.

Una vez configurado los valores de los parámetros de identificación, Samplers mostrará un fragment de inicio de sesión la primera vez que el usuario intente tomar una muestra. Si la identificación es opcional, se mostrará un botón para omitir el inicio de sesión y continuar con la toma de la muestra. También se muestra un botón para iniciar sesión en la activity principal (si se está usando la que provee Samplers).

Cuando la muestra es enviada, el id de usuario y el método de identificación (por defecto 'google') se envían junto con esta.


\subsection{Configurar identificación con el generador de clases de Gradle} \label{sec:identificacion_gradle}

Para usar identificación usando el generador de clases de Gradle, es necesario configurar los siguientes parámetros en el objeto \textbf{applicaction}:

\begin{itemize}

	\item \textbf{authenticationEnabled}: poner en true para habilitar la identificación
		
	\item \textbf{authenticationOptional}: poner en true si se desea que la identificación sea opcional, o en false si se desea que la identificación sea obligatoria.
	
	\item Dentro del parámetro \textbf{networkConfiguration} es necesario establecer los parámetros \textbf{paramNameUserId} y \textbf{paramNameAuthenticationType} con los nombres de los parámetros con los que irán el id de usuario y el tipo de identificación usada respectivamente dentro del mensaje HTTP POST.
	

\end{itemize}

\begin{comment}
\begin{figure}[H]
  \centering
    \includegraphics[scale=0.6]{50-anexos/B-uso/identificacion.png} 
    \caption{Ejemplo del objeto application configurado para usar identificación.}
\end{figure}	
\end{comment}

\begin{lstlisting}[language=XML, frame=tlbr, caption=Ejemplo del objeto application configurado para usar identificación (líneas 7\, 8\, 10 y 11).]
"application": {
	"title" : "App de prueba",
	"welcomeMessage" : "Bienvenido a la app de prueba!",
	"networkConfiguration" : {
		"url" : "http://192.168.1.10/samplers/upload.php",
		"paramName" : "sample",
		"paramNameUserId" : "user_id",
		"paramNameAuthenticationType" : "authentication_type"
	},
	"authenticationEnabled" : true,
	"authenticationOptional" : true,
	"googleMaps_API_KEY" : "your_google_maps_API_KEY",
	"mainHelpFileName" : "mainhelp.html"
}
\end{lstlisting}


\subsection{Configurar identificación manualmente} \label{sec:identificacion_manual}

Para usar identificación cuando se usan las clases directamente, es necesario definir la configuración de red y de identificación en el método \textbf{onCreate()} de la activity principal:

\begin{comment}
\begin{figure}[H]
  \centering
    \includegraphics[scale=0.6]{50-anexos/B-uso/identificacion_manual.png} 
    \caption{Ejemplo de configuración de identificación en el método onCreate.}
\end{figure}	
\end{comment}

\begin{lstlisting}[language=Java, frame=tlbr, caption=Ejemplo de configuración de identificación en el método onCreate.]	
@Override
protected void onCreate(Bundle savedInstanceState) {
	super.onCreate(savedInstanceState);

	// Establecer la configuración de red
	NetworkConfiguration.setURL("http://192.168.1.10/samplers/upload.php");
	NetworkConfiguration.setPARAM_NAME_SAMPLE("sample");
	
	// Establecer la configuración para usar Identificación
	NetworkConfiguration.setPARAM_NAME_USER_ID("user_id");
	NetworkConfiguration.setPARAM_NAME_AUTHENTICATION_TYPE("authentication_type");	
	AuthenticationManager.setAuthenticationEnabled(true);
	AuthenticationManager.setAuthenticationOptional(true);
}
\end{lstlisting}


\subsection{Usando un método de identificación propio} \label{sec:usar_auth_propia}

Con Samplers también se puede usar un método de identificación propio, definiendo un LoginFragment y una clase User (o varias clases si se desea proporcionar identificación con diferentes APIs, como Facebook, Instagram, Tweeter, etc.) y Samplers enviará junto con la muestra el id de usuario y el método de identificación usado.


\subsubsection{Definiendo un Login Fragment propio}

Es necesario crear un fragment que herede de LoginFragment, y configurar la clase AuthenticationManager para que lo use, llamando al método setLoginFragmentClass() dentro del método onCreate() de la ativity principal:

\begin{comment}
\begin{figure}[H]
  \centering
    \includegraphics[scale=0.6]{50-anexos/B-uso/identificacion_propio_configurar.png} 
    \caption{Ejemplo de como configurar un LoginFragment propio.}
\end{figure}	
\end{comment}

\begin{lstlisting}[language=Java, frame=tlbr, caption=Ejemplo de como configurar un LoginFragment propio (línea 6).]	
@Override
protected void onCreate(Bundle savedInstanceState) {
	super.onCreate(savedInstanceState);

	// Configurar AuthenticationManager para que use un LoginFragment propio
	AuthenticationManager.setLoginFragmentClass(MyCustomLoginFragment.class);
  
	// resto de las configuraciones
	// ...
}
\end{lstlisting}


El proceso de login y la interacción con las APIs responsabilidad del desarrollador, pero después de que el usuario inicia sesión en la API seleccionada, es necesario llamar al método login() en la clase AuthenticationManager y al método onLogin() en el objeto mListener heredado:

\begin{comment}
\begin{figure}[H]
  \centering
    \includegraphics[scale=0.6]{50-anexos/B-uso/identificacion_loginOk.png} 
    \caption{Ejemplo de como llamar al método onLogin.}
\end{figure}	
\end{comment}

\begin{lstlisting}[language=Java, frame=tlbr, caption=Ejemplo de como llamar al método onLogin.]	
if (loginOK) {
	AuthenticationManager.login(user, getActivity().getApplicationContext());
	mListener.onLogin(user);
}
\end{lstlisting}



\subsubsection{Definiendo una clase User propia}

Es necesario crear una clase usuario propia que implemente la interfaz User por cada método de identificación que se use. Los objetos de dichas clases se usarán para llamar al método login() de la clase AuthenticationManager.

\begin{comment}
\begin{figure}[H]
  \centering
    \includegraphics[scale=0.6]{50-anexos/B-uso/identificacion_user_propio.png} 
    \caption{Ejemplo de una clase usuario propia.}
\end{figure}	
\end{comment}


\begin{lstlisting}[language=Java, frame=tlbr, caption=Ejemplo de una clase usuario propia.]	
public class EMailUser implements User {

    public static final String AUTHENTICATION_TYPE = "email";

    private String userName;
    private String email;

    public EMailUser(String userName, String email) {
        this.userName = userName;
        this.email = email;
    }

    @Override
    public String getAuthenticationType() {
        return AUTHENTICATION_TYPE;
    }

    @Override
    public String getUserName() {
        return userName;
    }

    @Override
    public String getUserId() {
        return email;
    }

}
\end{lstlisting}




\begin{comment}

\section{Personalización}

\subsection{Temas y colores}

\subsection{Idiomas}

\end{comment}


\chapter{Los diferentes Steps y sus resultados (StepResult)} \label{sec:steps_detallados}
De manera predeterminada, Samplers provee los Steps que se explican a continuación. Además de los provistos por Samplers, un usuario programador puede definir y agregar sus propios Steps, junto con sus StepFragments y StepResults como se explica en la sección \ref{sec:definir_steps} 

Todos los Step tienen los siguientes campos:

\begin{itemize}
\item \textbf{id:} El identificador del Step, que debe ser único por instancia.
\item \textbf{type:} El tipo de Step, que debe ser uno de los que se detallan a continuación.
\item \textbf{nextStepId:} [Opcional] El identificador del siguiente Step que se debe ejecutar al finalizar con el actual. Si no se especifica, se asume que el Step es el final del Workflow.
\item \textbf{helpFileName:} [Opcional] El nombre del archivo HTML con la ayuda del Step (como se explicó en la sección \ref{sec:mostrar_ayuda}. Si no se especifica, no se mostrará el botón de ayuda para el Step.
\end{itemize}

Además de estos campos, los diferentes tipos de Step agregan campos adicionales como se detalla en la explicación de cada Step a continuación.


\section{PhotoStep: Tomar una foto}
PhotoStep se usa para solicitarle al científico ciudadano que tome una foto. Tiene un texto (text) que se muestran a modo de instrucciones o mensaje cuando la cámara esta encendida. Luego de tomada la foto, se muestra una vista preliminar de la misma, con un botón que da la opción de volver a tomar la foto.

\begin{figure}[H]
  \centering
    \includegraphics[scale=0.6]{50-anexos/C-steps/photo_json.png} 
    \caption{PhotoStep usando el generador de clases.}
\end{figure}	

\begin{figure}[H]
  \centering
    \includegraphics[scale=0.6]{50-anexos/C-steps/photo_java.png} 
    \caption{PhotoStep en Java.}
\end{figure}


\begin{figure}[H]
  \centering
    \includegraphics[scale=0.4]{05-implementacion/PhotoStep1.png} 
    \includegraphics[scale=0.4]{05-implementacion/PhotoStep2.png}     
   \caption{Ejemplo de un PhotoStep en ejecución. A la izquierda al momento de tomar la foto, y a la derecha al momento de mostrar la vista preliminar.}
   \label{fig:imgPhotoStep}
\end{figure}

\subsection{PhotoStepResult: El resultado de Tomar una foto}
El PhotoStepResult contiene el nombre del archivo de la foto tomada. La foto va como archivo jpg en la carpeta de la muestra.
Puede haber varias fotos si hay varios PhotoSteps en el Workflow.




\section{SoundRecordStep: Grabar sonido}
SoundRecordStep  se usa para solicitarle al científico ciudadano que grabe un sonido. Tiene un texto (text) que se muestran a modo de instrucciones o mensaje. Luego de grabado el sonido, el mismo se puede escuchar, y en todo caso volver a grabar otro sonido.


\begin{figure}[H]
  \centering
    \includegraphics[scale=0.6]{50-anexos/C-steps/sound_json.png} 
    \caption{SoundRecordStep usando el generador de clases.}
\end{figure}	

\begin{figure}[H]
  \centering
    \includegraphics[scale=0.6]{50-anexos/C-steps/sound_java.png} 
    \caption{SoundRecordStep en Java.}
\end{figure}


\subsection{SoundRecordStepResult: El resultado de Grabar sonido}
El SoundRecordStepResult contiene el nombre del archivo del sonido grabado. El sonido va como archivo mp4 en la carpeta de la muestra.
Puede haber varios sonidos si hay varios SoundRecordSteps en el Workflow.



\section{InformationStep: Mostrar información}
InformationStep se usa para mostrar una información (texto) al científico ciudadano, por ejemplo, para mostrar un mensaje de bienvenida o para explicar brevemente los pasos que seguirán a continuación para tomar la muestra.


\begin{figure}[H]
  \centering
    \includegraphics[scale=0.6]{50-anexos/C-steps/information_json.png} 
    \caption{InformationStep usando el generador de clases.}
\end{figure}	

\begin{figure}[H]
  \centering
    \includegraphics[scale=0.6]{50-anexos/C-steps/information_java.png} 
    \caption{InformationStep en Java.}
\end{figure}

\begin{figure}[H]
  \centering
    \includegraphics[scale=0.4]{05-implementacion/InformationStep.png} 
   \caption{Ejemplo de un InformationStep en ejecución}
   \label{fig:imgInformationStep}
\end{figure}


\subsection{InformationStepResult: El resultado de Mostrar información}
El resultado de mostrar información es nulo, es una clase vacía. Solo está para cerrar el circuito.





\section{SelectOneStep: Seleccionar una opción de un grupo de opciones}
SelectOneStep  se usa para solicitarle al científico ciudadano que seleccione una sola opción de varias opciones disponibles. Tiene un texto (title) que se muestran a modo de título o instrucciones. Las opciones disponibles para seleccionar son una lista de objetos SelectOneOption.
Cada objeto SelectOneOption tiene un id, textToShow y nextStepId, que se muestran en pantalla en forma de radio buttons.

Este caso particular de Step no tiene un identificador del siguiente Step a ejecutar (nextStepId), sino que cada opción seleccionable posee uno. Esto permite crear bifurcaciones en el Workflow, por ejemplo, se puede preguntar si se observa cierta característica y si la respuesta es si, pasar un PhotoStep para pedir que tome una foto de esa característica, pero si la respuesta es no, se puede saltar el paso de tomar una foto.

\begin{figure}[H]
  \centering
    \includegraphics[scale=0.6]{50-anexos/C-steps/select_one_json.png} 
    \caption{SelectOneStep usando el generador de clases.}
\end{figure}	

\begin{figure}[H]
  \centering
    \includegraphics[scale=0.6]{50-anexos/C-steps/select_one_java.png} 
    \caption{SelectOneStep en Java.}
\end{figure}


\subsection{SelectOneStepResult: El resultado de Seleccionar una opción de un grupo de opciones}
SelectOneStepResult contiene la opción seleccionada (un objeto SelectOneOption).





\section{MultipleSelectStep: Seleccionar varias opciones de un grupo de opciones}
MultipleSelectStep  se usa para solicitarle al científico ciudadano que seleccione una o más opciones de varias opciones disponibles. Tiene un texto (title) que se muestran a modo de título o instrucciones. Las opciones disponibles para seleccionar son una lista de objetos MultipleSelectOption.
Cada objeto MultipleSelectOption tiene un id, textToShow.
Las opciones disponibles para seleccionar se muestran en pantalla en forma de check-boxes.


\begin{figure}[H]
  \centering
    \includegraphics[scale=0.6]{50-anexos/C-steps/multiple_select_json.png} 
    \caption{MultipleSelectStep usando el generador de clases.}
\end{figure}	

\begin{figure}[H]
  \centering
    \includegraphics[scale=0.6]{50-anexos/C-steps/multiple_select_java.png} 
    \caption{MultipleSelectStep en Java.}
\end{figure}


\subsection{MultipleSelectStepResult: El resultado de Seleccionar varias opciones de un grupo de opciones}
MultipleSelectStepResult contiene una lista de las opciones seleccionadas (una lista de objetos MultipleSelectOption).





\section{LocationStep: Posicionar la muestra en el mapa con el GPS}
LocationStep se usa para solicitarle al científico ciudadano que tome la posición con el GPS del dispositivo móvil. Tiene un texto (text) que se muestran a modo de instrucciones o mensaje. 

Permite usar el GPS o posicionar la muestra manualmente en el mapa.

\begin{figure}[H]
  \centering
    \includegraphics[scale=0.6]{50-anexos/C-steps/location_json.png} 
    \caption{LocationStep usando el generador de clases.}
\end{figure}	

\begin{figure}[H]
  \centering
    \includegraphics[scale=0.6]{50-anexos/C-steps/location_java.png} 
    \caption{LocationStep en Java.}
\end{figure}

\subsection{LocationStepResult: El resultado de Posicionar la muestra en el mapa con el GPS}
LocationStepResult contiene la latitud y longitud (ambos de tipo double) de la posición GPS tomada.





\section{RouteStep: Grabar un recorrido en el mapa usando el GPS}
LocationStep se usa para solicitarle al científico ciudadano que grabe un recorrido con el GPS del dispositivo móvil. Tiene las siguientes propiedades:

\begin{itemize}
\item \textbf{text:} un texto que se muestran a modo de instrucciones o mensaje.
\item \textbf{interval:} [Opcional] el intervalo en milisegundos en el que se pedirán actualizaciones del GPS. Si no se especifica el valor por defecto es 5000.
\item \textbf{mapZoom:} [Opcional] el zoom con el que se muestra el mapa. Si no se especifica el valor por defecto es 15.
\end{itemize}


\begin{figure}[H]
  \centering
    \includegraphics[scale=0.6]{50-anexos/C-steps/route_json.png} 
    \caption{RouteStep usando el generador de clases.}
\end{figure}	

\begin{figure}[H]
  \centering
    \includegraphics[scale=0.6]{50-anexos/C-steps/route_java.png} 
    \caption{RouteStep en Java.}
\end{figure}


\subsection{RouteStepResult: El resultado de Grabar un recorrido en el mapa usando el GPS}
RouteStepResult tiene una lista de objetos de la clase Location proporcionada por Android.  Para
mas información sobre esta clase ver la documentación oficial de Android sobre Location \footnote{https://developer.android.com/reference/android/location/Location}.






\section{InsertTextStep: Ingresar texto}
InsertTextStep se usa para solicitarle al científico ciudadano que ingrese una anotación, que puede ser texto o números. Tiene las siguientes propiedades:

\begin{itemize}
\item \textbf{text:} un texto que se muestran a modo de instrucciones o mensaje.
\item \textbf{sampleText:} un texto, a modo de muestra.
\item \textbf{maxLength:} establece la cantidad máxima de caracteres permitidos.
\item \textbf{inputType:} indica el tipo de caracteres admitidos. Los valores pueden ser \textbf{text} para el ingreso de texto, \textbf{number} para el ingreso de número enteros, \textbf{decimal} para el ingreso de números con decimales.
\item \textbf{optional:} un booleano que indica si se puede dejar vacío (true) o si es obligatorio que ingrese una anotación (false).
\end{itemize}

\begin{figure}[H]
  \centering
    \includegraphics[scale=0.6]{50-anexos/C-steps/insert_text_json.png} 
    \caption{InsertTextStep usando el generador de clases.}
\end{figure}	

\begin{figure}[H]
  \centering
    \includegraphics[scale=0.6]{50-anexos/C-steps/insert_text_java.png} 
    \caption{InsertTextStep en Java.}
\end{figure}


\subsection{InsertTextStepResult: El resultado de Ingresar texto}
InsertTextStepResult contiene el String con la anotación ingresada.





\section{InsertDateStep e InsertTimeStep: Ingresar fecha y hora}
InsertDateStep e InsertTimeStep se usan para solicitarle al científico ciudadano que ingrese una fecha y una hora respectivamente. Tienen un texto que se muestran a modo de instrucciones o mensaje.


\begin{figure}[H]
  \centering
    \includegraphics[scale=0.6]{50-anexos/C-steps/insert_date_time_json.png} 
    \caption{InsertDateStep e InsertTimeStep usando el generador de clases.}
\end{figure}	

\begin{figure}[H]
  \centering
    \includegraphics[scale=0.6]{50-anexos/C-steps/insert_date_time_java.png} 
    \caption{InsertDateStep e InsertTimeStep en Java.}
\end{figure}


\subsection{InsertDateStepResult e InsertTimeStep: El resultado de Ingresar fecha y hora}
InsertDateStepResult e InsertTimeStep tienen un objeto Date de Java que tiene la fecha o la hora según corresponda.






\section{Definir un nuevo Step, StepFragment y StepResult} \label{sec:definir_steps}
[TO-DO: acá vamos a explicar como definir un nuevo Step para extender el framework]




\chapter{Instanciación del framework: Caso de uso}

En esta sección se presenta un caso de uso del framework Samplers tomando como ejemplo una app de ciencia ciudadana que ya se encuentra en funcionamiento, que es AppEAR, y se generará una versión usando Samplers, haciendo una breve comparación entre ambas.

\section{AppEAR}
AppEAR un sistema de ciencia ciudadana para cuidar y aprender de los ambientes acuáticos en Argentina, realizado por Joaquín Cochero, investigador del CONICET en el Instituto Platense de Limnología. Los científicos ciudadanos interactúan y colaboran en el proyecto de varias maneras, mientras que aprenden sobre los ambientes y responden a objetivos científicos. El objetivo final de AppEAR es tener un relevamiento completo y detallado de las aguas continentales de todo el territorio nacional para conocer los lugares en riesgo en los que urge trabajar. 

Los voluntarios de este proyecto descargan una aplicación para su dispositivo móvil y toman muestras para el proyecto. La aplicación guía a los usuarios a través de los pasos necesarios para tomar una muestra.\cite{appEar}

\begin{figure}[H]
  \centering
   \includegraphics[scale=0.5]{06-caso_de_uso/capturas_appear.png} 
    \caption{Capturas de pantalla de la app AppEAR}
\end{figure}

Como se mencionó antes, la app va guiando a los científicos ciudadanos a través de los pasos necesarios para tomar una muestra, que son básicamente preguntas acerca del lugar que están relevando y de las cosas que ven alrededor. AppEAR diferencia 4 tipos de ambientes, que son río de llanura, río de montaña, estuario/playa y laguna, y en base a eso realiza algunas preguntas que son específicas de cada tipo y el resto preguntas comunes a los 4 tipos de ambientes. A medida que se van completando las preguntas, la app va armando una especie de dibujo del lugar con figuras que representan los elementos observados por el científico ciudadano.

Analizando todos los pasos propuestos por AppEAR se puede deducir el Workflow para tomar la muestra representado en el siguiente grafo direccional:

\begin{figure}[H] \label{img_grafo_appear}
  \centering
   \includegraphics[scale=0.65]{06-caso_de_uso/flujo_appear.png} 
    \caption{Grafo direccional que representa el Workflow de AppEAR}
\end{figure}

\section{AppEar usando Samplers}
A continuación se muestra como sería AppEAR usando Samplers.

Observando el Workflow planteado en la figura \ref{img_grafo_appear}, se genera el archivo de configuración para Samplers de la siguiente manera:

\begin{figure}[H]
  \centering
   \includegraphics[scale=0.3]{06-caso_de_uso/archivo_configuracion.png} 
    \caption{Archivo de configuración para Samplers para generar una app equivalente a AppEAR}
\end{figure}

Una vez completado el archivo de configuración, se ejecuta Samplers y se genera la app.

\begin{figure}[H]
  \centering
   \includegraphics[scale=0.5]{06-caso_de_uso/capturas_appear_samplers.png} 
    \caption{Capturas de pantalla de la app AppEAR generada por Samplers}
\end{figure}

\section{Comparación y conclusión}
[breve comparacion y conclusion]
\chapter{Conclusiones y Trabajo Futuro}

En este capítulo se recordarán los objetivos propuestos al comienzo de este trabajo, los resultados y en qué medida se cumplieron dichas metas. También se detallaran trabajos a futuro que contribuirían a mejorar aspectos del proyecto como puede ser mejorar su usabilidad, mantener el proyecto para que siga cumpliendo los estándares requeridos para aplicaciones móviles o ampliar la base de usuarios soportando iOS.

\section{Conclusiones}

En el comienzo de este proyecto se participó de una reunión que se dió en el marco de lo que fué el inicio de varios proyectos paralelos, entre ellos el de este trabajo, que iban a integrar Cientópolis. Asistió a esa reunión Joaquín Cochero, biólogo e investigador del CONICET en el Instituto Platense de Limnologıía y creador de al app AppEAR para relevar estuarios utilizando ciencia ciudadana. Es en esa reunión donde el Cochero cuenta que tuvo que, a pesar de tener conocimientos básicos de programación, tuvo que ampliarlos para poder desarrollar la app, y que luego de eso recibió solicitudes de varios colegas para que los asista, de ser posible, en el desarrollo de apps para ellos también poder incluir Ciencia Ciudadana en sus proyectos de investigación. Con esta reunión que devino un poco en relevamiento de requerimientos, se pudo observar que algunos investigadores estaban encontrando una traba tecnológica a la hora de incluir ciencia ciudadana en sus proyectos.
Habiendo relevado la necesidad de facilitar que los desarrollos de aplicaciones móviles sean más accesibles inspiró este desarrollo de este trabajo orientado a cumplir los objetivos detallados a continuación:


\begin{itemize} 
  \item \textbf{Desarrollar un framework que dado un archivo de configuración produzca una aplicación móvil}
   
   Como se detalla en el capítulo 5, el framework Samplers recibe un archivo de configuración dónde se especifican los pasos necesarios para recolectar una muestra y con ello produce el código de una aplicación móvil para Android. El archivo de configuración también debe tener información adicional que pueda ser necesaria en el proyecto, como credenciales para acceder a servicios de geolocalización brindados por Google. Samplers genera el código fuente de una app que puede ser utilizada inmediatamente; puede ser modificada para cambiar el estilo y utilizar otros colores o fuentes que no sean los default y también pueden incluirse como librería en una aplicación y utilizar sus clases libremente.
   
   \item \textbf{Capturar multimedia, pregunta con una o varias respuestas, fecha y hora, texto e información}

Samplers permite sacar fotos, grabar un audio, indicar una posición con coordenadas GPS asistidas por mapa, permite grabar un recorrido, preguntas con una o varias respuestas, ingreso de texto, fecha y hora. También permite mostrar información. No se pudo alcanzar el objetivo de permitir captura de video. No declarado entre los objetivos se agregó poder especificar ayuda que puede ser relevante a la aplicación o relacionada a la actividad que esté ejecutando el usuario. 

   \item \textbf{Instanciar una aplicación básica}

En el capítulo 6 se instancia una aplicación cuyo conjunto de pasos definidos coincide con la app AppEAR para documentar de qué manera puede construirse una aplicación con Samplers. Se concluye que el resultado es similar en cuánto a funcionalidad.

\end{itemize} 

\section{Trabajo Futuro}

En esta sección se proponen posibles caminos a seguir para que la aplicación evolucione, pero que quedan fuera del alcance de este trabajo. Poder compilar para iOS ampliaría la base de usuarios. Aunque en el país el uso de dispositivos iOS, como mencionamos previamente en la sección \ref{ccDispMoviles}, representa un 6\% en contraposición con el 93\% que representa Android, sigue siendo un porcentaje que queda excluido de la app que se puede crear con Samplers. Para desarrollar en iOS se debe contar con una computadora Mac que pueda ejecutar la última versión de Xcode. De la misma manera que Android Studio es el IDE que deben utilizar las apps desarrolladas para dispositivos Android, Xcode es el IDE para apps Apple que se ejecuten en Mac o en iOS. Para publicar la app en la App Store se debe ser miembro del Apple Developer Program. Teniendo en cuenta que es necesario contar con hardware específico \cite{appleDeveloper} y que se requiere unirse al Apple Developer Program que tiene un costo anual \cite{appleEnrollment} lo hace una mejora con un costo económico ya más alto que el que se necesita para desarrollar en Android. 

Android actualiza regularmente su sistema operativo y en cada nueva versión introduce cambios y mejoras. Un trabajo a futuro sería mantener el código fuente actualizado para ajustarse a los cambios del sistema operativo, aprovechar las mejoras que puedan llegar a existir y respetar los estándares de seguridad requeridos. Lo mismo si hiciera falta una adecuación a las políticas de privacidad, en el caso de que cambien a políticas más restrictivas. 

Una mejora ya realizada por los participantes del trabajo de tesina Samplers2 es Muestre.AR, una interfaz web que permite a los investigadores definir el workflow de recolección de una muestra utilizando un sitio web y descargar la aplicación Android resultante. Con las instrucciones definidas por el usuario investigador, el sitio puede generar el archivo de configuración e ingresarlo en Samplers para crear una aplicación Andorid y descargar la app. \cite{samplers2}









\include{50-anexos/anexos}
\chapter{Instalación del framework}
En esta sección se describe como instalar el framework en Android Studio; se detallan los requisitos y los pasos a seguir para la instalación.

\section{Requerimientos mínimos:}

\begin{itemize}
\item Android Studio (Java): Si bien esta pensado para y probado en Android Studio, podria funcionar bien en otro entorno que use Java y deje importar archivos Android Archive (.aar).
\item Android SDK API17: Android 4.2 (Jelly Bean) o superior.
\end{itemize}

\section{Pasos para la Instalación:}

\begin{enumerate}
	\item Crear en Android Studio un nuevo proyecto vacío (sin ninguna Activity).
		\begin{itemize}
		\item Seleccionar API17 o superior como versión mínima de Android SDK
		\end{itemize}
	\item Importar la librería del framework en el proyecto creado
		\begin{itemize}
		\item Descargar la última versión de \textbf{samplersFramework.aar} desde https://github.com/cientopolis/samplers/releases/
		\item Importar la librería al proyecto: File -$>$ New -$>$ New Module -$>$ Import .JAR/.AAR Package
		\end{itemize}
	\item Agregar el repositorio de Google
		\begin{itemize}
			\item En el archivo build.gradle del proyecto agregar: 
			\begin{lstlisting}[language=XML, frame=single]
allprojects {
    repositories {
        jcenter()
        google()
    }
}
			\end{lstlisting}	
		\end{itemize}
	\item Agregar las dependencias necesarias:
		\begin{itemize}
			\item En el archivo build.gradle de la aplicación agregar: 
			\begin{lstlisting}[language=XML, frame=tlb]
dependencies {
  // acá estarían las dependencias predeterminadas creadas por Android Studio:
  // ...

  // si no son agregadas automáticamente, agregar las siguientes dependencias:
  // (se deberían agregar las de la ultima versión disponible)
  compile 'com.android.support:design:24.2.1' 
  compile 'com.android.support.constraint:constraint-layout:1.0.2'

  // para usar mapas y los servicios de geolocalización se deben agregar las 
  // siguientes dependencias (usar la ultima versión disponible)
  compile ('com.google.android.gms:play-services-location:12.0.1')
  compile ('com.google.android.gms:play-services-maps:12.0.1')
  
  // para usar autenticación con Google se deben agregar las 
  // siguientes dependencias (usar la ultima versión disponible)  
  compile ('com.google.android.gms:play-services-auth:12.0.1')

  // agregar la dependencia del framework Samplers
  compile project(":samplersFramework")
}
			\end{lstlisting}	
		\end{itemize}	

	\item Instanciar:
		\begin{itemize}
		\item La instanciación puede ser manual o usando el generador de clases en Gradle como se explica en la siguiente sección.
		\end{itemize}
\end{enumerate}	








%Los artículos que se citan en la tesis, se incluyen en el archivo bibliografía.bib, en formato bibtex.
%En la sección bibliografía, van a aparecer automáticamente aquellos que se citan desde el texto 
%utilizando el comando \cite con la etiqueta correspondiente - en la introducción hay un par de ejemplos. 

\bibliographystyle{ieeetr}
\bibliography{90-bibliografia}{}
%\bibliographystyle{plain}

\listoffigures

\end{document}
\end

