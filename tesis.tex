\documentclass[11pt]{report}
\pagestyle{plain}
\usepackage[spanish]{babel}
\selectlanguage{spanish}
\usepackage[utf8]{inputenc}
\usepackage{listings}
\usepackage{color}
\usepackage[table]{xcolor}
\usepackage{graphicx}
\usepackage{amssymb}
\usepackage{titlesec}


% ------------------- Titulo y  Autor -----------------------------
\title{Samplers: Framework para construir aplicaciones Android para recolectar muestras en proyectos de Ciencia Ciudadana}
\author{Laura Lus y Javier Ramírez}

\begin{document}

\maketitle

\begin{abstract}
%Este es un resumen de la tesis muy corto (media carilla). El que lo lee se tiene que quedar con la idea de: 1) en que tema trabajaron, 2) cual es el problema que intentaron resolver, 3) que técnica aplicaron para resolverlo y por qué es novedosa, 4) que descubrieron en el proceso (un adelanto de conclusión). 

La Ciencia Ciudadana involucra a la comunidad en proyectos de investigación con tareas que pueden requerir poco o ningún conocimiento especial como puede ser contar los animales o galaxias que aparecen en una fotografía o responder una serie de preguntas para recabar información de un ambiente que el voluntario está observando como puede ser el ecosistema que rodea una laguna. Haciendo partícipe a los ciudadanos de proyectos de investigación científica se consigue que entiendan y aprecien la ciencia[A new dawn for citizen science - Silvertown]. Los proyectos de investigación que incluyen Ciencia Ciudadana pueden clasificarse en acción, conservación, recolección, virtual y educativos. El presente trabajo se concentra en los proyectos de recolección, que son los que requieren recolectar muestras del medio físico. Más específicamente las que requieren recolectar muestras haciendo uso de dispositivos móviles. Samplers es un framework Android para construir aplicaciones que permitan la recolección de muestras utilizando las herramientas que brindan los dispositivos móviles, como puede ser geolocalización o toma de fotografías.

\end{abstract}

\tableofcontents

%Cada capitulo será un archivo .tex el cual estará en su propia carpeta, con sus propias imágenes).
%El comando include hace que el capitulo respectivo se incluya en el documento. No es necesario indicar la extensión 
%del archivo dado que asume que es .tex
%Los números en el nombre de las carpetas son solo para tenerlas ordenadas.

\chapter{Introducción}

\label{introduccion}

%[¿en que tema están trabajando? ] Introduce en contexto en el que están trabajando (p.e., el ámbito en el que se dá el problema). Da definiciones (brevemente) de conceptos que aparecen en el contexto. 

Acá introducción de ciencia ciudadana y samplers \cite{wtf}

\section{ Estructura de la Tesina }
Este trabajo de tesina se organiza de la siguiente manera:
\begin{itemize} 
	\item{Capítulo 1} 
		\begin{description}
		 El propósito de este capítulo es explicar la estructura de esta tesina y dar un resumen de cada capítulo, resaltando sus principlaes objetivos.
		\end{description}


	\item{Capítulo 2} 
		\begin{description} 
		Este capítulo comienza explicando qué es la ciencia ciudadana y cómo es que su popularidad está en aumento. Detalla una clasificación de proyectos de ciencia ciudadana y luego ahonda en uno de los tipos de la clasificación, que son aquellos proyectos donde la inclusión de los científicos ciudadanos se realiza en la recolección de información o muestras. Se describe el método científico ya que los proyectos de investigación que utilizan ciencia ciudadana o cuyo diseño gira en torno a incluir a científicos ciudadanos son proyectos que utilizan las bases de la investigación científica, es decir, son proyectos que utilizan el método científico. A diferencia de los proyectos de investigación donde todos sus participantes son científicos o personas con conocimiento o experticia en el área de estudio, los proyectos de ciencia ciudadana deben tener especial cuidado definiendo los protocolos de recolección de la información para que sus participantes puedan seguirlos. Por último, teniendo en cuenta que los dispositivos móviles están al alcance de muchas personas, presentamos datos de uso de dispositivos móviles y sistemas operativos en el país.
		\end{description}
	
	\item{Capítulo 3} 
		\begin{description} 
		Se introduce el concepto de framework y se describe una clasificación en base a su diseño y tipo de especialización. Luego se describe el estado de tres herramientas que asisten a los investigadores en la creación y la administración de proyectos de recolección que utilizan ciencia ciudadana. Se analizan las ventajas y las desventajas de las herramientas descritas en la sección.
		\end{description}
	
	\item{Capítulo 4} 
		\begin{description} 
		Se describe el sistema operativo para dispositivos móviles Android y se detallan sus principales componentes de aplicación. La Activity, el componente principal de las aplicaciones en Android. Características, ciclo de vida y posibles estados. Relación entre Fragment y Activity. Ciclo de vida y estads del Fragment. Propósito de los Services y tipos soportados. Método de suscripción de eventos del sistema y de otras aplicaciones, BroadcastReceiver. 
		\end{description} 

	\item{Capítulo 5} 
		\begin{description} 
		Samplers, un framework para construir aplicaciones Android para proyectos de recolección que utilizan ciencia ciudadana. Alcance y descripción de la solución propuesta con Samplers. Workflow o protocolo para la recolección de la muestra. Descripción y ejemplo del archivo para configuración de una aplicación. Pasos para la recolección de la muestra: Step y Workflow y su relación con los principales componentes de las aplicaciones Android. Sample, la muestra, resultado de la ejecución del workflow por parte de un científico ciudadano. Envío de muestras por internet y persistencia local para envío manual o cuando tenga disponibilidad.
		\end{description}
		
	\item{Capítulo 7} 
		\begin{description} [Breve explicación de lo que se trata en el capitulo 7]
		\end{description}				
\end{itemize}

\section{ Motivación }

Los proyectos de investigación científica a menudo requieren la realización de gran número de actividades que son difíciles de automatizar como puede ser la clasificación de fotos, anotaciones, observaciones y todo tipo de actividades que en esencia son simples, pero consumen mucho tiempo. Muchas veces estas actividades son sencillas y no se necesita de ninguna preparación académica o escolarizada previa para realizarlas, por ejemplo indicar si en una foto se observa o no un animal. La ciencia ciudadana es una forma de investigación en colaboración que involucra a los ciudadanos resolviendo este tipo de tareas simples en proyectos de investigación científica que buscan resolver problemas del mundo real \cite{wiggins2011conservation}. 

Un científico ciudadano es un voluntario que recoge y/o procesa información como parte de una investigación científica \cite{silvertown2009new}. Para que los voluntarios puedan participar en estos proyectos es necesario brindarles herramientas que los ayuden a contribuir. 
Nuestro interés está enfocado en los proyectos de recolección. Estos proyectos de investigación científica requieren la recopilación de datos del medio físico. Una forma de asistir a estos proyectos es por medio de sistemas informáticos que posibiliten la recolección de datos usando móviles. Un ejemplo de este tipo de proyectos es AppEAR un sistema de ciencia ciudadana para cuidar y aprender de los ambientes acuáticos en Argentina, realizado por Joaquín Cochero, investigador del CONICET en el Instituto Platense de Limnología. El objetivo final de AppEAR es tener un relevamiento completo y detallado de aguas continentales de todo el territorio nacional para conocer los lugares en riesgo en los que urge trabajar. Los voluntarios de este proyecto descargan una aplicación para su dispositivo móvil y toman muestras para el proyecto. La aplicación guía a los usuarios a través de los pasos necesarios para tomar una muestra.

La mayoría de los proyectos de ciencia ciudadana de recolección cuentan con aplicaciones desarrolladas específicamente para cada proyecto, en donde el principal problema a resolver es la secuencia de pasos que conforman el protocolo para la toma de la muestra y la combinación de este protocolo y de las herramientas del dispositivo móvil que se desean utilizar cómo puede ser la cámara, el GPS, el micrófono para grabar un audio. Consideramos que proveer un framework que resuelva esta problemática, la de la aplicación específica de cada proyecto, sería útil para la creciente comunidad de científicos que quieren incluir ciencia ciudadana en sus proyectos.

Este proyecto se enmarca dentro de Cientópolis\cite{cientopolis}, una plataforma para la promoción y el estudio de la Ciencia Ciudadana. Cientópolis se nuclea como un proyecto de investigación desde la Facultad de Informática de la UNLP pero articula su funcionamiento con investigadores de las facultades de Ciencias Astronómicas y Geofísicas, Humanidades y Ciencias de la Educación, Bellas Artes y Ciencias Naturales y Museo.

\section{ Objetivos }		
		
Se propone desarrollar un framework para instanciar aplicaciones móviles Android de ciencia ciudadana. El framework recibirá un archivo con la configuración requerida en formato JSON y generará una aplicación para ejecutarse en un dispositivo Android. En este archivo estará el conjunto de pasos que especifican el protocolo de recolección de muestras. Estos pasos pueden ser:
			\begin{itemize}
				\item captura de una foto, un video, un audio, una ubicación o un recorrido hecho con el dispositivo móvil.
				\item contestar una pregunta con respecto a la muestra. Esta pregunta puede tener una o múltiples respuestas posibles.
				\item introducir anotaciones de texto.
				\item indicar una fecha y hora.
				\item mostrar información de orientación y ayuda para la toma de la muestra.
			\end{itemize}

La aplicación generada servirá para tomar muestras siguiendo el protocolo de recolección especificado y las almacenará y empaquetará en el dispositivo móvil hasta que pueda ser enviada a un servidor web.
		
Se define el formato del archivo de configuración de la aplicación y la información adicional necesaria, como pueden ser credenciales para acceder a los servicios de Google Services o el posicionamiento por GPS.

Instanciar una aplicación básica de ejemplo con el framework en base a un archivo de configuración, que permita tomar algunas muestras y enviarlas a un servidor web que estará configurado para dicho propósito.
			
\chapter{Marco Teórico}
\section{Ciencia  Abierta y Ciencia Ciudadana}

No es fácil definir el término Ciencia Abierta ya que abarca una multiplicidad de participantes, cada uno con su enfoque particular. La Ciencia Abierta afecta a investigadores,desarrolladores de políticas, desarrolladores y operadores de plataformas, editoriales y público interesado. 

\subsection{Ciencia Abierta}
Ciencia Abierta es un término que engloba otros que tienen que ver con la creación y difusión del conocimiento en el futuro. Tradicionalmente el sistema de creación y difusión del conocimiento está basado en la publicación en revistas científicas. Es por ello que antes de imprimir y difundir un artículo el mismo debe estar completo y correcto, por una cuestión de costos. Publicar en revistas científicas tiene un costo, y si el artículo no está completo y correcto no vale la pena publicarlo. Sin embargo el artículo impreso podría no ser el único formato que esté disponible, habiendo otros menos costosos y que permitirían la publicación de resultados parciales e incluso la corrección o los comentarios por parte de pares o de personas que están trabajando en el mismo tema. Internet tiene los medios para que esto sea posible a través de Wikis y blogs. De esta forma no haría falta esperar que una investigación esté completa para acceder a ella. \cite{bartling2014opening}
Volviendo a la terminología, desde el punto de vista del libro "Opening Science" \cite{bartling2014opening} se proponen 5 corrientes de pensamiento:
\begin{itemize}
	\item {Escuela Pública}
	La Escuela Pública propone que la ciencia debe estar disponible para una audiencia mayor de interesados. Con las herramientas sociales de la web los investigadores pueden hacer público el proceso de investigación y preparar los resultados de su investigación para los interesados, no necesariamentes expertos en el área. Hay dos corrientes: una comprometida con el proceso de investigación (la producción) y otra con los resultados (el producto).
	\item {Escuela Democrática}
	\begin{itemize}
		\item{Datos Abiertos}	
		\item{Acceso Abierto}
	\end{itemize}
	\item {Escuela Pragmática}
	Los partidarios de la escuela pragmática se enfocan en lograr que el proceso de investigación sea más eficiente. Considera a la investigación científica como un proceso que puede ser optimizado:
	\begin{itemize}
		\item modularizando
		\item abriendo la "cadena de producción"
		\item incluyendo herramientas externas
		\item permitiendo la colaboración a través de internet
	\end{itemize}	
\end{itemize}	
	
\subsection{Cómo participan los ciudadanos?}

\begin{itemize}
	\item Ciencia Ciudadana
	\begin{itemize}
		\item Cómo participan los ciudadanos en la ciencia?        
		Esto de que hay que asignarles tareas acordes o darles una pequeña capacitación o ayuda en pantalla. Comunidades que hacen de soporte de voluntarios.
		\item De qué depende que un proyecto incluya ciencia ciudadana?
		Tipología de los proyectos de ciencia ciudadana, por ejemplo, que sea de educación, de investigación, que no cualquier proyecto puede tilizar ciencia ciudadana y no siempre se aplica en todo el proyecto. Muchas veces los ciudadanos colaboran con una parte.
	\end{itemize}   
	\item Ciencia Abierta   
	\begin{itemize}
		\item Por qué es importante la ciencia abierta? democracia y cuestiones políticas. Acceso público a la información de interés general.
		\item Qué relación tiene con la ciencia ciudadana? Básicamente los participantes en proyectos de investigación de ciencia ciudadana lo hacen por interés en el tema de investigación. Es una buena práctica que una vez finalizada la investigación se haga una devolución de los resultados de la misma para que los ciudadanos participantes quienes estaban interesados en el tema de movida puedan ver los resultados de la investigación. Este tema está directamente relacionado con la ciencia abierta que básicamente es abrir los datos, resultados y procesos utilizados para conseguir resultados a el público general.
	\end{itemize}
\end{itemize}

\section{ Dispositivos Móviles y Android }
\begin{itemize}
	\item Distribución de dispositivos móviles entre la población
	cantidad de personas que tienen dispositivos móviles. Que porcentaje de la población representan. Zonas de concentración de dispositivos:cómo están distribuidos
	\item Características de los dispositivos móviles
	cámaras, micrófonos, conexiones a redes, posibilidad de transferencias de archivos, navegabilidad en la interfaz de aplicación.
	\item Android
	el sistema operativo. Licencia. Estructura. Versiones y lo que ello implica.
	\item Ejemplos de aplicaciones de ciencia ciudadana y dispositivos móviles
	hablemos del ejemplo africano que no tenía palabras para que la población partipe sin necesidad de saber leer o escribir. AppEAR y Cazamosquitos. Ejemplo aplicado a salud Colombia
\end{itemize}

\section{ Frameworks }

\begin{itemize}
	\item Frameworks para construir aplicaciones
	\item Configuración de aplicaciones mediante archivos
\end{itemize}

%Hacer una tesis implica encontrar una pregunta que valga la pena responder o un problema que valga la pena resolver y darle respuesta o solución. Es una tarea de investigación que tiene como aspecto muy importante conocer lo que ya existe alrededor de la pregunta o problema que se elige. 

%Al llegar a este capitulo, el lector tiene una idea de cual es el problema. Seguramente se imagina problemas similares o soluciones al problema. El objetivo, en este momento, es convencerlo de que conocemos el problema y otros similares; que conocemos las formas en las que se lo ha intentado resolver (o a problemas similares); y que aún después de saber todo eso sigue siendo un problema importante, difícil y que nadie resolvió 8o nadie revolvió tan bien como nosotros).

%Para escribir este capitulo hay que leer. Hay que buscar soluciones a problemas similares y compararlas con lo que nosotros queremos hacer. Si sabemos que la nuestra es mejor, ya podemos marcar cuales son los puntos débiles de las existentes. También se puede escribir un poco sobre otras investigaciones, que si bien no atacaron problemas parecidos, pueden ser aplicadas a resolver parte de este. 

%Este capitulo es bueno ir escribiendolo en borrador cada vez que se lee algo (un articulo por ejemplo). Por lo menos hay que escribir un resumen de un párrafo de lo leído (registrando la referencia en el archivo bibliografia.bib y citando dede acá), y dar nuestra opinión al respecto en términos de su relación con el problema de nuestra tesis.

\chapter{Herramientas Utilizadas}
\label{estrategia}

\section{Android}
\begin{itemize}
	\item Versiones de Android y servicios de la API
	\item Android SDK
	\item Android Studio y Gradle
	\item Google Services
\end{itemize}	

\section{Wiki y Repositorio Git}
\begin{itemize}
	\item git
	\item issue tracking 
	\item wiki
	\item releases
\end{itemize}	

\section{Librerías Externas}
\begin{itemize}
	\item GSON para análisis de archivos JSON
	\item OkHttp para transferencia de datos
\end{itemize}	

%Un tipo de tesis común en Sistemas es la que propone una solución a un problema \footnote{hay otros tipos, por ejemplo aquellas que demuestran experimentalmente alguna cualidad de algún fenómeno}. Puede ser que el problema todavía no haya sido resuelto (poco probable); o puede ser que se proponga una solución que es mejor a las existentes en algún aspecto. Una forma interesante de imaginar el documento de tesis es como un espiral, que da cuatro vueltas, de adentro para afuera. En cada vuelta da mas detalles.

%\begin{itemize}
%\item Vuelta 1 (el resumen): se cuenta toda la tesis (problema, estrategia de solución, resultado obtenido) en un solo párrafo.
%\item Vuelta 2 (la introducción): En la introducción, se vuelve a contar el problema (ahora se introduce el contexto, se explica por que es un problema relevante y difícil, se dan algunas definiciones), se adelanta cual es la estrategia de solución aunque todavía no se puede explicar mucho, se listan las contribuciones principales.  
%\item Vuelta 3 (varios capítulos): Ahora se puede dedicar un capitulo completo a contar bien cual es la estrategia general (que método se aplica, que arquitectura, que tecnologías, que pasos tiene la solución, etc), y se puede dedicar un capitulo completo a cada parte interesante de la solución (esto depende mucho de lo que resuelvas y que partes imprtantes tenga).
%\end{itemize}  

%El capitulo de estrategia general tiene como objetivo contar cual es la estrategia/método de solución al problema elegido.  Por ejemplo, ¿se propone una metodología? ¿que pasos tiene? ¿Se construye un sistema? ¿que arquitectura tiene? ¿que partes importantes tiene? ¿que funcionalidad provee?

%Con este capítulo le debería alcanzar al que lee para entender como se resolvió el problema. Los capítulos que siguen a este pueden dar mas detalle sobre aquellos aspectos/partes que valga la pena detallar.  De alguna forma, este capitulo es el mapa que ordena los capítulos que siguen. 






\chapter{Samplers}

%Al capitulo \ref{estrategia}, que describe la estrategia general, lo siguen varios capítulos que entran en detalle en distintas partes de la solución. Uno, por ejemplo, puede describir el modelo de datos que utiliza la aplicación, otro puede describir el front-end de la aplicación, otro puede describir el algoritmo de recomendación de nuevos contenidos, etc. Por lo general hay que explicar en detalle aquellas cosas que no son obvias para quien no hizo la tesis y que las necesitaría si quiere reproducir lo que ustedes hicieron. 

%No es necesario escribir mucho. Simplemente es cuestión de preguntarse que le podría resultar interesante o novedoso a
%un compañero de estudio que no conoce el tema. 


\chapter{Conclusiones y Trabajo Futuro}

[falta desarrollar...]

%Supongamos que se quiso atacar el problema de la dificultad en el desarrollo de aplicaciones móviles multi-plataforma. Y que lo que se hizo fue desarrollar una librería de clases. ¿Cómo demuestro que la librería de clases resuelve el problema?

%Antes que nada, deberíamos haber dejado claro, en alguna sección del capítulo \ref{introduccion} cuales son los indicadores que miro para decir que hay "dificultad en el desarrollo de aplicaciones móviles". Por ejemplo, ¿cantidad de bugs específicos de la plataforma? ¿tiempo que lleva traducir los aspectos específicos?. Conocer esos indicadores (o aspectos) es importante para decidir a cuales de ellos voy a apuntar en mi solución (porque tal vez no puedo ser mejor en todos los aspectos). Es importante para poder comparar mi solución con otras. Y es importante porque en este capítulo tengo que demostrar que mi solución es mejor en términos del/los aspectos elegidos. 

%La evaluación se puede hacer de muchas formas y depende del caso en particular. Por ejemplo, podrías poner a varios compañeros a hacer la misma aplicación demo con tu librería y otras que ellos quieran. Y luego les hacés preguntas para saber si con tu libreria fué mejor. O podés contar la cantidad de bugs que se hicieron usando tu librería vs los que se hicieron sin ella. Hacer un experimento es complejo, pero hay muchas alternativas intermedias para que puedas demostrar que tu propuesta resuelve el problema planteado.







%\chapter{Conclusiones y trabajo futuro}

En el apartado de conclusiones se hace un breve resumen de lo que uno quiso hacer y lo que pudo hacer. Se sacan conclusiones respecto a la efectividad de la estrategia aplicada en resolver el problema. Se sacan conclusiones sobre la complejidad del problema y su importancia. Hasta se puede hablar de alternativas que probamos y no funcionaron. 
Esta sección es nuestra oportunidad para recordarle al lector que era un problema difícil y que pudimos resolverlo. Y de paso le recordamos la lista de contribuciones.  

Antes de terminar, hay que dejar algunas líneas para quien quiera continuar investigando el tema (trabajo futuro). ¿Qué hubiese sido bueno hacer y no se hizo porque no hubo tiempo? ¿Qué no funciona del todo bien y se podría mejorar? ¿Qué otras alternativas de solución se te ocurren ahora que podrían ser mejores? 






%Los artículos que se citan en la tesis, se incluyen en el archivo bibliografía.bib, en formato bibtex.
%En la sección bibliografía, van a aparecer automáticamente aquellos que se citan desde el texto 
%utilizando el comando \cite con la etiqueta correspondiente - en la introducción hay un par de ejemplos. 

\bibliographystyle{plain}
\bibliography{90-bibliografia}


\end{document}
\end

