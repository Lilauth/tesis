\documentclass[11pt]{report}
\pagestyle{plain}
\usepackage[spanish]{babel}
\selectlanguage{spanish}
\usepackage[utf8]{inputenc}
\usepackage{listings}
\usepackage{color}
\usepackage[table]{xcolor}
\usepackage{graphicx}
\usepackage{amssymb}


% ------------------- Titulo y  Autor -----------------------------
\title{Samplers: Framework para construir aplicaciones Android para recolectar muestras en proyectos de Ciencia Ciudadana}
\author{Laura Lus y Javier Ramírez}

\begin{document}

\maketitle

\begin{abstract}
%Este es un resumen de la tesis muy corto (media carilla). El que lo lee se tiene que quedar con la idea de: 1) en que tema trabajaron, 2) cual es el problema que intentaron resolver, 3) que técnica aplicaron para resolverlo y por qué es novedosa, 4) que descubrieron en el proceso (un adelanto de conclusión). 

La Ciencia Ciudadana involucra al público en proyectos de investigación científica. Las tareas de un voluntario o ciudadano científico pueden ser simples y no necesitar ningún conocimiento especial, o pueden ser más complejas y requerir capacitación previa. Ejemplos de tareas en proyectos de ciencia ciudadana pueden ser contar elementos que aparecen en una fotografía (si aparece un determinado animal o si puede reconocer una galaxia) o bien responder una serie de preguntas para recolectar información sobre un ambiente que el voluntario está observando (podría ser el ecosistema que rodea una laguna o un estuario). 

Hacer partícipes a los ciudadanos de proyectos de investigación científica persigue varios fines, entre de ellos poder realizar investigaciones a gran escala temporal y espacial\cite{bonney2009citizen}, brindar la oportunidad de participar en proyecto reales e interactuar con científicos, o bien perseguir fines educativos. Los proyectos de investigación que incluyen Ciencia Ciudadana pueden clasificarse en acción, conservación, recolección, virtual y educativos \cite{wiggins2011conservation}. 

El presente trabajo se concentra en los proyectos de recolección, que son los que requieren recolectar muestras del medio físico. Más específicamente las que requieren recolectar muestras haciendo uso de dispositivos móviles. 
Samplers es un framework Android para construir aplicaciones que permitan la recolección de muestras utilizando las herramientas que brindan los dispositivos móviles, como puede ser geolocalización o toma de fotografías.

\end{abstract}

\tableofcontents

%Cada capitulo será un archivo .tex el cual estará en su propia carpeta, con sus propias imágenes).
%El comando include hace que el capitulo respectivo se incluya en el documento. No es necesario indicar la extensión 
%del archivo dado que asume que es .tex
%Los números en el nombre de las carpetas son solo para tenerlas ordenadas.

\chapter{Introducción}

\label{introduccion}

%[¿en que tema están trabajando? ] Introduce en contexto en el que están trabajando (p.e., el ámbito en el que se dá el problema). Da definiciones (brevemente) de conceptos que aparecen en el contexto. 

Acá introducción de ciencia ciudadana y samplers \cite{wtf}

\section{ Estructura de la Tesina }
Este trabajo de tesina se organiza de la siguiente manera:
\begin{itemize} 
	\item{Capítulo 1} 
		\begin{description}
		 El propósito de este capítulo es explicar la estructura de esta tesina y dar un resumen de cada capítulo, resaltando sus principlaes objetivos.
		\end{description}


	\item{Capítulo 2} 
		\begin{description} 
		Este capítulo comienza explicando qué es la ciencia ciudadana y cómo es que su popularidad está en aumento. Detalla una clasificación de proyectos de ciencia ciudadana y luego ahonda en uno de los tipos de la clasificación, que son aquellos proyectos donde la inclusión de los científicos ciudadanos se realiza en la recolección de información o muestras. Se describe el método científico ya que los proyectos de investigación que utilizan ciencia ciudadana o cuyo diseño gira en torno a incluir a científicos ciudadanos son proyectos que utilizan las bases de la investigación científica, es decir, son proyectos que utilizan el método científico. A diferencia de los proyectos de investigación donde todos sus participantes son científicos o personas con conocimiento o experticia en el área de estudio, los proyectos de ciencia ciudadana deben tener especial cuidado definiendo los protocolos de recolección de la información para que sus participantes puedan seguirlos. Por último, teniendo en cuenta que los dispositivos móviles están al alcance de muchas personas, presentamos datos de uso de dispositivos móviles y sistemas operativos en el país.
		\end{description}
	
	\item{Capítulo 3} 
		\begin{description} 
		Se introduce el concepto de framework y se describe una clasificación en base a su diseño y tipo de especialización. Luego se describe el estado de tres herramientas que asisten a los investigadores en la creación y la administración de proyectos de recolección que utilizan ciencia ciudadana. Se analizan las ventajas y las desventajas de las herramientas descritas en la sección.
		\end{description}
	
	\item{Capítulo 4} 
		\begin{description} 
		Se describe el sistema operativo para dispositivos móviles Android y se detallan sus principales componentes de aplicación. La Activity, el componente principal de las aplicaciones en Android. Características, ciclo de vida y posibles estados. Relación entre Fragment y Activity. Ciclo de vida y estads del Fragment. Propósito de los Services y tipos soportados. Método de suscripción de eventos del sistema y de otras aplicaciones, BroadcastReceiver. 
		\end{description} 

	\item{Capítulo 5} 
		\begin{description} 
		Samplers, un framework para construir aplicaciones Android para proyectos de recolección que utilizan ciencia ciudadana. Alcance y descripción de la solución propuesta con Samplers. Workflow o protocolo para la recolección de la muestra. Descripción y ejemplo del archivo para configuración de una aplicación. Pasos para la recolección de la muestra: Step y Workflow y su relación con los principales componentes de las aplicaciones Android. Sample, la muestra, resultado de la ejecución del workflow por parte de un científico ciudadano. Envío de muestras por internet y persistencia local para envío manual o cuando tenga disponibilidad.
		\end{description}
		
	\item{Capítulo 7} 
		\begin{description} [Breve explicación de lo que se trata en el capitulo 7]
		\end{description}				
\end{itemize}

\section{ Motivación }

Los proyectos de investigación científica a menudo requieren la realización de gran número de actividades que son difíciles de automatizar como puede ser la clasificación de fotos, anotaciones, observaciones y todo tipo de actividades que en esencia son simples, pero consumen mucho tiempo. Muchas veces estas actividades son sencillas y no se necesita de ninguna preparación académica o escolarizada previa para realizarlas, por ejemplo indicar si en una foto se observa o no un animal. La ciencia ciudadana es una forma de investigación en colaboración que involucra a los ciudadanos resolviendo este tipo de tareas simples en proyectos de investigación científica que buscan resolver problemas del mundo real \cite{wiggins2011conservation}. 

Un científico ciudadano es un voluntario que recoge y/o procesa información como parte de una investigación científica \cite{silvertown2009new}. Para que los voluntarios puedan participar en estos proyectos es necesario brindarles herramientas que los ayuden a contribuir. 
Nuestro interés está enfocado en los proyectos de recolección. Estos proyectos de investigación científica requieren la recopilación de datos del medio físico. Una forma de asistir a estos proyectos es por medio de sistemas informáticos que posibiliten la recolección de datos usando móviles. Un ejemplo de este tipo de proyectos es AppEAR un sistema de ciencia ciudadana para cuidar y aprender de los ambientes acuáticos en Argentina, realizado por Joaquín Cochero, investigador del CONICET en el Instituto Platense de Limnología. El objetivo final de AppEAR es tener un relevamiento completo y detallado de aguas continentales de todo el territorio nacional para conocer los lugares en riesgo en los que urge trabajar. Los voluntarios de este proyecto descargan una aplicación para su dispositivo móvil y toman muestras para el proyecto. La aplicación guía a los usuarios a través de los pasos necesarios para tomar una muestra.

La mayoría de los proyectos de ciencia ciudadana de recolección cuentan con aplicaciones desarrolladas específicamente para cada proyecto, en donde el principal problema a resolver es la secuencia de pasos que conforman el protocolo para la toma de la muestra y la combinación de este protocolo y de las herramientas del dispositivo móvil que se desean utilizar cómo puede ser la cámara, el GPS, el micrófono para grabar un audio. Consideramos que proveer un framework que resuelva esta problemática, la de la aplicación específica de cada proyecto, sería útil para la creciente comunidad de científicos que quieren incluir ciencia ciudadana en sus proyectos.

Este proyecto se enmarca dentro de Cientópolis\cite{cientopolis}, una plataforma para la promoción y el estudio de la Ciencia Ciudadana. Cientópolis se nuclea como un proyecto de investigación desde la Facultad de Informática de la UNLP pero articula su funcionamiento con investigadores de las facultades de Ciencias Astronómicas y Geofísicas, Humanidades y Ciencias de la Educación, Bellas Artes y Ciencias Naturales y Museo.

\section{ Objetivos }		
		
Se propone desarrollar un framework para instanciar aplicaciones móviles Android de ciencia ciudadana. El framework recibirá un archivo con la configuración requerida en formato JSON y generará una aplicación para ejecutarse en un dispositivo Android. En este archivo estará el conjunto de pasos que especifican el protocolo de recolección de muestras. Estos pasos pueden ser:
			\begin{itemize}
				\item captura de una foto, un video, un audio, una ubicación o un recorrido hecho con el dispositivo móvil.
				\item contestar una pregunta con respecto a la muestra. Esta pregunta puede tener una o múltiples respuestas posibles.
				\item introducir anotaciones de texto.
				\item indicar una fecha y hora.
				\item mostrar información de orientación y ayuda para la toma de la muestra.
			\end{itemize}

La aplicación generada servirá para tomar muestras siguiendo el protocolo de recolección especificado y las almacenará y empaquetará en el dispositivo móvil hasta que pueda ser enviada a un servidor web.
		
Se define el formato del archivo de configuración de la aplicación y la información adicional necesaria, como pueden ser credenciales para acceder a los servicios de Google Services o el posicionamiento por GPS.

Instanciar una aplicación básica de ejemplo con el framework en base a un archivo de configuración, que permita tomar algunas muestras y enviarlas a un servidor web que estará configurado para dicho propósito.
			
\chapter{Marco Teórico}
		
	
\section{Introducción a la Ciencia Ciudadana}
    
    Ciencia Ciudadana es un termino que engloba las diferentes maneras en las que los ciudadanos participan en ciencia. Estas actividades pueden ser enviar información desde aplicaciones instaladas en sus dispositivos móviles acerca de especies invasivas, comportamiento de las aves, la cantidad de mariposas presentes al comienzo de la primavera o cualquier otro tema generalmente relacionado a la ecología y la conservación; como así también participar en debates locales de políticas que afectan directamente el ecosistema de una determinada ciudad o zona, como puede ser el fracking, la minería y los pesticidas en la agricultura y de esta manera influir en las políticas locales respecto de cómo regular esas actividades.\cite{envCitizenScience}
    
	Algunos de estos proyectos cuantan con una larga historia de investigación (como el CBC, Christmas Bird Count), otros con gran cantidad de participantes, quienes interactúan a través de foros y aplicaciones web desde la comodidad de sus hogares (como sucede en proyectos como Zooniverse) o bien abarcando grandes extensiones geográficas (como puede ser en The Big Butterfly Count). \cite{shirk2012public} 
		
	Aunque la ciencia ciudadana no es un concepto nuevo, su notoriedad es relativamente reciente. Los científicos ciudadanos o voluntarios ahora participan de proyectos relacionados con el cambio climático, las especies invasivas, la conservación de ecosistemas biológicos, el monitoreo de la calidad del agua y varios tópicos más. Esta popularidad se vio impulsada principalmente por tres factores:

\begin{itemize}
	\item {Disponibilidad Tecnológica}
	La disponibilidad de herramientas tecnológicas que permitan distribuir información acerca de los proyectos y también permitan recolectar la información generada por los voluntarios. De estas herramientas internet es la más representativa, pero la tecnología móvil está jugando un rol fundamental con la popularización de smartphones. \cite{silvertown2009new}
	\item {Reconocimiento del Aporte de los Voluntarios por parte de los Profesionales}
	La participación de voluntarios en un proyecto de investigación aporta diferentes atributos que pueden ser trabajo, poder de cómputo o habilidad \cite{cohn2008citizen}
	\item {Inclusión del Público General en Proyectos Científicos}
	La mejor manera para que el común de la gente entienda y se involucre en proyectos científicos es siendo parte de ellos. \cite{silvertown2009new}
\end{itemize} 

	En ciencias como pueden ser la arqueología, la astronomía o las ciencias naturales la capacidad de observación es a veces es más importante que el equipamiento costoso. Los voluntarios o científicos ciudadanos realizan actividades como parte de un proyecto científico que está especialmente diseñado o bien fue adaptado para que cumplan un rol, ya sea para fines educativos de los mismos voluntarios o para beneficio del proyecto. En los mejores ejemplos, se benefician ambos: los voluntarios y el proyecto.\cite{silvertown2009new}
	Para ello vamos a utilizar la clasificación provista por Wiggins and Crowston para tener un marco de lo que son los proyectos de recolección según esta clasificación, y por qué son los que mejor se relacionan con los proyectos de Ciencia Ciudadana en donde los colaboradores actúan como investigadores de campo recolectando muestras para introducirel tema de la toma y el envío de muestras recolectadas mediante aplicaciones móviles.	

\section{Clasificación de los Proyectos de Ciencia Ciudadana}	

\begin{itemize}
	\item {Acción}
		
		Los proyectos clasificados en esta categoría no son planificados o iniciados por científicos, sino más bien por los ciudadanos, y generalmente requieren un compromiso a lo largo del tiempo en los problemas ambientales locales por lo cual las actividades científicas están orientadas al ambiente físico. 
		
		Estos proyectos solicitan la colaboración de científicos como asesores o consultores, y no como organizadores. Los datos o resultados obtenidos no persiguen un fin académico, sino más bien buscan fundamentar con evidencia para poder tomar acciones sobre alguna situación. 
	\item {Conservación} 
	
	Al igual que los proyectos de Acción, los proyectos de Conservación son fuertemente regionales, y las actividades de los voluntarios están enfocadas mayormente en la recolección de información. La mayoría de los proyectos de investigación tienen contenido o fines educativos. También tienden a ser de alcance regional, como lo reflejan sus desafíos y metas.
	
	Estos proyectos buscan generar información principalmente como fuente para la toma de decisiones relacionadas al manejo de recursos, y también buscan la promoción de la administración y reconocimiento del voluntariado. También prestan especial atención a la generación de información científicamente válida. Son esfuerzos de monitoreo a largo plazo y generalmente no tienen problemas de sustentabilidad, ya que son subvencionadas por fondos públicos o reciben ingresos de agencias que son las que nuclean los proyectos o son las interesadas en sus resultados.
	
	\item {Investigación o Recolección} 
	
	Los proyectos de investigación concentran su atención en investigaciones científicas cuyos objetivos requieren la recolección de información del medio físico. Este tipo de proyectos es el que mejor encaja en la definición de Ciencia Ciudadana. Y aunque los objetivos de educación no son los principales de estos proyectos, forman parte de ellos como material de capacitación o incluyendo estructuras de recolección que alientan el aprendizaje al aire libre. El alcance varía de regional a internacional, y puede lograr participación masiva de hasta decenas de miles de voluntarios y obtener millones de observaciones (de voluntarios) anuales. La mayoría de estos proyectos están enfocados en la investigación biológica, medioambiental o meteorológica, por dar algunos ejemplos. 
	
	Una de las principales preocupaciones de este tipo de proyectos es generar resultados científicamente válidos, ya que son concebidos para generar conocimiento formal y son mayormente organizados por científicos. El cuidadoso diseño del proyecto y de las tareas son los que permiten lograr resultados válidos. Aparte utilizan toda una serie de metodologías que permiten la validación de la información generada. Los voluntarios están dispersos geográficamente y esto es un recurso valioso ya que este tipo de proyectos intenta muchas veces registrar la distribución geográfica de determinadas especies o la ocurrencia de fenómenos naturales. 
	
	\item {Virtual} 
	
	En los proyectos de ciencia ciudadana denominados Virtuales todas las actividades son mediante tecnologías de la información y la comunicación, sin la intervención de elementos del entorno físico.
	
	Los proyectos que cumplen las condiciones para ser clasificados como virtuales provienen de la astronomía, la paleontología y la proteómica, que es una rama de la microbiología que estudia la estructura de las proteínas. Algunos proyectos de psicología podrían clasificar, pero no lo hacen porque los voluntarios colaboran como sujetos de pruebas, y esto no es formalmente considerado como colaboración en la investigación. 
	
	Galaxy Zoo es un ejemplo de proyecto virtual de ciencia ciudadana. Desde hace más de diez años los voluntarios que colaboran con el proyecto clasifican galaxias en fotografías. Responden una serie de preguntas respecto de la foto que están observando, y de esta manera los científicos encargados del proyecto obtienen una primera clasificación, que se construye en base a las observaciones de varios voluntarios de manera independiente. \cite{GalaxyZoo} 
	
	Al igual que en los proyectos de recolección anteriormente mencionados, los proyectos virtuales encuentran dificultades a la hora de conseguir resultados válidos en términos científicos. Estos resultados se obtienen mediante el desarrollo cuidadoso de las actividades. Como la participación es mayormente virtual, poder mantener a los voluntarios comprometidos con el proyecto es un desafío. Por eso muchas veces estos proyectos incluyen técnicas de gamificación, de competencia amigable entre participantes o de valoración de la contribución hecha por el voluntario (feedback).
	 
	\item {Educación} 
	
	Los proyectos de ciencia ciudadana pertenecientes a esta categoría son aquellos cuyo principal objetivo es educar. Los participantes de estos proyectos tienen como objetivo educar, y aportan recursos educativos informales; mientras los proyectos ofrecen material educativo formal. También, las actividades están pensadas para que el participante vaya acumulando conocimientos.
	
	Un ejemplo de este tipo de proyectos es Fossil Finders, que centra la investigación en el análisis de fósiles del Devónico (período de la era Paleozóica) proveyendo de materiales de estudio a estudiantes y profesores de escuelas secundarias. Los estudiantes van a identificar y medir fósiles en muestras de rocas enviadas a sus aulas. Luego ingresarán los datos obtenidos en una base de datos online, y podrán comparar sus datos con los de otras escuelas participantes. Con ello van a tener la oportunidad de involucrarse con métodos de investigación reales y de asistir a los investigadores del Instituto de Investigación Paleontológica a reconstruir el pasado geológico de Nueva York. \cite{FossilFinders}

	La mayoría de estos proyectos proyectos persiguen un fin educativo, el aprendizaje y el desarrollo de habilidades científicas. Por ello incluyen actividades de análisis de datos o muestras, brindando la posibilidad de desarrollar pensamiento crítico. 
	\end{itemize} 
	
	Los proyectos de Ciencia Ciudadana buscan resultados que generalmente caen en tres grandes categorías: resultados que sirven a la investigación; resultados que le sirven a los participantes como pueden ser adquirir nuevas habilidades o conocimientos y/o resultados que tienen que ver con sistemas socio-ecológicos, como es influenciar políticas, construir bases para la toma de decisiones en una comunidad o participar de acciones para la conservación del medio ambiente. \cite{shirk2012public}

\section{Proyectos de Recolección}	 
	En los proyectos de recolección, según la clasificación antes descripta, es en donde Samplers  brindaría su aporte, ya que está pensado para crear aplicaciones Android que sirvan para recolectar muestras en proyectos de Ciencia Ciudadana.
	
	Hay tres puntos en los que se debe enfatizar para que los voluntarios pueden recolectar y enviar información confiable: proveer información clara acerca de los protocolos de recolección, proveer formularios para el ingreso de datos que sean lo más lógicos y simples posibles y por último brindar soporte para que los participantes entiendan cómo seguir los protocolos y cómo enviar la información.\cite{bonney2009citizen}

\begin{itemize}
	\item {Protocolos}	
			Los datos que se obtienen en proyectos de Ciencia Ciudadana son recolectados mediante protocolos que especifican dónde, cuándo y cómo esos datos deben ser recolectados. Los protocolos deben definir un diseño formal o un plan de acción que permita combinar las muestras que fueron tomadas por múltiples participantes en diferentes ubicaciones para su posterior análisis. Los protocolos utilizados en proyectos de ciencia ciudadana deben ser fáciles de ejecutar, deben poder ser explicados de manera simple y directa, y deben ser desafiantes para los voluntarios.\cite{bonney2009citizen}
		Como se explica más adelante, la implementación del protocolo de recolección de muestras es el workflow de Samplers.
		
	\item {Formularios de ingreso de Datos}	
			En conjunto con un protocolo de recolección bien diseñado están los formularios de ingreso de datos. Los formularios en los que los usuarios ingresan sus observaciones deben ser fáciles de entender y completar. En cada paso de su wokflow, Samplers provee formularios de ingreso de datos que pueden ser preguntas de respuesta simple o compuesta (como componentes radio o checkbox), permite sacar fotografías o informar una ubicación entre otras funcionalidad provistas. 
			Es recomendable definir límites en el rango de datos que los formularios recogen para simplificar su posterior análisis. Por ejemplo, si la respuesta esperada a la pregunta 'Cuántos árboles cuenta en una cuadra' debería ser un número entre cero y 15 una respuesta como 150 podría indicar un error de tipeo. Entonces, es aconsejable pedirle al usuario que reporta que chequee si la información ingresada es correcta. De esta manera, los usuarios pueden revisar sus respuestas cvuando están fuera de los rangos establecidos. Aun así, suponiendo que el usuario indique un respuesta de esas características, es decir, una respuesta que se salga de los límites esperados; ese formulario debería guardarse con alguna marca que permita identificarlo para que los responsables del proyecto puedan analizarlo con más detalle y ver si se debe a un cambio en el entorno que está siendo observado o si es realmente un error de ingreso de datos del usuario.
			
	\item {Material Educativo}
		Los participantes deben ser provistos de material educativo para entender y seguir de manera satisfactoria los protocolos del proyecto. El material educativo puede incluir guías de identificación, posters, manuales, videos, podcasts, listas de correo y FAQ para que los participantes puedan consultar e incluso participar en foros y discusiones acerca del relevamiento que están haciendo o de cómo se espera que completen los formularios provistos. 
		Samplers ofrece ayuda en cada formulario o ventana de ingreso de datos que puede ser configurada para brindar información acerca de cómo se espera que el voluntario la complete.
		
\end{itemize} 

	
\section{Ciencia Ciudadana y Dispositivos Móviles}	

	Los proyectos de ciencia ciudadana para ampliar su alcance a múltiples lugares, lograr expandir su permanencia en el tiempo y llegar a diferentes escalas sociales necesitan adoptar las nuevas tecnologías. De esta manera, mediante la utilización de tecnologías móviles como smartphones y tablets, tienen la posibilidad de involucrar audiencias más amplias, motivar voluntarios, mejorar la recolección de información y controlar su calidad.\cite{newman2012future}
	(Estadísticas de distribución de móviles) 
	Los dispositivos móviles tienen integrados servicios de captura de imagen (cámaras) y de posicionamiento (GPS) que mejoran la frecuencia y la calidad de la información relevada.\cite{newman2012future}
	
	La proliferación de tecnologías móviles está enriqueciendo los ambientes urbanos en lo relacionado a sensing, proveyendo herramientas para recolectar datos y creando oportunidades para que el común de la gente pueda involucrarse en actividades científicas. En resumen, los dispositivos móviles son ideales para la recolección espontánea de información por el común de la gente. Ahora bien, detrás de su facilidad de uso y de su masividad hay una complejidad técnica y de infraestructura a la hora de desarrollar aplicaciones, que pueden significar una inversión de tiempo, dinero o ambas y pueden limitar el acceso de pequeñas organizaciones o proyectos a este tipo de desarrollos. \citep{kim2013sensr}
	
	



\chapter{Herramientas Utilizadas}
\label{estrategia}

\section{Android}
\begin{itemize}
	\item Versiones de Android y servicios de la API
	\item Android SDK
	\item Android Studio y Gradle
	\item Google Services
\end{itemize}	

\section{Wiki y Repositorio Git}
\begin{itemize}
	\item git
	\item issue tracking 
	\item wiki
	\item releases
\end{itemize}	

\section{Librerías Externas}
\begin{itemize}
	\item GSON para análisis de archivos JSON
	\item OkHttp para transferencia de datos
\end{itemize}	

%Un tipo de tesis común en Sistemas es la que propone una solución a un problema \footnote{hay otros tipos, por ejemplo aquellas que demuestran experimentalmente alguna cualidad de algún fenómeno}. Puede ser que el problema todavía no haya sido resuelto (poco probable); o puede ser que se proponga una solución que es mejor a las existentes en algún aspecto. Una forma interesante de imaginar el documento de tesis es como un espiral, que da cuatro vueltas, de adentro para afuera. En cada vuelta da mas detalles.

%\begin{itemize}
%\item Vuelta 1 (el resumen): se cuenta toda la tesis (problema, estrategia de solución, resultado obtenido) en un solo párrafo.
%\item Vuelta 2 (la introducción): En la introducción, se vuelve a contar el problema (ahora se introduce el contexto, se explica por que es un problema relevante y difícil, se dan algunas definiciones), se adelanta cual es la estrategia de solución aunque todavía no se puede explicar mucho, se listan las contribuciones principales.  
%\item Vuelta 3 (varios capítulos): Ahora se puede dedicar un capitulo completo a contar bien cual es la estrategia general (que método se aplica, que arquitectura, que tecnologías, que pasos tiene la solución, etc), y se puede dedicar un capitulo completo a cada parte interesante de la solución (esto depende mucho de lo que resuelvas y que partes imprtantes tenga).
%\end{itemize}  

%El capitulo de estrategia general tiene como objetivo contar cual es la estrategia/método de solución al problema elegido.  Por ejemplo, ¿se propone una metodología? ¿que pasos tiene? ¿Se construye un sistema? ¿que arquitectura tiene? ¿que partes importantes tiene? ¿que funcionalidad provee?

%Con este capítulo le debería alcanzar al que lee para entender como se resolvió el problema. Los capítulos que siguen a este pueden dar mas detalle sobre aquellos aspectos/partes que valga la pena detallar.  De alguna forma, este capitulo es el mapa que ordena los capítulos que siguen. 






\chapter{Samplers}

%Al capitulo \ref{estrategia}, que describe la estrategia general, lo siguen varios capítulos que entran en detalle en distintas partes de la solución. Uno, por ejemplo, puede describir el modelo de datos que utiliza la aplicación, otro puede describir el front-end de la aplicación, otro puede describir el algoritmo de recomendación de nuevos contenidos, etc. Por lo general hay que explicar en detalle aquellas cosas que no son obvias para quien no hizo la tesis y que las necesitaría si quiere reproducir lo que ustedes hicieron. 

%No es necesario escribir mucho. Simplemente es cuestión de preguntarse que le podría resultar interesante o novedoso a
%un compañero de estudio que no conoce el tema. 


\chapter{Conclusiones y Trabajo Futuro}

[falta desarrollar...]

%Supongamos que se quiso atacar el problema de la dificultad en el desarrollo de aplicaciones móviles multi-plataforma. Y que lo que se hizo fue desarrollar una librería de clases. ¿Cómo demuestro que la librería de clases resuelve el problema?

%Antes que nada, deberíamos haber dejado claro, en alguna sección del capítulo \ref{introduccion} cuales son los indicadores que miro para decir que hay "dificultad en el desarrollo de aplicaciones móviles". Por ejemplo, ¿cantidad de bugs específicos de la plataforma? ¿tiempo que lleva traducir los aspectos específicos?. Conocer esos indicadores (o aspectos) es importante para decidir a cuales de ellos voy a apuntar en mi solución (porque tal vez no puedo ser mejor en todos los aspectos). Es importante para poder comparar mi solución con otras. Y es importante porque en este capítulo tengo que demostrar que mi solución es mejor en términos del/los aspectos elegidos. 

%La evaluación se puede hacer de muchas formas y depende del caso en particular. Por ejemplo, podrías poner a varios compañeros a hacer la misma aplicación demo con tu librería y otras que ellos quieran. Y luego les hacés preguntas para saber si con tu libreria fué mejor. O podés contar la cantidad de bugs que se hicieron usando tu librería vs los que se hicieron sin ella. Hacer un experimento es complejo, pero hay muchas alternativas intermedias para que puedas demostrar que tu propuesta resuelve el problema planteado.








%Los artículos que se citan en la tesis, se incluyen en el archivo bibliografía.bib, en formato bibtex.
%En la sección bibliografía, van a aparecer automáticamente aquellos que se citan desde el texto 
%utilizando el comando \cite con la etiqueta correspondiente - en la introducción hay un par de ejemplos. 

\bibliographystyle{ieeetr}
\bibliography{90-bibliografia}


\end{document}
\end

