\documentclass[11pt]{report}
\pagestyle{plain}
\usepackage[spanish]{babel}
\selectlanguage{spanish}
\usepackage[utf8]{inputenc}
\usepackage{listings}
\usepackage{color}
\usepackage[table]{xcolor}
\usepackage{graphicx}
\usepackage{amssymb}
\usepackage{titlesec}


% ------------------- Titulo y  Autor -----------------------------
\title{Samplers: Framework para construir aplicaciones Android para recolectar muestras en proyectos de Ciencia Ciudadana}
\author{Laura Lus y Javier Ramírez}

\begin{document}

\maketitle

\begin{abstract}
%Este es un resumen de la tesis muy corto (media carilla). El que lo lee se tiene que quedar con la idea de: 1) en que tema trabajaron, 2) cual es el problema que intentaron resolver, 3) que técnica aplicaron para resolverlo y por qué es novedosa, 4) que descubrieron en el proceso (un adelanto de conclusión). 

La Ciencia Ciudadana involucra a la comunidad en proyectos de investigación con tareas que pueden requerir poco o ningún conocimiento especial como puede ser contar los animales o galaxias que aparecen en una fotografía o responder una serie de preguntas para recabar información de un ambiente que el voluntario está observando como puede ser el ecosistema que rodea una laguna. Haciendo partícipe a los ciudadanos de proyectos de investigación científica se consigue que entiendan y aprecien la ciencia[A new dawn for citizen science - Silvertown]. Los proyectos de investigación que incluyen Ciencia Ciudadana pueden clasificarse en acción, conservación, recolección, virtual y educativos. El presente trabajo se concentra en los proyectos de recolección, que son los que requieren recolectar muestras del medio físico. Más específicamente las que requieren recolectar muestras haciendo uso de dispositivos móviles. Samplers es un framework Android para construir aplicaciones que permitan la recolección de muestras utilizando las herramientas que brindan los dispositivos móviles, como puede ser geolocalización o toma de fotografías.

\end{abstract}

\tableofcontents

%Cada capitulo será un archivo .tex el cual estará en su propia carpeta, con sus propias imágenes).
%El comando include hace que el capitulo respectivo se incluya en el documento. No es necesario indicar la extensión 
%del archivo dado que asume que es .tex
%Los números en el nombre de las carpetas son solo para tenerlas ordenadas.

\chapter{Introducción}

\label{introduccion}

\section{Ciencia  Ciudadana y Ciencia Abierta}

\begin{itemize}
   \item Ciencia Ciudadana
   \begin{itemize}
        \item Cómo participan los ciudadanos en la ciencia?        
        Esto de que hay que asignarles tareas acordes o darles una pequeña capacitación o ayuda en pantalla. Comunidades que hacen de soporte de voluntarios.
        \item De qué depende que un proyecto incluya ciencia ciudadana?
        Tipología de los proyectos de ciencia ciudadana, por ejemplo, que sea de educación, de investigación, que no cualquier proyecto puede tilizar ciencia ciudadana y no siempre se aplica en todo el proyecto. Muchas veces los ciudadanos colaboran con una parte.
   \end{itemize}   
   \item Ciencia Abierta   
   \begin{itemize}
        \item \item Por qué es importante la ciencia abierta? democracia y cuestiones políticas. Acceso público a la información de interés general. 
        \item Qué relación tiene con la ciencia ciudadana? Básicamente los participantes en proyectos de investigación de ciencia ciudadana lo hacen por interés en el tema de investigación. Es una buena práctica que una vez finalizada la investigación se haga una devolución de los resultados de la misma para que los ciudadanos participantes quienes estaban interesados en el tema de movida puedan ver los resultados de la investigación. Este tema está directamente relacionado con la ciencia abierta que básicamente es abrir los datos, resultados y procesos utilizados para conseguir resultados a el público general.
   \end{itemize}
\end{itemize}

\section{ Dispositivos Móviles y Android }
\begin{itemize}
	\item Distribución de dispositivos móviles entre la población
	cantidad de personas que tienen dispositivos móviles. Que porcentaje de la población representan. Zonas de concentración de dispositivos:cómo están distribuidos
	\item Características de los dispositivos móviles
	cámaras, micrófonos, conexiones a redes, posibilidad de transferencias de archivos, navegabilidad en la interfaz de aplicación. 
    \item Android 
    el sistema operativo. Licencia. Estructura. Versiones y lo que ello implica.
    \item Ejemplos de aplicaciones de ciencia ciudadana y dispositivos móviles 
    hablemos del ejemplo africano que no tenía palabras para que la población partipe sin necesidad de saber leer o escribir. AppEAR y Cazamosquitos. Ejemplo aplicado a salud Colombia
\end{itemize}

\section{ Frameworks }

\begin{itemize}
	\item Frameworks para construir aplicaciones
    \item Configuración de aplicaciones mediante archivos
\end{itemize}

\begin{figure}
\begin{center}
\includegraphics[width=0.8\textwidth]{00-introduccion/zobel-page-46}
\caption{Extracto del libro Writing for Computers Science de Justin Zobel}
\label{zobel-page-46}
\end{center}
\end{figure}


\chapter{Marco Teórico}
\section{Ciencia  Abierta y Ciencia Ciudadana}

Escribir una introducción de ciencia ciudadana y ciencia abierta.

\subsection{Ciencia Abierta}
Ciencia Abierta es un término que engloba otros que tienen que ver con la creación y difusión del conocimiento en el futuro. Tradicionalmente el sistema de creación y difusión del conocimiento está basado en la publicación en revistas científicas. Es por ello que antes de imprimir y difundir un artículo el mismo debe estar completo y correcto. Esto se daba por una cuestión de costos. Publicar en revistas científicas tiene un costo, y si el artículo no está completo y correcto no vale la pena publicarlo. Sin embargo el artículo impreso podría no ser el formato elegido, habiendo otros formatos menos costosos y que permitrían la publicación de resultados parciales e incluso la corrección o los comentarios por parte de pares o de personas que están trabajando en el mismo tema. Internet tiene los medios para que esto sea posible a través de Wikis y blogs. De esta forma no haría falta esperar que una investigación esté completa para acceder a ella. \cite{bartling2014opening}
Volviendo a la terminología, ciencia abierta agrupa 5 corrientes de pensamiento:
\begin{itemize}
	\item {Escuela Pública}
	\item {Escuela Democrática}
	\begin{itemize}
		\item{Datos Abiertos}	
		\item{Acceso Abierto}
	\end{itemize}
	\item {Escuela Pragmática}
\end{itemize}	
	
\subsection{Cómo participan los ciudadanos?}

\begin{itemize}
	\item Ciencia Ciudadana
	\begin{itemize}
		\item Cómo participan los ciudadanos en la ciencia?        
		Esto de que hay que asignarles tareas acordes o darles una pequeña capacitación o ayuda en pantalla. Comunidades que hacen de soporte de voluntarios.
		\item De qué depende que un proyecto incluya ciencia ciudadana?
		Tipología de los proyectos de ciencia ciudadana, por ejemplo, que sea de educación, de investigación, que no cualquier proyecto puede tilizar ciencia ciudadana y no siempre se aplica en todo el proyecto. Muchas veces los ciudadanos colaboran con una parte.
	\end{itemize}   
	\item Ciencia Abierta   
	\begin{itemize}
		\item Por qué es importante la ciencia abierta? democracia y cuestiones políticas. Acceso público a la información de interés general.
		\item Qué relación tiene con la ciencia ciudadana? Básicamente los participantes en proyectos de investigación de ciencia ciudadana lo hacen por interés en el tema de investigación. Es una buena práctica que una vez finalizada la investigación se haga una devolución de los resultados de la misma para que los ciudadanos participantes quienes estaban interesados en el tema de movida puedan ver los resultados de la investigación. Este tema está directamente relacionado con la ciencia abierta que básicamente es abrir los datos, resultados y procesos utilizados para conseguir resultados a el público general.
	\end{itemize}
\end{itemize}

\section{ Dispositivos Móviles y Android }
\begin{itemize}
	\item Distribución de dispositivos móviles entre la población
	cantidad de personas que tienen dispositivos móviles. Que porcentaje de la población representan. Zonas de concentración de dispositivos:cómo están distribuidos
	\item Características de los dispositivos móviles
	cámaras, micrófonos, conexiones a redes, posibilidad de transferencias de archivos, navegabilidad en la interfaz de aplicación.
	\item Android
	el sistema operativo. Licencia. Estructura. Versiones y lo que ello implica.
	\item Ejemplos de aplicaciones de ciencia ciudadana y dispositivos móviles
	hablemos del ejemplo africano que no tenía palabras para que la población partipe sin necesidad de saber leer o escribir. AppEAR y Cazamosquitos. Ejemplo aplicado a salud Colombia
\end{itemize}

\section{ Frameworks }

\begin{itemize}
	\item Frameworks para construir aplicaciones
	\item Configuración de aplicaciones mediante archivos
\end{itemize}

%Hacer una tesis implica encontrar una pregunta que valga la pena responder o un problema que valga la pena resolver y darle respuesta o solución. Es una tarea de investigación que tiene como aspecto muy importante conocer lo que ya existe alrededor de la pregunta o problema que se elige. 

%Al llegar a este capitulo, el lector tiene una idea de cual es el problema. Seguramente se imagina problemas similares o soluciones al problema. El objetivo, en este momento, es convencerlo de que conocemos el problema y otros similares; que conocemos las formas en las que se lo ha intentado resolver (o a problemas similares); y que aún después de saber todo eso sigue siendo un problema importante, difícil y que nadie resolvió 8o nadie revolvió tan bien como nosotros).

%Para escribir este capitulo hay que leer. Hay que buscar soluciones a problemas similares y compararlas con lo que nosotros queremos hacer. Si sabemos que la nuestra es mejor, ya podemos marcar cuales son los puntos débiles de las existentes. También se puede escribir un poco sobre otras investigaciones, que si bien no atacaron problemas parecidos, pueden ser aplicadas a resolver parte de este. 

%Este capitulo es bueno ir escribiendolo en borrador cada vez que se lee algo (un articulo por ejemplo). Por lo menos hay que escribir un resumen de un párrafo de lo leído (registrando la referencia en el archivo bibliografia.bib y citando dede acá), y dar nuestra opinión al respecto en términos de su relación con el problema de nuestra tesis.

\include{20-herramientas/herramientas}
\include{30-solucion/solucion}
\chapter{Conclusiones y Trabajo Futuro}

En este capítulo se recordarán los objetivos propuestos al comienzo de este trabajo, los resultados y en qué medida se cumplieron dichas metas. También se detallaran trabajos a futuro que contribuirían a mejorar aspectos del proyecto como puede ser mejorar su usabilidad, mantener el proyecto para que siga cumpliendo los estándares requeridos para aplicaciones móviles o ampliar la base de usuarios soportando iOS.

\section{Conclusiones}

En el comienzo de este proyecto se participó de una reunión que se dió en el marco de lo que fué el inicio de varios proyectos paralelos, entre ellos el de este trabajo, que iban a integrar Cientópolis. Asistió a esa reunión Joaquín Cochero, biólogo e investigador del CONICET en el Instituto Platense de Limnologıía y creador de al app AppEAR para relevar estuarios utilizando ciencia ciudadana. Es en esa reunión donde el Cochero cuenta que tuvo que, a pesar de tener conocimientos básicos de programación, tuvo que ampliarlos para poder desarrollar la app, y que luego de eso recibió solicitudes de varios colegas para que los asista, de ser posible, en el desarrollo de apps para ellos también poder incluir Ciencia Ciudadana en sus proyectos de investigación. Con esta reunión que devino un poco en relevamiento de requerimientos, se pudo observar que algunos investigadores estaban encontrando una traba tecnológica a la hora de incluir ciencia ciudadana en sus proyectos.
Habiendo relevado la necesidad de facilitar que los desarrollos de aplicaciones móviles sean más accesibles inspiró este desarrollo de este trabajo orientado a cumplir los objetivos detallados a continuación:


\begin{itemize} 
  \item \textbf{Desarrollar un framework que dado un archivo de configuración produzca una aplicación móvil}
   
   Como se detalla en el capítulo 5, el framework Samplers recibe un archivo de configuración dónde se especifican los pasos necesarios para recolectar una muestra y con ello produce el código de una aplicación móvil para Android. El archivo de configuración también debe tener información adicional que pueda ser necesaria en el proyecto, como credenciales para acceder a servicios de geolocalización brindados por Google. Samplers genera el código fuente de una app que puede ser utilizada inmediatamente; puede ser modificada para cambiar el estilo y utilizar otros colores o fuentes que no sean los default y también pueden incluirse como librería en una aplicación y utilizar sus clases libremente.
   
   \item \textbf{Capturar multimedia, pregunta con una o varias respuestas, fecha y hora, texto e información}

Samplers permite sacar fotos, grabar un audio, indicar una posición con coordenadas GPS asistidas por mapa, permite grabar un recorrido, preguntas con una o varias respuestas, ingreso de texto, fecha y hora. También permite mostrar información. No se pudo alcanzar el objetivo de permitir captura de video. No declarado entre los objetivos se agregó poder especificar ayuda que puede ser relevante a la aplicación o relacionada a la actividad que esté ejecutando el usuario. 

   \item \textbf{Instanciar una aplicación básica}

En el capítulo 6 se instancia una aplicación cuyo conjunto de pasos definidos coincide con la app AppEAR para documentar de qué manera puede construirse una aplicación con Samplers. Se concluye que el resultado es similar en cuánto a funcionalidad.

\end{itemize} 

\section{Trabajo Futuro}

En esta sección se proponen posibles caminos a seguir para que la aplicación evolucione, pero que quedan fuera del alcance de este trabajo. Poder compilar para iOS ampliaría la base de usuarios. Aunque en el país el uso de dispositivos iOS, como mencionamos previamente en la sección \ref{ccDispMoviles}, representa un 6\% en contraposición con el 93\% que representa Android, sigue siendo un porcentaje que queda excluido de la app que se puede crear con Samplers. Para desarrollar en iOS se debe contar con una computadora Mac que pueda ejecutar la última versión de Xcode. De la misma manera que Android Studio es el IDE que deben utilizar las apps desarrolladas para dispositivos Android, Xcode es el IDE para apps Apple que se ejecuten en Mac o en iOS. Para publicar la app en la App Store se debe ser miembro del Apple Developer Program. Teniendo en cuenta que es necesario contar con hardware específico \cite{appleDeveloper} y que se requiere unirse al Apple Developer Program que tiene un costo anual \cite{appleEnrollment} lo hace una mejora con un costo económico ya más alto que el que se necesita para desarrollar en Android. 

Android actualiza regularmente su sistema operativo y en cada nueva versión introduce cambios y mejoras. Un trabajo a futuro sería mantener el código fuente actualizado para ajustarse a los cambios del sistema operativo, aprovechar las mejoras que puedan llegar a existir y respetar los estándares de seguridad requeridos. Lo mismo si hiciera falta una adecuación a las políticas de privacidad, en el caso de que cambien a políticas más restrictivas. 

Una mejora ya realizada por los participantes del trabajo de tesina Samplers2 es Muestre.AR, una interfaz web que permite a los investigadores definir el workflow de recolección de una muestra utilizando un sitio web y descargar la aplicación Android resultante. Con las instrucciones definidas por el usuario investigador, el sitio puede generar el archivo de configuración e ingresarlo en Samplers para crear una aplicación Andorid y descargar la app. \cite{samplers2}









%\include{80-conclusiones/conclusiones}


%Los artículos que se citan en la tesis, se incluyen en el archivo bibliografía.bib, en formato bibtex.
%En la sección bibliografía, van a aparecer automáticamente aquellos que se citan desde el texto 
%utilizando el comando \cite con la etiqueta correspondiente - en la introducción hay un par de ejemplos. 

\bibliographystyle{plain}
\bibliography{90-bibliografia}


\end{document}
\end

