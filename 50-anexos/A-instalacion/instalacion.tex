\chapter{Instalación del framework}
En esta sección se describe como instalar el framework en Android Studio; se detallan los requisitos y los pasos a seguir para la instalación.

\section{Requerimientos mínimos:}

\begin{itemize}
\item Android Studio (Java): Si bien esta pensado para y probado en Android Studio, podria funcionar bien en otro entorno que use Java y deje importar archivos Android Archive (.aar).
\item Android SDK API17: Android 4.2 (Jelly Bean) o superior.
\end{itemize}

\section{Pasos para la Instalación:}

\begin{enumerate}
	\item Crear en Android Studio un nuevo proyecto vacío (sin ninguna Activity).
		\begin{itemize}
		\item Seleccionar API17 o superior como versión mínima de Android SDK
		\end{itemize}
	\item Importar la librería del framework en el proyecto creado
		\begin{itemize}
		\item Descargar la última versión de \textbf{samplersFramework.aar} desde https://github.com/cientopolis/samplers/releases/
		\item Importar la librería al proyecto: File -$>$ New -$>$ New Module -$>$ Import .JAR/.AAR Package
		\end{itemize}
	\item Agregar el repositorio de Google
		\begin{itemize}
			\item En el archivo build.gradle del proyecto agregar: 
			\begin{lstlisting}[language=XML, frame=single]
allprojects {
    repositories {
        jcenter()
        google()
    }
}
			\end{lstlisting}	
		\end{itemize}
	\item Agregar las dependencias necesarias:
		\begin{itemize}
			\item En el archivo build.gradle de la aplicación agregar: 
			\begin{lstlisting}[language=XML, frame=tlb]
dependencies {
  // acá estarían las dependencias predeterminadas creadas por Android Studio:
  // ...

  // si no son agregadas automáticamente, agregar las siguientes dependencias:
  // (se deberían agregar las de la ultima versión disponible)
  compile 'com.android.support:design:24.2.1' 
  compile 'com.android.support.constraint:constraint-layout:1.0.2'

  // para usar mapas y los servicios de geolocalización se deben agregar las 
  // siguientes dependencias (usar la ultima versión disponible)
  compile ('com.google.android.gms:play-services-location:12.0.1')
  compile ('com.google.android.gms:play-services-maps:12.0.1')
  
  // para usar autenticación con Google se deben agregar las 
  // siguientes dependencias (usar la ultima versión disponible)  
  compile ('com.google.android.gms:play-services-auth:12.0.1')

  // agregar la dependencia del framework Samplers
  compile project(":samplersFramework")
}
			\end{lstlisting}	
		\end{itemize}	

	\item Instanciar:
		\begin{itemize}
		\item La instanciación puede ser manual o usando el generador de clases en Gradle como se explica en la siguiente sección.
		\end{itemize}
\end{enumerate}	





