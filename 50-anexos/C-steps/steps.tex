\section{Los diferentes Steps y sus resultados (StepResult)} \label{sec:steps_detallados}
De manera predeterminada, Samplers provee los Steps que se explican a continuación. Además de los provistos por Samplers, un usuario programador puede definir y agregar sus propios Steps, junto con sus StepFragments y StepResults como se explica en la sección \ref{sec:definir_steps} 

Todos los Step tienen los siguientes campos:

\begin{itemize}
\item \textbf{id:} El identificador del Step, que debe ser único por instancia.
\item \textbf{type:} El tipo de Step, que debe ser uno de los que se detallan a continuación.
\item \textbf{nextStepId:} [Opcional] El identificador del siguiente Step que se debe ejecutar al finalizar con el actual. Si no se especifica, se asume que el Step es el final del Workflow.
\item \textbf{helpFileName:} [Opcional] El nombre del archivo HTML con la ayuda del Step (como se explicó en la sección \ref{sec:mostrar_ayuda}. Si no se especifica, no se mostrará el botón de ayuda para el Step.
\end{itemize}

Además de estos campos, los diferentes tipos de Step agregan campos adicionales como se detalla en la explicación de cada Step a continuación.


\subsection{PhotoStep: Tomar una foto}
PhotoStep se usa para solicitarle al científico ciudadano que tome una foto. Tiene un texto (text) que se muestran a modo de instrucciones o mensaje cuando la cámara esta encendida. Luego de tomada la foto, se muestra una vista preliminar de la misma, con un botón que da la opción de volver a tomar la foto.

\begin{comment}
\begin{figure}[H]
  \centering
    \includegraphics[scale=0.6]{50-anexos/C-steps/photo_json.png} 
    \caption{PhotoStep usando el generador de clases.}
\end{figure}	
\end{comment}

\begin{lstlisting}[language=XML, frame=tlbr, caption=PhotoStep usando el generador de clases.]	
{
  "id" : 1,
  "type" : "Photo",
  "text" : "Tome una foto de su gato",
  "nextStepId" : 2
}
\end{lstlisting}

\begin{comment}
\begin{figure}[H]
  \centering
    \includegraphics[scale=0.6]{50-anexos/C-steps/photo_java.png} 
    \caption{PhotoStep en Java.}
\end{figure}
\end{comment}

\begin{lstlisting}[language=Java, frame=tlbr, caption=PhotoStep en Java.]	
Step step = new PhotoStep(1,"Tome una foto de su gato",2);
workflow.addStep(step);
\end{lstlisting}


\begin{figure}[H]
  \centering
   \includegraphics[scale=0.23]{50-anexos/C-steps/photo_screen1.jpg} 
   \includegraphics[scale=0.23]{50-anexos/C-steps/photo_screen2.jpg}    
    \caption{Ejemplo de un PhotoStep en ejecución. A la izquierda al momento de tomar la foto, y a la derecha al momento de mostrar la vista preliminar.}
\end{figure}

\subsubsection{PhotoStepResult: El resultado de Tomar una foto}
El PhotoStepResult contiene el nombre del archivo de la foto tomada. La foto va como archivo jpg en la carpeta de la muestra.
Puede haber varias fotos si hay varios PhotoSteps en el Workflow.




\subsection{SoundRecordStep: Grabar sonido}
SoundRecordStep  se usa para solicitarle al científico ciudadano que grabe un sonido. Tiene un texto (text) que se muestran a modo de instrucciones o mensaje. Luego de grabado el sonido, el mismo se puede escuchar, y en todo caso volver a grabar otro sonido.

\begin{comment}
\begin{figure}[H]
  \centering
    \includegraphics[scale=0.6]{50-anexos/C-steps/sound_json.png} 
    \caption{SoundRecordStep usando el generador de clases.}
\end{figure}	
\end{comment}


\begin{lstlisting}[language=XML, frame=tlbr, caption=SoundRecordStep usando el generador de clases.]	
{
  "id" : 1,
  "type" : "Sound",
  "text" : "Grabe el sonido de su auto",
  "nextStepId" : 2
}
\end{lstlisting}


\begin{comment}
\begin{figure}[H]
  \centering
    \includegraphics[scale=0.6]{50-anexos/C-steps/sound_java.png} 
    \caption{SoundRecordStep en Java.}
\end{figure}
\end{comment}

\begin{lstlisting}[language=Java, frame=tlbr, caption=SoundRecordStep en Java.]	
Step step = new SoundRecordStep(1,"Grabe el sonido de su auto",2); 
workflow.addStep(step);
\end{lstlisting}


\begin{figure}[H]
  \centering
    \includegraphics[scale=0.3]{50-anexos/C-steps/sound_screen.jpg} 
    \caption{SoundRecordStep en ejecución.}
\end{figure}



\subsubsection{SoundRecordStepResult: El resultado de Grabar sonido}
El SoundRecordStepResult contiene el nombre del archivo del sonido grabado. El sonido va como archivo mp4 en la carpeta de la muestra.
Puede haber varios sonidos si hay varios SoundRecordSteps en el Workflow.



\subsection{InformationStep: Mostrar información}
InformationStep se usa para mostrar una información (texto) al científico ciudadano, por ejemplo, para mostrar un mensaje de bienvenida o para explicar brevemente los pasos que seguirán a continuación para tomar la muestra.

\begin{comment}
\begin{figure}[H]
  \centering
    \includegraphics[scale=0.6]{50-anexos/C-steps/information_json.png} 
    \caption{InformationStep usando el generador de clases.}
\end{figure}	
\end{comment}

\begin{lstlisting}[language=XML, frame=tlbr, caption=InformationStep usando el generador de clases.]	
{
	"id":1,
	"type" : "Information",
	"text" : "Por favor siga las instrucciones para tomar la muestra",
	"nextStepId": 2
}
\end{lstlisting}


\begin{comment}
\begin{figure}[H]
  \centering
    \includegraphics[scale=0.6]{50-anexos/C-steps/information_java.png} 
    \caption{InformationStep en Java.}
\end{figure}
\end{comment}

\begin{lstlisting}[language=Java, frame=tlbr, caption=InformationStep en Java.]	
Step step = new InformationStep(1,"Por favor siga las instrucciones para tomar la muestra",2);
workflow.addStep(step);
\end{lstlisting}


\begin{figure}[H]
  \centering
    \includegraphics[scale=0.3]{50-anexos/C-steps/information_screen.jpg} 
    \caption{InformationStep en ejecución.}
\end{figure}


\subsubsection{InformationStepResult: El resultado de Mostrar información}
El resultado de mostrar información es nulo, es una clase vacía. Solo está para cerrar el circuito.





\subsection{SelectOneStep: Seleccionar una opción de un grupo de opciones}
SelectOneStep  se usa para solicitarle al científico ciudadano que seleccione una sola opción de varias opciones disponibles. Tiene un texto (title) que se muestran a modo de título o instrucciones. Las opciones disponibles para seleccionar son una lista de objetos SelectOneOption.
Cada objeto SelectOneOption tiene un id, textToShow y nextStepId, que se muestran en pantalla en forma de radio buttons.

Este caso particular de Step no tiene un identificador del siguiente Step a ejecutar (nextStepId), sino que cada opción seleccionable posee uno. Esto permite crear bifurcaciones en el Workflow, por ejemplo, se puede preguntar si se observa cierta característica y si la respuesta es si, pasar un PhotoStep para pedir que tome una foto de esa característica, pero si la respuesta es no, se puede saltar el paso de tomar una foto.

\begin{comment}
\begin{figure}[H]
  \centering
    \includegraphics[scale=0.6]{50-anexos/C-steps/select_one_json.png} 
    \caption{SelectOneStep usando el generador de clases.}
\end{figure}	
\end{comment}


\begin{lstlisting}[language=XML, frame=tlbr, caption=SelectOneStep usando el generador de clases.]	
{
	"id": 1,
	"type": "SelectOne",
	"title": "Indique si observa árboles",
	"options": [{
			"id": 1,
			"text": "Muchos árboles",
			"nextStepId": 2
		}, {
			"id": 2,
			"text": "Pocos árboles",
			"nextStepId": 2
		}, {
			"id": 4,
			"text": "No hay árboles",
			"nextStepId": 3
		}]
}
\end{lstlisting}

\begin{comment}
\begin{figure}[H]
  \centering
    \includegraphics[scale=0.6]{50-anexos/C-steps/select_one_java.png} 
    \caption{SelectOneStep en Java.}
\end{figure}
\end{comment}


\begin{lstlisting}[language=Java, frame=tlbr, caption=SelectOneStep en Java.]	
ArrayList<SelectOneOption> optionsToSelect = new ArrayList<>();
optionsToSelect.add(new SelectOneOption(1,"Muchos árboles",2));
optionsToSelect.add(new SelectOneOption(2,"Pocos árboles",2));
optionsToSelect.add(new SelectOneOption(3,"No hay árboles",3));

Step step = new SelectOneStep(1, optionsToSelect, "Indique si observa árboles");
workflow.addStep(step);
\end{lstlisting}


\begin{figure}[H]
  \centering
    \includegraphics[scale=0.3]{50-anexos/C-steps/select_one_screen.jpg} 
    \caption{SelectOneStep en ejecución.}
\end{figure}


\subsubsection{SelectOneStepResult: El resultado de Seleccionar una opción de un grupo de opciones}
SelectOneStepResult contiene la opción seleccionada (un objeto SelectOneOption).





\subsection{MultipleSelectStep: Seleccionar varias opciones de un grupo de opciones}
MultipleSelectStep  se usa para solicitarle al científico ciudadano que seleccione una o más opciones de varias opciones disponibles. Tiene un texto (title) que se muestran a modo de título o instrucciones. Las opciones disponibles para seleccionar son una lista de objetos MultipleSelectOption.
Cada objeto MultipleSelectOption tiene un id, textToShow.
Las opciones disponibles para seleccionar se muestran en pantalla en forma de check-boxes.


\begin{comment}
\begin{figure}[H]
  \centering
    \includegraphics[scale=0.6]{50-anexos/C-steps/multiple_select_json.png} 
    \caption{MultipleSelectStep usando el generador de clases.}
\end{figure}	
\end{comment}

\clearpage

\begin{lstlisting}[language=XML, frame=tlbr, caption=MultipleSelectStep usando el generador de clases.]	
{
	"id": 1,
	"type": "MultipleSelect",
	"title" : "Seleccione lo que observa",
	"options": [{
					"id": 1,
					"text": "Árboles"
				}, {
					"id": 2,
					"text": "Basura"
				}, {
					"id": 3,
					"text": "Agua"
				}, {
					"id": 4,
					"text": "Animales"
				}],
	"nextStepId": 2
}
\end{lstlisting}


\begin{comment}
\begin{figure}[H]
  \centering
    \includegraphics[scale=0.6]{50-anexos/C-steps/multiple_select_java.png} 
    \caption{MultipleSelectStep en Java.}
\end{figure}
\end{comment}

\begin{lstlisting}[language=Java, frame=tlbr, caption=MultipleSelectStep en Java.]	
ArrayList<MultipleSelectOption> optionsToSelect = new ArrayList<MultipleSelectOption>();
optionsToSelect.add(new MultipleSelectOption(1,"Árboles"));
optionsToSelect.add(new MultipleSelectOption(2,"Basura"));
optionsToSelect.add(new MultipleSelectOption(3,"Agua"));
optionsToSelect.add(new MultipleSelectOption(4,"Animales"));

Step step = new MultipleSelectStep(1, optionsToSelect,          "Seleccione lo que observa", 2);
workflow.addStep(step);
\end{lstlisting}



\begin{figure}[H]
  \centering
    \includegraphics[scale=0.3]{50-anexos/C-steps/multiple_select_screen.jpg} 
    \caption{MultipleSelectStep en ejecución.}
\end{figure}


\subsubsection{MultipleSelectStepResult: El resultado de Seleccionar varias opciones de un grupo de opciones}
MultipleSelectStepResult contiene una lista de las opciones seleccionadas (una lista de objetos MultipleSelectOption).





\subsection{LocationStep: Posicionar la muestra en el mapa con el GPS}
LocationStep se usa para solicitarle al científico ciudadano que tome la posición con el GPS del dispositivo móvil. Tiene un texto (text) que se muestran a modo de instrucciones o mensaje. 

Permite usar el GPS o posicionar la muestra manualmente en el mapa.

\begin{comment}
\begin{figure}[H]
  \centering
    \includegraphics[scale=0.6]{50-anexos/C-steps/location_json.png} 
    \caption{LocationStep usando el generador de clases.}
\end{figure}	
\end{comment}

\clearpage

\begin{lstlisting}[language=XML, frame=tlbr, caption=LocationStep usando el generador de clases.]	
{
	"id": 1,
	"type": "Location",
	"text": "Por favor posicione la muestra en el mapa",
	"nextStepId": 2
}
\end{lstlisting}


\begin{comment}
\begin{figure}[H]
  \centering
    \includegraphics[scale=0.6]{50-anexos/C-steps/location_java.png} 
    \caption{LocationStep en Java.}
\end{figure}
\end{comment}


\begin{lstlisting}[language=Java, frame=tlbr, caption=LocationStep en Java.]	
Step step = new LocationStep(1,"Por favor posiciones la muestra en el mapa",2);
workflow.addStep(step);
\end{lstlisting}



\begin{figure}[H]
  \centering
    \includegraphics[scale=0.3]{50-anexos/C-steps/location_screen.jpg} 
    \caption{LocationStep en ejecución.}
\end{figure}

\subsubsection{LocationStepResult: El resultado de Posicionar la muestra en el mapa con el GPS}
LocationStepResult contiene la latitud y longitud (ambos de tipo double) de la posición GPS tomada.





\subsection{RouteStep: Grabar un recorrido en el mapa usando el GPS}
LocationStep se usa para solicitarle al científico ciudadano que grabe un recorrido con el GPS del dispositivo móvil. Tiene las siguientes propiedades:

\begin{itemize}
\item \textbf{text:} un texto que se muestran a modo de instrucciones o mensaje.
\item \textbf{interval:} [Opcional] el intervalo en milisegundos en el que se pedirán actualizaciones del GPS. Si no se especifica el valor por defecto es 5000.
\item \textbf{mapZoom:} [Opcional] el zoom con el que se muestra el mapa. Si no se especifica el valor por defecto es 15.
\end{itemize}

\begin{comment}
\begin{figure}[H]
  \centering
    \includegraphics[scale=0.6]{50-anexos/C-steps/route_json.png} 
    \caption{RouteStep usando el generador de clases.}
\end{figure}	
\end{comment}


\begin{lstlisting}[language=XML, frame=tlbr, caption=RouteStep usando el generador de clases.]	
{
	"id": 1,
	"type": "Route",
	"text": "Inicie el recorrido cuando este listo",
	"interval": 30000,
	"mapZoom": 20,
	"nextStepId": 2
}
\end{lstlisting}


\begin{comment}
\begin{figure}[H]
  \centering
    \includegraphics[scale=0.6]{50-anexos/C-steps/route_java.png} 
    \caption{RouteStep en Java.}
\end{figure}
\end{comment}


\begin{lstlisting}[language=Java, frame=tlbr, caption=RouteStep en Java.]	
RouteStep step = new RouteStep(1,"Inicie el recorrido cuendo esté listo",2);
step.setInterval(30000);
step.setMapZoom(20);
workflow.addStep(step);
\end{lstlisting}



\begin{figure}[H]
  \centering
    \includegraphics[scale=0.3]{50-anexos/C-steps/route_screen.jpg} 
    \caption{RouteStep en ejecución.}
\end{figure}


\subsubsection{RouteStepResult: El resultado de Grabar un recorrido en el mapa usando el GPS}
RouteStepResult tiene una lista de objetos de la clase Location proporcionada por Android.  Para
mas información sobre esta clase ver la documentación oficial de Android sobre Location \footnote{https://developer.android.com/reference/android/location/Location}.






\subsection{InsertTextStep: Ingresar texto}
InsertTextStep se usa para solicitarle al científico ciudadano que ingrese una anotación, que puede ser texto o números. Tiene las siguientes propiedades:

\begin{itemize}
\item \textbf{text:} un texto que se muestran a modo de instrucciones o mensaje.
\item \textbf{sampleText:} un texto, a modo de muestra.
\item \textbf{maxLength:} establece la cantidad máxima de caracteres permitidos.
\item \textbf{inputType:} indica el tipo de caracteres admitidos. Los valores pueden ser \textbf{text} para el ingreso de texto, \textbf{number} para el ingreso de número enteros, \textbf{decimal} para el ingreso de números con decimales.
\item \textbf{optional:} un booleano que indica si se puede dejar vacío (true) o si es obligatorio que ingrese una anotación (false).
\end{itemize}

\begin{comment}
\begin{figure}[H]
  \centering
    \includegraphics[scale=0.6]{50-anexos/C-steps/insert_text_json.png} 
    \caption{InsertTextStep usando el generador de clases.}
\end{figure}	
\end{comment}


\begin{lstlisting}[language=XML, frame=tlbr, caption=InsertTextStep usando el generador de clases.]	
{
	"id": 1,
	"type": "InsertText",
	"text": "Por favor, ingrese el nombre del lago",
	"sampleText" : "Nombre del lago",
	"inputType" : "text",
	"maxLength" : 50,
	"optional" : true,
	"nextStepId": 2
}
\end{lstlisting}


\begin{comment}
\begin{figure}[H]
  \centering
    \includegraphics[scale=0.6]{50-anexos/C-steps/insert_text_java.png} 
    \caption{InsertTextStep en Java.}
\end{figure}
\end{comment}



\begin{lstlisting}[language=Java, frame=tlbr, caption=InsertTextStep en Java.]	
Step step = new InsertTextStep(1,"Por favor, ingrese el nombre del lago","Nombre del lago",50, InsertTextStep.InputType.TYPE_TEXT, true,2);
workflow.addStep(step);
\end{lstlisting}



\begin{figure}[H]
  \centering
    \includegraphics[scale=0.3]{50-anexos/C-steps/insert_text_screen.jpg} 
    \caption{InsertTextStep en ejecución.}
\end{figure}


\subsubsection{InsertTextStepResult: El resultado de Ingresar texto}
InsertTextStepResult contiene el String con la anotación ingresada.





\subsection{InsertDateStep e InsertTimeStep: Ingresar fecha y hora}
InsertDateStep e InsertTimeStep se usan para solicitarle al científico ciudadano que ingrese una fecha y una hora respectivamente. Tienen un texto que se muestran a modo de instrucciones o mensaje.


\begin{comment}
\begin{figure}[H]
  \centering
    \includegraphics[scale=0.6]{50-anexos/C-steps/insert_date_time_json.png} 
    \caption{InsertDateStep e InsertTimeStep usando el generador de clases.}
\end{figure}	
\end{comment}

\clearpage

\begin{lstlisting}[language=XML, frame=tlbr, caption=InsertDateStep e InsertTimeStep usando el generador de clases.]	
{
{
	"id": 1,
	"type": "InsertDate",
	"text": "Ingrese la fecha que encontró el mosquito",
	"nextStepId": 2
},
{
	"id": 2,
	"type": "InsertTime",
	"text": "Ingrese la hora que encontró el mosquito",
	"nextStepId": 3
}	

}
\end{lstlisting}


\begin{comment}
\begin{figure}[H]
  \centering
    \includegraphics[scale=0.6]{50-anexos/C-steps/insert_date_time_java.png} 
    \caption{InsertDateStep e InsertTimeStep en Java.}
\end{figure}
\end{comment}


\begin{lstlisting}[language=Java, frame=tlbr, caption=InsertDateStep e InsertTimeStep en Java.]	
Step step = new InsertDateStep(1,"Ingrese la fecha que encontró el mosquito",2);
workflow.addStep(step);

step = new InsertTimeStep(2,"Ingrese la hora que encontró el mosquito",3);
workflow.addStep(step);

\end{lstlisting}


\begin{figure}[H]
  \centering
    \includegraphics[scale=0.3]{50-anexos/C-steps/insert_date_screen.jpg} 
    \includegraphics[scale=0.3]{50-anexos/C-steps/insert_time_screen.jpg} 
    \caption{InsertDateStep (izquierda) e InsertTimeStep (derecha) en ejecución.}
\end{figure}


\subsubsection{InsertDateStepResult e InsertTimeStep: El resultado de Ingresar fecha y hora}
InsertDateStepResult e InsertTimeStep tienen un objeto Date de Java que tiene la fecha o la hora según corresponda.



\subsection{Definir un nuevo Step, StepFragment y StepResult} \label{sec:definir_steps}
Como se explicó en la sección \ref{sec:clases_core} un Step tiene los parámetros necesarios para que se ejecute un StepFragment y genere un  StepResult con la interacción del usuario.

Para definir un nuevo Step personalizado, se debe definir también un StepFragment personalizado que sepa ejecutarlo y StepResult personalizado que contendrá el resultado de esa ejecución. A continuación se explica cómo definir cada uno de ellos.

\subsubsection{Definir un nuevo Step}
Para definir un nuevo Step personalizado se debe implementar la interface Step, la cual extiende Serializable y define los siguientes métodos que se deben implementar:

\begin{itemize} 
  \item \textbf{\textit{String getStepResourceName()}:} debe devolver el nombre o etiqueta del Step, por ejemplo «Tomar foto».
  
  \item \textbf{\textit{int getId()}:} debe devolver el id del Step.
  
  
  \item \textbf{\textit{ \textless T extends StepFragment\textgreater }} 
  
   \textbf{\textit{Class\textless T\textgreater   getStepFragmentClass()}:}  debe devolver la clase del StepFragment que lo ejecutará.
    
    
  \item \textbf{\textit{ \textless T extends StepFragment\textgreater }} 
  
   \textbf{\textit{void setStepFragmentClass(Class\textless T\textgreater type)}:} se usa para establecer la clase del StepFragment que lo ejecutará.
  
  
  \item \textbf{\textit{@Nullable Integer getNextStepId(StepResult stepResult)}:}  debe devolver el id del siguiente Step a ejecutar en el Workflow. Puede ser null y en ese caso se considerará el final del Workflow.
    
    
  \item \textbf{\textit{@Nullable Integer getHelpResourseId()}:}   debe devolver el id del recurso que se usará para mostrar la ayuda del Step. Puede ser null y en ese caso se considerará que el Step no tiene ayuda para mostrar.
  
  \item \textbf{\textit{void setHelpResourseId(Integer helpResourseId)}:}   se usa para establecer el recurso que se usará para mostrar la ayuda del Step.

\end{itemize}


\subsubsection{Definir un nuevo StepFragment}
Para definir un nuevo StepFragment personalizado se debe extender la clase StepFragment, la cual es una clase abstracta que extiende de Fragment y define los siguientes métodos que se deben implementar:

\begin{itemize} 
  \item \textbf{\textit{int getLayoutResource()}:} debe devolver el «Layout Resource» que contiene diseño con la estructura de la interfaz de usuario. Es el que se usará en el método \textit{onCreateView} para cargar dicha estructura de interfaz de usuario.
  
  
  \item \textbf{\textit{void onCreateViewStepFragment(View rootView, Bundle savedInstanceState)}:} este método se llama internamente en el método \textit{onCreateView} luego de cargar la estructura de la interfaz de usuario y debe usarse para inicializar  y guardar referencia a los elementos de la IU. El parámetro \textit{rootView} contiene la estructura de la interfaz de usuario cargada y el parámetro \textit{savedInstanceState} contiene el estado de instancia guardado que se recibe en el método \textit{onCreateView}.
    

  
  \item \textbf{\textit{boolean validate()}:} este método se llama internamente cuando el usuario aprieta el botón «Siguiente» y se debe usar para hacer las validaciones pertinentes antes de finalizar el StepFragment, como por ejemplo validar que el usuario haya seleccionado una opción de las dadas. Debe devolver \textbf{true} si la validación es exitosa o \textbf{false} en caso contrario.  
  
  \item \textbf{\textit{StepResult getStepResult()}:} debe devolver el StepResult generado.


\end{itemize}

\subsubsection{Definir un nuevo StepResult}
Para definir un nuevo StepResult personalizado se debe implementar la interface StepResult, la cual extiende Serializable y define los siguientes métodos que se deben implementar:

\begin{itemize} 
  \item \textbf{\textit{Step getStep()}:} debe devolver el Step que generó el StepFragment.
  
  
  \item \textbf{\textit{boolean hasMultimediaFile()}:} debe devolver \textbf{true} si el StepResult tiene un archivo multimedia (foto, video, audio, etc.) asociado o \textbf{false} en caso contrario.
    
    
  \item \textbf{\textit{String getMultimediaFileName()}:} debe devolver el nombre del archivo multimedia asociado al StepResult.

\end{itemize}


