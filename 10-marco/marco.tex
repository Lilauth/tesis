\chapter{Marco Teórico}
\section{Ciencia  Abierta y Ciencia Ciudadana}

Escribir una introducción de ciencia ciudadana y ciencia abierta.

\subsection{Ciencia Abierta}
Ciencia Abierta es un término que engloba otros que tienen que ver con la creación y diseminación del conocimiento en el futuro. Tradicionalmente el sistema de creación y diseminación del conocimiento está basado en la publicación en revistas científicas. Es por ello que antes de imprimir y difundir conocimiento el mismo debe estar completo y correcto. Esto se daba por una cuestión de costos. Publicar en revistas científicas tiene un costo, y si el artículo no está completo y correcto no vale la pena publicarlo. Sin embargo el artículo impreso podría no ser el formato elegido, habiendo otros formatos menos costosos y que permitrían la publicación de resultados parciales e incluso la corrección o los comentarios por parte de pares o de personas que están trabajando en el mismo tema. Internet tiene los medios para que esto sea posible a través de Wikis y blogs. De esta forma no haría falta esperar que una investigación esté completa para acceder a ella. \cite{bartling2014opening}
Ciencia Abierta es un término que quiere enmarcar 5 lineamientos:
\begin{itemize}
\item {Educación Pública}
\begin{description}
De la mano de abrir las investigaciones científicas a una audiencia mayor se encuentra la responsabilidad de hacerlas accesibles. Se busca hacer accesible el proceso de investigación (la producción) y comprensible el resultado (el producto).
\end{description}

\item {Educación Pública}
\begin{description}
De la mano de abrir las investigaciones científicas a una audiencia mayor se encuentra la responsabilidad de hacerlas accesibles. Se busca hacer accesible el proceso de investigación (la producción) y comprensible el resultado (el producto).
\end{description}

\end{itemize}



\subsection{Cómo participan los ciudadanos?}

\begin{itemize}
	\item Ciencia Ciudadana
	\begin{itemize}
		\item Cómo participan los ciudadanos en la ciencia?        
		Esto de que hay que asignarles tareas acordes o darles una pequeña capacitación o ayuda en pantalla. Comunidades que hacen de soporte de voluntarios.
		\item De qué depende que un proyecto incluya ciencia ciudadana?
		Tipología de los proyectos de ciencia ciudadana, por ejemplo, que sea de educación, de investigación, que no cualquier proyecto puede tilizar ciencia ciudadana y no siempre se aplica en todo el proyecto. Muchas veces los ciudadanos colaboran con una parte.
	\end{itemize}   
	\item Ciencia Abierta   
	\begin{itemize}
		\item Por qué es importante la ciencia abierta? democracia y cuestiones políticas. Acceso público a la información de interés general.
		\item Qué relación tiene con la ciencia ciudadana? Básicamente los participantes en proyectos de investigación de ciencia ciudadana lo hacen por interés en el tema de investigación. Es una buena práctica que una vez finalizada la investigación se haga una devolución de los resultados de la misma para que los ciudadanos participantes quienes estaban interesados en el tema de movida puedan ver los resultados de la investigación. Este tema está directamente relacionado con la ciencia abierta que básicamente es abrir los datos, resultados y procesos utilizados para conseguir resultados a el público general.
	\end{itemize}
\end{itemize}

\section{ Dispositivos Móviles y Android }
\begin{itemize}
	\item Distribución de dispositivos móviles entre la población
	cantidad de personas que tienen dispositivos móviles. Que porcentaje de la población representan. Zonas de concentración de dispositivos:cómo están distribuidos
	\item Características de los dispositivos móviles
	cámaras, micrófonos, conexiones a redes, posibilidad de transferencias de archivos, navegabilidad en la interfaz de aplicación.
	\item Android
	el sistema operativo. Licencia. Estructura. Versiones y lo que ello implica.
	\item Ejemplos de aplicaciones de ciencia ciudadana y dispositivos móviles
	hablemos del ejemplo africano que no tenía palabras para que la población partipe sin necesidad de saber leer o escribir. AppEAR y Cazamosquitos. Ejemplo aplicado a salud Colombia
\end{itemize}

\section{ Frameworks }

\begin{itemize}
	\item Frameworks para construir aplicaciones
	\item Configuración de aplicaciones mediante archivos
\end{itemize}

%Hacer una tesis implica encontrar una pregunta que valga la pena responder o un problema que valga la pena resolver y darle respuesta o solución. Es una tarea de investigación que tiene como aspecto muy importante conocer lo que ya existe alrededor de la pregunta o problema que se elige. 

%Al llegar a este capitulo, el lector tiene una idea de cual es el problema. Seguramente se imagina problemas similares o soluciones al problema. El objetivo, en este momento, es convencerlo de que conocemos el problema y otros similares; que conocemos las formas en las que se lo ha intentado resolver (o a problemas similares); y que aún después de saber todo eso sigue siendo un problema importante, difícil y que nadie resolvió 8o nadie revolvió tan bien como nosotros).

%Para escribir este capitulo hay que leer. Hay que buscar soluciones a problemas similares y compararlas con lo que nosotros queremos hacer. Si sabemos que la nuestra es mejor, ya podemos marcar cuales son los puntos débiles de las existentes. También se puede escribir un poco sobre otras investigaciones, que si bien no atacaron problemas parecidos, pueden ser aplicadas a resolver parte de este. 

%Este capitulo es bueno ir escribiendolo en borrador cada vez que se lee algo (un articulo por ejemplo). Por lo menos hay que escribir un resumen de un párrafo de lo leído (registrando la referencia en el archivo bibliografia.bib y citando dede acá), y dar nuestra opinión al respecto en términos de su relación con el problema de nuestra tesis.