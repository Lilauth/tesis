\chapter{Marco Teórico}
		
	
\section{Introducción a la Ciencia Ciudadana}
    
    Ciencia Ciudadana es un termino que engloba las diferentes maneras en las que los ciudadanos participan en ciencia. Estas actividades pueden ser enviar información desde aplicaciones instaladas en sus dispositivos móviles acerca de especies invasivas, comportamiento de las aves, la cantidad de mariposas presentes al comienzo de la primavera o cualquier otro tema generalmente relacionado a la ecología y la conservación; como así también participar en debates locales de políticas que afectan directamente el ecosistema de una determinada ciudad o zona, como puede ser el fracking, la minería y los pesticidas en la agricultura y de esta manera influir en las políticas locales respecto de cómo regular esas actividades.\cite{envCitizenScience}
    
	Algunos de estos proyectos cuantan con una larga historia de investigación (como el CBC, Christmas Bird Count), otros con gran cantidad de participantes, quienes interactúan a través de foros y aplicaciones web desde la comodidad de sus hogares (como sucede en proyectos como Zooniverse) o bien abarcando grandes extensiones geográficas (como puede ser en The Big Butterfly Count). \cite{shirk2012public} 
		
	Aunque la ciencia ciudadana no es un concepto nuevo, su notoriedad es relativamente reciente. Los científicos ciudadanos o voluntarios ahora participan de proyectos relacionados con el cambio climático, las especies invasivas, la conservación de ecosistemas biológicos, el monitoreo de la calidad del agua y varios tópicos más. Esta popularidad se vio impulsada principalmente por tres factores:

\begin{itemize}
	\item {Disponibilidad Tecnológica}
	La disponibilidad de herramientas tecnológicas que permitan distribuir información acerca de los proyectos y también permitan recolectar la información generada por los voluntarios. De estas herramientas internet es la más representativa, pero la tecnología móvil está jugando un rol fundamental con la popularización de smartphones. \cite{silvertown2009new}
	\item {Reconocimiento del Aporte de los Voluntarios por parte de los Profesionales}
	La participación de voluntarios en un proyecto de investigación aporta diferentes atributos que pueden ser trabajo, poder de cómputo o habilidad \cite{cohn2008citizen}
	\item {Inclusión del Público General en Proyectos Científicos}
	La mejor manera para que el común de la gente entienda y se involucre en proyectos científicos es siendo parte de ellos. \cite{silvertown2009new}
\end{itemize} 

	En ciencias como pueden ser la arqueología, la astronomía o las ciencias naturales la capacidad de observación es a veces es más importante que el equipamiento costoso. Los voluntarios o científicos ciudadanos realizan actividades como parte de un proyecto científico que está especialmente diseñado o bien fue adaptado para que cumplan un rol, ya sea para fines educativos de los mismos voluntarios o para beneficio del proyecto. En los mejores ejemplos, se benefician ambos: los voluntarios y el proyecto.\cite{silvertown2009new}
	Para ello vamos a utilizar la clasificación provista por Wiggins and Crowston para tener un marco de lo que son los proyectos de recolección según esta clasificación, y por qué son los que mejor se relacionan con los proyectos de Ciencia Ciudadana en donde los colaboradores actúan como investigadores de campo recolectando muestras para introducirel tema de la toma y el envío de muestras recolectadas mediante aplicaciones móviles.	

\section{Clasificación de los Proyectos de Ciencia Ciudadana}	

\begin{itemize}
	\item {Acción}
		
		Los proyectos clasificados en esta categoría no son planificados o iniciados por científicos, sino más bien por los ciudadanos, y generalmente requieren un compromiso a lo largo del tiempo en los problemas ambientales locales por lo cual las actividades científicas están orientadas al ambiente físico. 
		
		Estos proyectos solicitan la colaboración de científicos como asesores o consultores, y no como organizadores. Los datos o resultados obtenidos no persiguen un fin académico, sino más bien buscan fundamentar con evidencia para poder tomar acciones sobre alguna situación. 
	\item {Conservación} 
	
	Al igual que los proyectos de Acción, los proyectos de Conservación son fuertemente regionales, y las actividades de los voluntarios están enfocadas mayormente en la recolección de información. La mayoría de los proyectos de investigación tienen contenido o fines educativos. También tienden a ser de alcance regional, como lo reflejan sus desafíos y metas.
	
	Estos proyectos buscan generar información principalmente como fuente para la toma de decisiones relacionadas al manejo de recursos, y también buscan la promoción de la administración y reconocimiento del voluntariado. También prestan especial atención a la generación de información científicamente válida. Son esfuerzos de monitoreo a largo plazo y generalmente no tienen problemas de sustentabilidad, ya que son subvencionadas por fondos públicos o reciben ingresos de agencias que son las que nuclean los proyectos o son las interesadas en sus resultados.
	
	\item {Investigación o Recolección} 
	
	Los proyectos de investigación concentran su atención en investigaciones científicas cuyos objetivos requieren la recolección de información del medio físico. Este tipo de proyectos es el que mejor encaja en la definición de Ciencia Ciudadana. Y aunque los objetivos de educación no son los principales de estos proyectos, forman parte de ellos como material de capacitación o incluyendo estructuras de recolección que alientan el aprendizaje al aire libre. El alcance varía de regional a internacional, y puede lograr participación masiva de hasta decenas de miles de voluntarios y obtener millones de observaciones (de voluntarios) anuales. La mayoría de estos proyectos están enfocados en la investigación biológica, medioambiental o meteorológica, por dar algunos ejemplos. 
	
	Una de las principales preocupaciones de este tipo de proyectos es generar resultados científicamente válidos, ya que son concebidos para generar conocimiento formal y son mayormente organizados por científicos. El cuidadoso diseño del proyecto y de las tareas son los que permiten lograr resultados válidos. Aparte utilizan toda una serie de metodologías que permiten la validación de la información generada. Los voluntarios están dispersos geográficamente y esto es un recurso valioso ya que este tipo de proyectos intenta muchas veces registrar la distribución geográfica de determinadas especies o la ocurrencia de fenómenos naturales. 
	
	\item {Virtual} 
	
	En los proyectos de ciencia ciudadana denominados Virtuales todas las actividades son mediante tecnologías de la información y la comunicación, sin la intervención de elementos del entorno físico.
	
	Los proyectos que cumplen las condiciones para ser clasificados como virtuales provienen de la astronomía, la paleontología y la proteómica, que es una rama de la microbiología que estudia la estructura de las proteínas. Algunos proyectos de psicología podrían clasificar, pero no lo hacen porque los voluntarios colaboran como sujetos de pruebas, y esto no es formalmente considerado como colaboración en la investigación. 
	
	Galaxy Zoo es un ejemplo de proyecto virtual de ciencia ciudadana. Desde hace más de diez años los voluntarios que colaboran con el proyecto clasifican galaxias en fotografías. Responden una serie de preguntas respecto de la foto que están observando, y de esta manera los científicos encargados del proyecto obtienen una primera clasificación, que se construye en base a las observaciones de varios voluntarios de manera independiente. \cite{GalaxyZoo} 
	
	Al igual que en los proyectos de recolección anteriormente mencionados, los proyectos virtuales encuentran dificultades a la hora de conseguir resultados válidos en términos científicos. Estos resultados se obtienen mediante el desarrollo cuidadoso de las actividades. Como la participación es mayormente virtual, poder mantener a los voluntarios comprometidos con el proyecto es un desafío. Por eso muchas veces estos proyectos incluyen técnicas de gamificación, de competencia amigable entre participantes o de valoración de la contribución hecha por el voluntario (feedback).
	 
	\item {Educación} 
	
	Los proyectos de ciencia ciudadana pertenecientes a esta categoría son aquellos cuyo principal objetivo es educar. Los participantes de estos proyectos tienen como objetivo educar, y aportan recursos educativos informales; mientras los proyectos ofrecen material educativo formal. También, las actividades están pensadas para que el participante vaya acumulando conocimientos.
	
	Un ejemplo de este tipo de proyectos es Fossil Finders, que centra la investigación en el análisis de fósiles del Devónico (período de la era Paleozóica) proveyendo de materiales de estudio a estudiantes y profesores de escuelas secundarias. Los estudiantes van a identificar y medir fósiles en muestras de rocas enviadas a sus aulas. Luego ingresarán los datos obtenidos en una base de datos online, y podrán comparar sus datos con los de otras escuelas participantes. Con ello van a tener la oportunidad de involucrarse con métodos de investigación reales y de asistir a los investigadores del Instituto de Investigación Paleontológica a reconstruir el pasado geológico de Nueva York. \cite{FossilFinders}

	La mayoría de estos proyectos proyectos persiguen un fin educativo, el aprendizaje y el desarrollo de habilidades científicas. Por ello incluyen actividades de análisis de datos o muestras, brindando la posibilidad de desarrollar pensamiento crítico. 
	\end{itemize} 
	
	Los proyectos de Ciencia Ciudadana buscan resultados que generalmente caen en tres grandes categorías: resultados que sirven a la investigación; resultados que le sirven a los participantes como pueden ser adquirir nuevas habilidades o conocimientos y/o resultados que tienen que ver con sistemas socio-ecológicos, como es influenciar políticas, construir bases para la toma de decisiones en una comunidad o participar de acciones para la conservación del medio ambiente. \cite{shirk2012public}

\section{Proyectos de Recolección}	 
	En los proyectos de recolección, según la clasificación antes descripta, es en donde Samplers  brindaría su aporte, ya que está pensado para crear aplicaciones Android que sirvan para recolectar muestras en proyectos de Ciencia Ciudadana.
	
	Hay tres puntos en los que se debe enfatizar para que los voluntarios pueden recolectar y enviar información confiable: proveer información clara acerca de los protocolos de recolección, proveer formularios para el ingreso de datos que sean lo más lógicos y simples posibles y por último brindar soporte para que los participantes entiendan cómo seguir los protocolos y cómo enviar la información.\cite{bonney2009citizen}

\begin{itemize}
	\item {Protocolos}	
			Los datos que se obtienen en proyectos de Ciencia Ciudadana son recolectados mediante protocolos que especifican dónde, cuándo y cómo esos datos deben ser recolectados. Los protocolos deben definir un diseño formal o un plan de acción que permita combinar las muestras que fueron tomadas por múltiples participantes en diferentes ubicaciones para su posterior análisis. Los protocolos utilizados en proyectos de ciencia ciudadana deben ser fáciles de ejecutar, deben poder ser explicados de manera simple y directa, y deben ser desafiantes para los voluntarios.\cite{bonney2009citizen}
		Como se explica más adelante, la implementación del protocolo de recolección de muestras es el workflow de Samplers.
		
	\item {Formularios de ingreso de Datos}	
			En conjunto con un protocolo de recolección bien diseñado están los formularios de ingreso de datos. Los formularios en los que los usuarios ingresan sus observaciones deben ser fáciles de entender y completar. En cada paso de su wokflow, Samplers provee formularios de ingreso de datos que pueden ser preguntas de respuesta simple o compuesta (como componentes radio o checkbox), permite sacar fotografías o informar una ubicación entre otras funcionalidad provistas. 
			Es recomendable definir límites en el rango de datos que los formularios recogen para simplificar su posterior análisis. Por ejemplo, si la respuesta esperada a la pregunta 'Cuántos árboles cuenta en una cuadra' debería ser un número entre cero y 15 una respuesta como 150 podría indicar un error de tipeo. Entonces, es aconsejable pedirle al usuario que reporta que chequee si la información ingresada es correcta. De esta manera, los usuarios pueden revisar sus respuestas cvuando están fuera de los rangos establecidos. Aun así, suponiendo que el usuario indique un respuesta de esas características, es decir, una respuesta que se salga de los límites esperados; ese formulario debería guardarse con alguna marca que permita identificarlo para que los responsables del proyecto puedan analizarlo con más detalle y ver si se debe a un cambio en el entorno que está siendo observado o si es realmente un error de ingreso de datos del usuario.
			
	\item {Material Educativo}
		Los participantes deben ser provistos de material educativo para entender y seguir de manera satisfactoria los protocolos del proyecto. El material educativo puede incluir guías de identificación, posters, manuales, videos, podcasts, listas de correo y FAQ para que los participantes puedan consultar e incluso participar en foros y discusiones acerca del relevamiento que están haciendo o de cómo se espera que completen los formularios provistos. 
		Samplers ofrece ayuda en cada formulario o ventana de ingreso de datos que puede ser configurada para brindar información acerca de cómo se espera que el voluntario la complete.
		
\end{itemize} 

	
\section{Ciencia Ciudadana y Dispositivos Móviles}

	En el reporte Mercado Celular Argentino 2019 elaborado por Carrier y Asociados se observa que un 6\% de los smartphones que hay en funcionamiento en este momento es algún modelo de iPhone, el 93\% pertenece a las marcas que utilizan Android y un 1\% de otras marcas que utilizan otros sistemas operativos (como Windows Phone, Symbian, BlackBerry, etc). Utilizando números absolutos, de los 34 millones de celulares en uso que hay en el país, 2.100.000 son iPhone y 31.620.000 tienen sistema operativo Android. \cite{carrier}		

	
	Las nuevas tecnologías sirven de soporte para que los proyectos de ciencia ciudadana logren ampliar su alcance a múltiples lugares, expandir su permanencia en el tiempo y llegar a diferentes escalas sociales. Es mediante la utilización de tecnologías móviles, como pueden ser smartphones y tablets, que los proyectos tienen la posibilidad de involucrar audiencias más amplias, de motivar voluntarios, de mejorar la recolección de información y controlar su calidad.\cite{newman2012future}
	
	

	Los dispositivos móviles tienen integrados servicios de captura de imagen (cámaras) y de posicionamiento (GPS) que mejoran la frecuencia y la calidad de la información relevada.\cite{newman2012future}
	
	La proliferación de tecnologías móviles está enriqueciendo los ambientes urbanos en lo relacionado a sensing, proveyendo herramientas para recolectar datos y creando oportunidades para que cualquier persona interesada pueda involucrarse en actividades científicas. En resumen, los dispositivos móviles son ideales como soporte para permitir la recolección espontánea de información. Ahora bien, detrás de su facilidad de uso y de su masividad hay una complejidad técnica y de infraestructura a la hora de desarrollar aplicaciones, que pueden significar una inversión de tiempo, dinero o ambas y pueden limitar el acceso de pequeñas organizaciones o proyectos a este tipo de desarrollos. \cite{kim2013sensr}
	
	
	
	


