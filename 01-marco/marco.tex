\chapter{Marco Teórico}
\section{Ciencia  Abierta y Ciencia Ciudadana}

No es fácil definir el término Ciencia Abierta ya que abarca una multiplicidad de participantes, cada uno con su enfoque particular. La Ciencia Abierta afecta a investigadores,políticos, desarrolladores y operadores de plataformas, editoriales y público interesado. 

\subsection{Ciencia Abierta}
Ciencia Abierta es un término que engloba otros que tienen que ver con la creación y difusión del conocimiento en el futuro. Tradicionalmente el sistema de creación y difusión del conocimiento está basado en la publicación en revistas científicas. Es por ello que antes de imprimir y difundir un artículo el mismo debe estar completo y correcto, por una cuestión de costos. Publicar en revistas científicas tiene un costo, y si el artículo no está completo y correcto no vale la pena publicarlo. Sin embargo el artículo impreso podría no ser el único formato que esté disponible, habiendo otros menos costosos y que permitirían la publicación de resultados parciales e incluso la corrección o los comentarios por parte de pares o de personas que están trabajando en el mismo tema. Internet tiene los medios para que esto sea posible a través de Wikis y blogs. De esta forma no haría falta esperar que una investigación esté completa para acceder a ella.
Volviendo a la terminología, desde el punto de vista del libro 'Opening Science' \cite{bartling2014opening} se proponen 5 corrientes de pensamiento:
\begin{itemize}
	\item {Escuela Pública}
	
	La Escuela Pública está comprometida con hacer que la ciencia sea accesible al público; que esté disponible para una audiencia mayor de interesados. Con las herramientas sociales de la web los investigadores pueden hacer público el proceso de investigación y preparar los resultados para interesados no necesariamentes expertos en el área. Hay dos corrientes: una comprometida con la accesibilidad en el proceso de investigación (la producción) y otra con la comprensibilidad de los resultados (el producto).
	
	Al abrir el proceso de investigación, se logra que el público interesado pueda participar como pasó con Foldit, qué fue la interfaz que plantearon los mismos colaboradores que en un primer momento aportaban cediendo fuerza de cómputo mientras sus computadoras no estaban siendo utilizadas. Propusieron una mejora, que era una interfaz en forma de juego que se llamó Foldit. Mientras jugaban resolvían el mapeo de proteínas. En este caso, los colaboradores hicieron su aporte más como una fuerza de trabajo que como pares de una investigación. La idea es poder incluirlos de maneras más significativas. Todavía falta investigar de qué manera se puede involucrar al público en el proceso de investigación.
	
	Por otro lado, los resultados de las investigaciones deberían estar en un formato accesible, no sólo en cuanto a lenguaje utilizado, sino también a los medios de difusión. Los blogs o las herramientas sociales como puede ser Twitter son medios que podrían utilizarse para difundir resultados. El beneficio de que más gente comprenda por qué una investigación puede resultar beneficiosa puede lograr la presión necesaria para obtener fondos de financiamiento. Cuando el lenguaje utilizado en la redacción es complejo o la difusión es escasa, menos oportunidades hay de que el público se entere de qué cosas se están investigando o por qué es importante financiar tal o cual investigación.

	\item {Escuela Democrática}
	\begin{itemize}
	
		La Escuela democrática nos habla de accesibilidad no en términos de participación en la investigación o de comprensión de los resultados, sino más bien en acceder a los resultados y la información que generan, pero también a los papers e investigaciones que se usaron como fuentes, multimedia y representaciones gráficas. Todos deberian poder acceder a los resultados de una investigación, sobre todo si está financiada con recursos del estado.
		\item{Datos Abiertos (Open Data)}
		En el caso de los datos abiertos, el concepto es parecido al del software libre: el derecho a usar información científica debe quedar en la comunidad científica y no ser propiedad de la editorial que lo publica. Las editoriales no deberían tener derechos sobre las publicaciones más que validarlas, en cambio los investigadores si. También, deberían poner la información en formatos accesibles; por ejemplo una tabla en un PDF no es una tabla que pueda revisarse en detalle o modificarse. A la hora de abrir los datos de las investigaciones se presentan dificultades que pueden ser culturales o tecnológicas. Entre las dificultades tecnológicas, muchas veces el sistema de repositorios para soporte de datos es muy complejo y ello provoca resistencia a la hora de compartir información. Desde el punto de vista cultural uno de los problemas es la resistencia a compartir datos que fueron difíciles de reunir. También, muchas veces los datos se comparten de manera más fluída cuando se sabe que va a haber una contraprestación; es decir, se comparte más fácilmente con investigadores que de alguna manera van a aportar o pueden ser un recurso en un determinado momento.
			
		\item{Acceso Abierto}
		
		
	\end{itemize}
	\item {Escuela Pragmática}
	Los partidarios de la escuela pragmática se enfocan en lograr que el proceso de investigación sea más eficiente. Considera a la investigación científica como un proceso que puede ser optimizado:
	\begin{itemize}
		\item modularizando
		\item abriendo la "cadena de producción"
		\item incluyendo herramientas externas
		\item permitiendo la colaboración a través de internet
	\end{itemize}	
\end{itemize}	
	
\subsection{Cómo participan los ciudadanos?}

\begin{itemize}
	\item Ciencia Ciudadana
	\begin{itemize}
		\item Cómo participan los ciudadanos en la ciencia?        
		Esto de que hay que asignarles tareas acordes o darles una pequeña capacitación o ayuda en pantalla. Comunidades que hacen de soporte de voluntarios.
		\item De qué depende que un proyecto incluya ciencia ciudadana?
		Tipología de los proyectos de ciencia ciudadana, por ejemplo, que sea de educación, de investigación, que no cualquier proyecto puede tilizar ciencia ciudadana y no siempre se aplica en todo el proyecto. Muchas veces los ciudadanos colaboran con una parte.
	\end{itemize}   
	\item Ciencia Abierta   
	\begin{itemize}
		\item Por qué es importante la ciencia abierta? democracia y cuestiones políticas. Acceso público a la información de interés general.
		\item Qué relación tiene con la ciencia ciudadana? Básicamente los participantes en proyectos de investigación de ciencia ciudadana lo hacen por interés en el tema de investigación. Es una buena práctica que una vez finalizada la investigación se haga una devolución de los resultados de la misma para que los ciudadanos participantes quienes estaban interesados en el tema de movida puedan ver los resultados de la investigación. Este tema está directamente relacionado con la ciencia abierta que básicamente es abrir los datos, resultados y procesos utilizados para conseguir resultados a el público general.
	\end{itemize}
\end{itemize}

\section{ Dispositivos Móviles y Android }
\begin{itemize}
	\item Distribución de dispositivos móviles entre la población
	cantidad de personas que tienen dispositivos móviles. Que porcentaje de la población representan. Zonas de concentración de dispositivos:cómo están distribuidos
	\item Características de los dispositivos móviles
	cámaras, micrófonos, conexiones a redes, posibilidad de transferencias de archivos, navegabilidad en la interfaz de aplicación.
	\item Android
	el sistema operativo. Licencia. Estructura. Versiones y lo que ello implica.
	\item Ejemplos de aplicaciones de ciencia ciudadana y dispositivos móviles
	hablemos del ejemplo africano que no tenía palabras para que la población partipe sin necesidad de saber leer o escribir. AppEAR y Cazamosquitos. Ejemplo aplicado a salud Colombia
\end{itemize}

\section{ Frameworks }

\begin{itemize}
	\item Frameworks para construir aplicaciones
	\item Configuración de aplicaciones mediante archivos
\end{itemize}

%Hacer una tesis implica encontrar una pregunta que valga la pena responder o un problema que valga la pena resolver y darle respuesta o solución. Es una tarea de investigación que tiene como aspecto muy importante conocer lo que ya existe alrededor de la pregunta o problema que se elige. 

%Al llegar a este capitulo, el lector tiene una idea de cual es el problema. Seguramente se imagina problemas similares o soluciones al problema. El objetivo, en este momento, es convencerlo de que conocemos el problema y otros similares; que conocemos las formas en las que se lo ha intentado resolver (o a problemas similares); y que aún después de saber todo eso sigue siendo un problema importante, difícil y que nadie resolvió 8o nadie revolvió tan bien como nosotros).

%Para escribir este capitulo hay que leer. Hay que buscar soluciones a problemas similares y compararlas con lo que nosotros queremos hacer. Si sabemos que la nuestra es mejor, ya podemos marcar cuales son los puntos débiles de las existentes. También se puede escribir un poco sobre otras investigaciones, que si bien no atacaron problemas parecidos, pueden ser aplicadas a resolver parte de este. 

%Este capitulo es bueno ir escribiendolo en borrador cada vez que se lee algo (un articulo por ejemplo). Por lo menos hay que escribir un resumen de un párrafo de lo leído (registrando la referencia en el archivo bibliografia.bib y citando dede acá), y dar nuestra opinión al respecto en términos de su relación con el problema de nuestra tesis.
