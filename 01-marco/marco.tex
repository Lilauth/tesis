\chapter{Marco Teórico}
\section{Ciencia  Abierta y Ciencia Ciudadana}

No es fácil definir el término Ciencia Abierta ya que abarca una multiplicidad de participantes, cada uno con su enfoque particular. La Ciencia Abierta afecta a investigadores, políticos, desarrolladores y operadores de plataformas, editoriales y público interesado. 

\subsection{Ciencia Abierta}
Ciencia Abierta es un término que engloba otros que tienen que ver con la creación y difusión del conocimiento en el futuro. Tradicionalmente el sistema de creación y difusión del conocimiento está basado en la publicación en revistas científicas. Es por ello que antes de imprimir y difundir un artículo el mismo debe estar completo y correcto, por una cuestión de costos. Publicar en revistas científicas tiene un costo, y si el artículo no está completo y correcto no vale la pena publicarlo. Sin embargo el artículo impreso podría no ser el único formato que esté disponible, habiendo otros menos costosos y que permitirían la publicación de resultados parciales e incluso la corrección o los comentarios por parte de pares o de personas que están trabajando en el mismo tema. Internet tiene los medios para que esto sea posible a través de Wikis y blogs. De esta forma no haría falta esperar que una investigación esté completa para acceder a ella.
Volviendo a la terminología, desde el punto de vista del libro 'Opening Science' \cite{bartling2014opening} se proponen 5 corrientes de pensamiento:

\subsubsection{Escuela Pública} 

	La Escuela Pública está comprometida con hacer que la ciencia sea accesible al público; que esté disponible para una audiencia mayor de interesados. Con las herramientas sociales de la web los investigadores pueden hacer público el proceso de investigación y preparar los resultados para interesados no necesariamente expertos en el área. Hay dos corrientes: una comprometida con la accesibilidad en el proceso de investigación (la producción) y otra con la comprensibilidad de los resultados (el producto).
	
	Al abrir el proceso de investigación, se logra que el público interesado pueda participar como pasó con Foldit, qué fue la interfaz que plantearon los mismos colaboradores que en un primer momento aportaban cediendo fuerza de cómputo mientras sus computadoras no estaban siendo utilizadas. Propusieron una mejora, que era una interfaz en forma de juego que se llamó Foldit. Mientras jugaban resolvían el mapeo de proteínas. En este caso, los colaboradores hicieron su aporte más como una fuerza de trabajo que como pares de una investigación. La idea es poder incluirlos de maneras más significativas. Todavía falta investigar de qué manera se puede involucrar al público en el proceso de investigación.
	
	Por otro lado, los resultados de las investigaciones deberían estar en un formato accesible, no sólo en cuanto a lenguaje utilizado, sino también a los medios de difusión. Los blogs o las herramientas sociales como puede ser Twitter son medios que podrían utilizarse para difundir resultados. El beneficio de que más gente comprenda por qué una investigación puede resultar beneficiosa puede lograr la presión necesaria para obtener fondos de financiamiento. Cuando el lenguaje utilizado en la redacción es complejo o la difusión es escasa, menos oportunidades hay de que el público se entere de qué cosas se están investigando o por qué es importante financiar tal o cual investigación.
	
\subsubsection{Escuela Democrática}	
	
		La Escuela democrática nos habla de accesibilidad no en términos de participación en la investigación o de comprensión de los resultados, sino más bien en acceder a los resultados y la información que generan, pero también a los papers e investigaciones que se usaron como fuentes, multimedia y representaciones gráficas. Todos deberían poder acceder a los resultados de una investigación, sobre todo si está financiada con recursos del estado.
	\begin{itemize}
		\item{Datos Abiertos (Open Data)}
		En el caso de los datos abiertos, el concepto es parecido al del software libre: el derecho a usar información científica debe quedar en la comunidad científica y no ser propiedad de la editorial que lo publica. Las editoriales no deberían tener derechos sobre las publicaciones más que validarlas, en cambio los investigadores si. También, deberían poner la información en formatos accesibles; por ejemplo una tabla en un PDF no es una tabla que pueda revisarse en detalle o modificarse. A la hora de abrir los datos de las investigaciones se presentan dificultades que pueden ser culturales o tecnológicas. Entre las dificultades tecnológicas, muchas veces el sistema de repositorios para soporte de datos es muy complejo y ello provoca resistencia a la hora de compartir información. Desde el punto de vista cultural uno de los problemas es la resistencia a compartir datos que fueron difíciles de reunir. También, muchas veces los datos se comparten de manera más fluida cuando se sabe que va a haber una contraprestación; es decir, se comparte más fácilmente con investigadores que de alguna manera van a aportar o pueden ser un recurso en un determinado momento.
		\item{Acceso Abierto}		
		El Acceso Abierto a las publicaciones científicas se plantea desde un punto de vista político, como un derecho de los humanos de poder acceder al conocimiento. Los países de bajos recursos tienen más dificultades para acceder a material de investigación y esta es otra traba para su posible desarrollo. El acceso al material de investigaciones debe ser de libre acceso sobre todo en investigaciones financiadas por el estado mediante el pago de impuestos de los ciudadanos. De esta manera se evita que los ciudadanos paguen dos veces por un recurso (mediante el pago de impuestos y para acceder al material).
		\end{itemize}

\subsubsection{Escuela Pragmática}
	
	La escuela pragmática se concentra en lograr que el proceso de investigación sea más eficiente. Considera a la investigación científica como un proceso que puede ser optimizado modularizando, abriendo la ``cadena de producción'', incorporando conocimiento externo, permitiendo la colaboración a través de internet. Con las herramientas disponibles en la web el proceso de investigación puede desarrollarse de forma colaborativa, haciendo uso de la inteligencia colectiva. Los problemas que se investigan fueron haciéndose más complejos con el paso del tiempo, y se requiere un esfuerzo conjunto para obtener resultados. Otros beneficios son la mejora en la productividad y la inclusión de conocimiento externo, que apunta otros puntos de vista. 
	
\subsubsection{Escuela de Infraestructura} 
		
		La escuela de infraestructura se ocupa de las tecnologías que hacen posible el desarrollo colaborativo de la ciencia. Son las redes, aplicaciones, plataformas y software que hacen posible esta manera de pensar la ciencia. Estas infraestructuras tecnológicas hacen más fluída la colaboración mediante la existencia de repositorios digitales de información o herramientas on-line de escritura colaborativa. Hay dos tendencias de infraestructura que merecen ser destacadas:
		\begin{enumerate}
			\item {Computación Distribuída}
			El proyecto Open Science Grid es un buen ejemplo de computación distribuída. Open Science Grid, mediante una capa de software permite la interconexión de computadoras para que  formen parte de una grilla de procesamiento. Estas computadoras generalmente están en campus de universidades y laboratorios. De esta manera los investigadores pueden procesar grandes volúmenes de datos y disponer de almacenamiento distribuído. Los proyectos de investigación que utilizan computación distribuída son proyectos que generalmente necesitan analizar un gran volumen de datos. La computación distribuída permite a los investigadores no estar condicionados por los recursos que poseen a la hora de definir los objetivos o el volumen de datos a analizar.
			\item {Redes sociales y de colaboración entre científicos}
			Un Entorno Social Virtual de Investigación (Social Virtual Research Environment - SVRE) debe poseer cuatro características claves:
			\begin{itemize}
				\item Debe permitir administrar y compartir objetos de investigación. Estos pueden ser cualquier elemento digital utilizado en la investigación como información o metodologías.
				\item Debe incentivar a los investigadores a compartir elementos y metodologías de investigación.
				\item El entorno debe ser abierto y fácilmente configurable. Debe permitir la integración de software, servicios y herramientas.
				\item Debe proveer una plataforma para activar la investigación. Esto es lo que hace que el entorno sea considerado una plataforma de investigación.				
			\end{itemize}	
			ResearchGate, Mendeley son ejemplos de plataformas SVRE, en donde los científicos pueden interactuar y compartir.
		\end{enumerate}
		
\subsubsection{Escuela de Valoración} 	
	
	La escuela de valoración quiere hallar formas alternativas de medir el impacto de la salida de la investigación científica. La cantidad de citas a un artículo es la manera de medir el impacto de una investigación. Cuando hablamos de Ciencia Abierta es importante establecer de qué manera se puede medir el impacto de una investigación en la era digital. La medición del impacto tiene los siguientes inconvenientes:
	\begin{itemize}
	\item La revisión entre pares es una actividad que consume tiempo.
	\item El impacto está más asociado a la revista científica que al artículo.
	\item Los nuevos formatos de publicación (blogs, revistas de Acceso Abierto) no se encuentran en el mismo formato que la revista cientifica, y es más complicado hacer una medición del impacto.
	\end{itemize}
	A medida que los trabajos académicos migran a otros formatos su utlización general (compartir, leer, referenciar, utilizar como base de otros trabajos, discutir, etc) va dejando rastros digitales y ofrecen una manera alternativa de medir su impacto. La unidad de medida utilizada se denomina ``altmetric'' (que es la contracción de las palabras alternative metric). Altmetric tiene en cuenta un conjunto de referencias más amplio que sólo la cita: valora los tweets, las discusiones, los bookmarks (en Mendeley, por ejemplo). Altmetric es una aproximación para valorar el impacto de una investigación o un proceso de investigación. 	
	
	
\subsection{Ciencia Ciudadana}
	
	El concepto de Ciencia Ciudadana no es para nada nuevo. Ciencias como la astronomía, las ciencias naturales o proyectos de ecología y conservación hacen uso de la ciencia ciudadana desde antes de que la investigación sea considerada un trabajo formal. Muchos científicos investigaban al margen de su trabajo formal. Ejemplos de ello son Darwin 

\subsubsection{Ciencia Ciudadana en sus Comienzos (Christmas Bird Count)}
	Hasta fines del siglo XIX en Estados Unidos, los cazadores participaban de una cacería tradicional que se celebraba para las fiestas: Christmas ``Side Hunt''. Estos elegían un bando y salían con sus armas a cazar. El bando que reuniera la mayor pila de pieles y plumas era el ganador. Por esta época los conservacionistas, tanto naturalistas como científicos, empezaban a mostrarse preocupados por el declive en la población de aves. Empezando el día de Navidad del año 1900, el ornitólogo Frank M. Chapman propuso una nueva tradición navideña: un censo de aves. La idea era contar las aves en vez de cazarlas. Así comenzó el Christmas Bird Count. Cada año entre el 14 de diciembre y el 5 de enero decenas de miles de voluntarios se unen al esfuerzo de censar aves en América del Norte. El Christmas Bird Count es uno de los proyectos de ciencia ciudadana que más se extendió en el tiempo, y que ha generado información valiosa respecto de la salud de las aves y como base para orientar los pasos a seguir para mejorar su conservación. \cite{CBC}
	
\subsubsection{Calidad de la Información Obtenida}	

Depender de voluntarios no es lo mismo que depender de asistentes. Los voluntarios pueden cometer errores y pueden no entender completamente el contexto del proyecto de investigación. Se destacan dos motivos para utilizar voluntarios: uno de ellos económico y el otro es educativo. Incluir voluntarios permite recolectar datos grandes extensiones geográficas y durante largos períodos de tiempo, que de otra manera no se podrían obtener. Esto sirve para detectar anomalías en la información y comparar datos entre áreas o períodos de tiempo entre otros. Incluir voluntarios hace que los participantes tomen contacto con el mundo natural y participen en el proceso de hacer ciencia. Los proyectos que incluyen ciencia ciudadana muchas veces buscan involucrar a la comunidad de esa manera: haciendo ciencia con sus propias manos.

Con respecto a la calidad de los datos generados, los voluntarios pueden manejar equipamiento, leer resultados y recolectar datos tan bien como cualquier científico o asistente si se les explica cómo hacerlo. Muchas veces recolectar datos no es más que sacar una foto y adjuntarla a un correo electrónico. De todas maneras, los pasos de la recolección de datos deben estar pensados para científicos ciudadanos. También hay que corroborar los datos para determinar si realmente son fiables. En los primeros proyectos de ciencia ciudadana los datos no eran lo suficientemente precisos. Los datos generados por los voluntarios son más parecidos a un rango de valores que a valores específicos y hace más complejo el proceso de obtener resultados comparables o respaldar conclusiones. 

Es recomendable no solicitar que los voluntarios recolecten información compleja y detallada. Pero no es una regla que hay que cumplir. Los voluntarios de proyectos de ciencia ciudadana son en su mayoría científicos, profesores con sus estudiantes, mochileros, integrantes de grupos de conservación, es decir, gente que disfruta del aire libre y que tiene conocimientos sobre temas de biología o ecología y muchas veces son expertos en el tema de investigación. Una estrategia que utilizan los investigadores es entrenar asistentes que tomen muestras en los mismos lugares que los voluntarios y medir los resultados de unos y otros. De esa manera pueden establecer un parámetro de calidad de la información reportada. En un proyecto de investigación que busca identificar cangrejos nativos de la zona con introducidos, en la Costa Atlántica desde New Jersey hasta Maine, encontraron que las tasas de aciertos a la hora de identificar los especímenes es del 95 por ciento para alumnos de séptimo grado y del ochenta por ciento para alumnos de tercer grado. Esta tasa es bastante aceptable para la mayoría de los estudios de ecología. Estos datos sirven para que los investigadores midan la expansión de especies invasoras y puedan generar estrategias de control. \cite{cohn2008citizen}
	

\subsubsection{Clasificación de Proyectos}	

\begin{itemize}
	\item {Acción}
		
		Los proyectos clasificados en esta categoría no son planificados o iniciados por científicos, sino más bien por los ciudadanos, y generalmente requieren un compromiso a lo largo del tiempo en los problemas ambientales locales por lo cual las actividades científicas están orientadas al ambiente físico. 
		
		Estos proyectos solicitan la colaboración de científicos como asesores o consultores, y no como organizadores. Los datos o resultados obtenidos no persiguen un fin académico, sino más bien buscan fundamentar con evidencia para poder tomar acciones sobre alguna situación. 
	\item {Conservación} 
	
	Al igual que los proyectos de Acción, los proyectos de Conservación son fuertemente regionales, y las actividades de los voluntarios están enfocadas mayormente en la recolección de información. La mayoría de los proyectos de investigación tienen contenido o fines educativos. También tienden a ser de alcance regional, como lo reflejan sus desafíos y metas.
	
	Estos proyectos buscan generar información principalmente como fuente para la toma de decisiones relacionadas al manejo de recursos, y también buscan la promoción de la administración y reconocimiento del voluntariado. También prestan especial atención a la generación de información científicamente válida. Son esfuerzos de monitoreo a largo plazo y generalmente no tienen problemas de sustentabilidad, ya que son subvencionadas por fondos públicos o reciben ingresos de agencias que son las que nuclean los proyectos o son las interesadas en sus resultados.
	
	\item {Investigación o Recolección} 
	
	Los proyectos de investigación concentran su atención en investigaciones científicas cuyos objetivos requieren la recolección de información del medio físico. Este tipo de proyectos es el que mejor encaja en la definición de Ciencia Ciudadana. Y aunque los objetivos de educación no son los principales de estos proyectos, forman parte de ellos como material de capacitación o incluyendo estructuras de recolección que alientan el aprendizaje al aire libre. El alcance varía de regional a internacional, y puede lograr participación masiva de hasta decenas de miles de voluntarios y obtener millones de observaciones (de voluntarios) anuales. La mayoría de estos proyectos están enfocados en la investigación biológica, medioambiental o meteorológica, por dar algunos ejemplos. 
	
	Una de las principales preocupaciones de este tipo de proyectos es generar resultados científicamente válidos, ya que son concebidos para generar conocimiento formal y son mayormente organizados por científicos. El cuidadoso diseño del proyecto y de las tareas son los que permiten lograr resultados válidos. Aparte utilizan toda una serie de metodologías que permiten la validación de la información generada. Los voluntarios están dispersos geográficamente y esto es un recurso valioso ya que este tipo de proyectos intenta muchas veces registrar la distribución geográfica de determinadas especies o la ocurrencia de fenómenos naturales. 
	
	\item {Virtual} 
	
	En los proyectos de ciencia ciudadana denominados Virtuales todas las actividades son mediante tecnologías de la información y la comunicación, sin la intervención de elementos del entorno físico.
	
	Los proyectos que cumplen las condiciones para ser clasificados como virtuales provienen de la astronomía, la paleontología y la proteómica, que es una rama de la microbiología que estudia la estructura de las proteínas. Algunos proyectos de psicología podrían clasificar, pero no lo hacen porque los voluntarios colaboran como sujetos de pruebas, y esto no es formalmente considerado como colaboración en la investigación. 
	
	Galaxy Zoo es un ejemplo de proyecto virtual de ciencia ciudadana. Desde hace más de diez años los voluntarios que colaboran con el proyecto clasifican galaxias en fotografías. Responden una serie de preguntas respecto de la foto que están observando, y de esta manera los científicos encargados del proyecto obtienen una primera clasificación, que se construye en base a las observaciones de varios voluntarios de manera independiente. \cite{GalaxyZoo} 
	
	Al igual que en los proyectos de recolección anteriormente mencionados, los proyectos virtuales encuentran dificultades a la hora de conseguir resultados válidos en términos científicos. Estos resultados se obtienen mediante el desarrollo cuidadoso de las actividades. Como la participación es mayormente virtual, poder mantener a los voluntarios comprometidos con el proyecto es un desafío. Por eso muchas veces estos proyectos incluyen técnicas de gamificación, de competencia amigable entre participantes o de valoración de la contribución hecha por el voluntario (feedback).
	 
	\item {Educación} 
	
	Los proyectos de ciencia ciudadana pertenecientes a esta categoría son aquellos cuyo principal objetivo es educar. Los participantes de estos proyectos tienen como objetivo educar, y aportan recursos educativos informales; mientras los proyectos ofrecen material educativo formal. También, las actividades están pensadas para que el participante vaya acumulando conocimientos.
	
	Un ejemplo de este tipo de proyectos es Fossil Finders, que centra la investigación en el análisis de fósiles del Devónico (período de la era Paleozóica) proveyendo de materiales de estudio a estudiantes y profesores de escuelas secundarias. Los estudiantes van a identificar y medir fósiles en muestras de rocas enviadas a sus aulas. Luego ingresarán los datos obtenidos en una base de datos online, y podrán comparar sus datos con los de otras escuelas participantes. Con ello van a tener la oportunidad de involucrarse con métodos de investigación reales y de asistir a los investigadores del Instituto de Investigación Paleontológica a reconstruir el pasado geológico de Nueva York. \cite{FossilFinders}

	La mayoría de estos proyectos proyectos persiguen un fin educativo, el aprendizaje y el desarrollo de habilidades científicas. Por ello incluyen actividades de análisis de datos o muestras, brindando la posibilidad de desarrollar pensamiento crítico. 
	\end{itemize} 

\subsubsection{Proyectos de Recolección}	 
Son los proyectos que nos interesan porque nuestro framework es para ayudar a generar aplicaciones para tomar muestras para proyectos de investigación de recolección. 

\subsubsection{Tecnologías de la Información y Ciencia Ciudadana}	 
Acá podemos hablar un poco de cómo las tecnologías son un buen soporte de los proyectos de ciencia ciudadana. Hablaremos de Zooniverse, de Galaxy Conqueror (que incluye gamificación) y de la comunidad que se forma para preguntar cosas interesantes o despejar inquietudes antes de dar algo por supuesto. Podemos hablar de la astrónoma amateur que encontró una nube verde en el espacio y terminó siendo algo que nadie había visto antes. Lo mismo con el astrónomo amateur rosarino que pudo registrar el nacimiento o muerte de una estrella, cosa que antes no había podido ser captada.

