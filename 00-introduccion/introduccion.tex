\chapter{Introducción}

\label{introduccion}

%[¿en que tema están trabajando? ] Introduce en contexto en el que están trabajando (p.e., el ámbito en el que se dá el problema). Da definiciones (brevemente) de conceptos que aparecen en el contexto. 

Acá introducción de ciencia ciudadana y samplers \cite{wtf}

\section{ Estructura de la Tesina }
Este trabajo de tesina se organiza de la siguiente manera:
\begin{itemize} 
	\item{Capítulo 1} 
		\begin{description} [Breve explicación de lo que se trata en el capitulo 1]
		%En este capítulo se introduce el marco teórico que encuadra este proyecto. Se describe brevemente cuál es el problema que se quiere resolver y por qué es interesante resolverlo. Se plantean los objetivos de este trabajo de tesina.
		\end{description}

	\item{Capítulo 2} 
		\begin{description} [Breve explicación de lo que se trata en el capitulo 2]
		%En este capítulo se describe el marco teórico de manera más exhaustiva, explicando cuáles son los fundamentos teóricos que apoyan a este trabajo. Damos una introducción a la Ciencia Ciudadana y la Ciencia Abierta, como así también de frameworks.
		\end{description}
	
	\item{Capítulo 3} 
		\begin{description} [Breve explicación de lo que se trata en el capitulo 3]
		%Describe las herramientas utilizadas en la construcción del framework. Android como sistema operativo. Características de los dispositivos móviles. Gradle para scripting y manejo de dependencias. Android Studio como estándar para desarrollar para Android. Google Services para identificación del usuario y de la geolocalización. 
		\end{description}
	
	\item{Capítulo 4} 
		\begin{description} [Breve explicación de lo que se trata en el capitulo 4]
		%Describe el framework desarrollado. Jerarquía de clases. Objetivos y colaboración. Estructura de los resultados de la muestra tomada. Almacenamiento y transmisión de los datos.
		\end{description}

	\item{Capítulo 5} 
		\begin{description} [Breve explicación de lo que se trata en el capitulo 5]
		%En este capítulo se presentan las conclusiones de este trabajo. Se analiza de qué manera Samplers puede ser una buena contribución a los proyectos de ciencia ciudadana. También, se habla de los resultados y lo aprendido en el desarrollo. Se habla de Samplers2, la interfaz web para crear proyectos de ciencia ciudadana con Android y se documentan posibles lineamientos para trabajos futuros.
		\end{description}
		
	\item{Capítulo 6} 
		\begin{description} [Breve explicación de lo que se trata en el capitulo 6]
		\end{description}
		
	\item{Capítulo 7} 
		\begin{description} [Breve explicación de lo que se trata en el capitulo 7]
		\end{description}				
\end{itemize}

\section{ Motivación }

Los proyectos de investigación científica a menudo requieren la realización de gran número de actividades que son difíciles de automatizar como puede ser la clasificación de fotos, anotaciones, observaciones y todo tipo de actividades que en esencia son simples, pero consumen mucho tiempo. Muchas veces estas actividades son sencillas y no se necesita de ninguna preparación académica o escolarizada previa para realizarlas, por ejemplo indicar si en una foto se observa o no un animal. La ciencia ciudadana es una forma de investigación en colaboración que involucra a los ciudadanos resolviendo este tipo de tareas simples en proyectos de investigación científica que buscan resolver problemas del mundo real \cite{wiggins2011conservation}. 

Un científico ciudadano es un voluntario que recoge y/o procesa información como parte de una investigación científica \cite{silvertown2009new}. Para que los voluntarios puedan participar en estos proyectos es necesario brindarles herramientas que los ayuden a contribuir. 
Nuestro interés está enfocado en los proyectos de recolección. Estos proyectos de investigación científica requieren la recopilación de datos del medio físico. Una forma de asistir a estos proyectos es por medio de sistemas informáticos que posibiliten la recolección de datos usando móviles. Un ejemplo de este tipo de proyectos es AppEAR un sistema de ciencia ciudadana para cuidar y aprender de los ambientes acuáticos en Argentina, realizado por Joaquín Cochero, investigador del CONICET en el Instituto Platense de Limnología. El objetivo final de AppEAR es tener un relevamiento completo y detallado de aguas continentales de todo el territorio nacional para conocer los lugares en riesgo en los que urge trabajar. Los voluntarios de este proyecto descargan una aplicación para su dispositivo móvil y toman muestras para el proyecto. La aplicación guía a los usuarios a través de los pasos necesarios para tomar una muestra.

La mayoría de los proyectos de ciencia ciudadana de recolección cuentan con aplicaciones desarrolladas específicamente para cada proyecto, en donde el principal problema a resolver es la secuencia de pasos que conforman el protocolo para la toma de la muestra y la combinación de este protocolo y de las herramientas del dispositivo móvil que se desean utilizar cómo puede ser la cámara, el GPS, el micrófono para grabar un audio. Consideramos que proveer un framework que resuelva esta problemática, la de la aplicación específica de cada proyecto, sería útil para la creciente comunidad de científicos que quieren incluir ciencia ciudadana en sus proyectos.

Este proyecto se enmarca dentro de Cientópolis\cite{cientopolis}, una plataforma para la promoción y el estudio de la Ciencia Ciudadana. Cientópolis se nuclea como un proyecto de investigación desde la Facultad de Informática de la UNLP pero articula su funcionamiento con investigadores de las facultades de Ciencias Astronómicas y Geofísicas, Humanidades y Ciencias de la Educación, Bellas Artes y Ciencias Naturales y Museo.

\section{ Objetivos }		
		
Se propone desarrollar un framework para instanciar aplicaciones móviles Android de ciencia ciudadana. El framework recibirá un archivo con la configuración requerida en formato JSON y generará una aplicación para ejecutarse en un dispositivo Android. En este archivo estará el conjunto de pasos que especifican el protocolo de recolección de muestras. Estos pasos pueden ser:
			\begin{itemize}
				\item captura de una foto, un video, un audio, una ubicación o un recorrido hecho con el dispositivo móvil.
				\item contestar una pregunta con respecto a la muestra. Esta pregunta puede tener una o múltiples respuestas posibles.
				\item introducir anotaciones de texto.
				\item indicar una fecha y hora.
				\item mostrar información de orientación y ayuda para la toma de la muestra.
			\end{itemize}

La aplicación generada servirá para tomar muestras siguiendo el protocolo de recolección especificado y las almacenará y empaquetará en el dispositivo móvil hasta que pueda ser enviada a un servidor web.
		
Se define el formato del archivo de configuración de la aplicación y la información adicional necesaria, como pueden ser credenciales para acceder a los servicios de Google Services o el posicionamiento por GPS.

Instanciar una aplicación básica de ejemplo con el framework en base a un archivo de configuración, que permita tomar algunas muestras y enviarlas a un servidor web que estará configurado para dicho propósito.
			