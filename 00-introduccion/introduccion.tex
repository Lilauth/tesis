\chapter{Introducción}

\label{introduccion}

\section{ Motivación } \label{sec:motivacion}

Los proyectos de investigación científica a menudo requieren la realización de gran número de actividades que son difíciles de automatizar como puede ser la clasificación de fotos, anotaciones, observaciones y todo tipo de actividades que en esencia son simples, pero consumen mucho tiempo. Muchas veces estas actividades son sencillas y no se necesita de ninguna preparación académica o escolarizada previa para realizarlas, por ejemplo indicar si en una foto se observa o no un animal. La ciencia ciudadana es una forma de investigación en colaboración que involucra a los ciudadanos resolviendo este tipo de tareas simples en proyectos de investigación científica que buscan resolver problemas del mundo real \cite{wiggins2011conservation}. 

Un científico ciudadano es un voluntario que recoge y/o procesa información como parte de una investigación científica \cite{silvertown2009new}. Para que los voluntarios puedan participar en estos proyectos es necesario brindarles herramientas que los ayuden a contribuir. 
Nuestro interés está enfocado en los proyectos de recolección. Estos proyectos de investigación científica requieren la recopilación de datos del medio físico. Una forma de asistir a estos proyectos es por medio de sistemas informáticos que posibiliten la recolección de datos usando móviles. Un ejemplo de este tipo de proyectos es AppEAR un sistema de ciencia ciudadana para cuidar y aprender de los ambientes acuáticos en Argentina, realizado por Joaquín Cochero, investigador del CONICET en el Instituto Platense de Limnología. El objetivo final de AppEAR es tener un relevamiento completo y detallado de aguas continentales de todo el territorio nacional para conocer los lugares en riesgo en los que urge trabajar. Los voluntarios de este proyecto descargan una aplicación para su dispositivo móvil y toman muestras para el proyecto. La aplicación guía a los usuarios a través de los pasos necesarios para tomar una muestra.

La mayoría de los proyectos de ciencia ciudadana de recolección cuentan con aplicaciones desarrolladas específicamente para cada proyecto, en donde el principal problema a resolver es la secuencia de pasos que conforman el protocolo para la toma de la muestra y la combinación de este protocolo y de las herramientas del dispositivo móvil que se desean utilizar cómo puede ser la cámara, el GPS, el micrófono para grabar un audio. Consideramos que proveer un framework que resuelva esta problemática, la de la aplicación específica de cada proyecto, sería útil para la creciente comunidad de científicos que quieren incluir ciencia ciudadana en sus proyectos.

Este proyecto se enmarca dentro de Cientópolis\cite{cientopolis}, una plataforma para la promoción y el estudio de la Ciencia Ciudadana. Cientópolis se nuclea como un proyecto de investigación desde la Facultad de Informática de la UNLP pero articula su funcionamiento con investigadores de las facultades de Ciencias Astronómicas y Geofísicas, Humanidades y Ciencias de la Educación, Bellas Artes y Ciencias Naturales y Museo.

\section{ Objetivos }\label{sec:objetivos}		
		
Se propone desarrollar un framework para instanciar aplicaciones móviles Android de ciencia ciudadana. El framework recibirá un archivo con la configuración requerida en formato JSON y generará una aplicación para ejecutarse en un dispositivo Android. En este archivo estará el conjunto de pasos que especifican el protocolo de recolección de muestras. Estos pasos pueden ser:
			\begin{itemize}
				\item captura de una foto, un video, un audio, una ubicación o un recorrido hecho con el dispositivo móvil.
				\item contestar una pregunta con respecto a la muestra. Esta pregunta puede tener una o múltiples respuestas posibles.
				\item introducir anotaciones de texto.
				\item indicar una fecha y hora.
				\item mostrar información de orientación y ayuda para la toma de la muestra.
			\end{itemize}

La aplicación generada servirá para tomar muestras siguiendo el protocolo de recolección especificado y las almacenará y empaquetará en el dispositivo móvil hasta que pueda ser enviada a un servidor web.
		
Se define el formato del archivo de configuración de la aplicación y la información adicional necesaria, como pueden ser credenciales para acceder a los servicios de Google Services o el posicionamiento por GPS.

Instanciar una aplicación básica de ejemplo con el framework en base a un archivo de configuración, que permita tomar algunas muestras y enviarlas a un servidor web que estará configurado para dicho propósito.

\section{ Estructura de la Tesina }
Este trabajo de tesina se organiza de la siguiente manera:
\begin{itemize} 
	\item{Capítulo 1} 
		 En este capítulo se plantea la motivación de este trabajo y se definen los objetivos a alcanzar. Se detalla la estructura de esta tesina.

	\item{Capítulo 2} 
		
		Este capítulo comienza explicando qué es la ciencia ciudadana y cómo es que su popularidad está en aumento. Detalla una clasificación de proyectos de ciencia ciudadana y luego ahonda en uno de los tipos de la clasificación, que son aquellos proyectos donde la inclusión de los científicos ciudadanos se realiza en la recolección de información o muestras. Se describe el método científico ya que los proyectos de investigación que utilizan ciencia ciudadana o cuyo diseño gira en torno a incluir a científicos ciudadanos son proyectos que utilizan las bases de la investigación científica, es decir, son proyectos que utilizan el método científico. A diferencia de los proyectos de investigación donde todos sus participantes son científicos o personas con conocimiento o experticia en el área de estudio, los proyectos de ciencia ciudadana deben tener especial cuidado definiendo los protocolos de recolección de la información para que sus participantes puedan seguirlos. Por último, teniendo en cuenta que los dispositivos móviles están al alcance de muchas personas, presentamos datos de uso de dispositivos móviles y sistemas operativos en el país.
		
	
	\item{Capítulo 3} 
		
		Se introduce el concepto de framework y se describe una clasificación en base a su diseño y tipo de especialización. Luego se describe el estado de tres herramientas que asisten a los investigadores en la creación y la administración de proyectos de recolección que utilizan ciencia ciudadana. Se analizan las ventajas y las desventajas de las herramientas descritas en la sección.
		
	
	\item{Capítulo 4} 
		
		Se describe el sistema operativo para dispositivos móviles Android y se detallan sus principales componentes de aplicación. La Activity, el componente principal de las aplicaciones en Android. Características, ciclo de vida y posibles estados. Relación entre Fragment y Activity. Ciclo de vida y estads del Fragment. Propósito de los Services y tipos soportados. Método de suscripción de eventos del sistema y de otras aplicaciones, BroadcastReceiver. 
		 

	\item{Capítulo 5} 
		
		Samplers, un framework para construir aplicaciones Android para proyectos de recolección que utilizan ciencia ciudadana. Alcance y descripción de la solución propuesta con Samplers. Workflow o protocolo para la recolección de la muestra. Descripción y ejemplo del archivo para configuración de una aplicación. Pasos para la recolección de la muestra: Step y Workflow y su relación con los principales componentes de las aplicaciones Android. Sample, la muestra, resultado de la ejecución del workflow por parte de un científico ciudadano. Envío de muestras por internet y persistencia local para envío manual o cuando tenga disponibilidad.
		
		
	\item{Capítulo 6} 
		Utilizando el framework se instancia la aplicación AppEar y se comparan en cuanto a funcionalidad. Los resultados obtenidos demuestran que con Samplers se puede instanciar fácilmente una aplicación para ciencia ciudadana.
		
		
	\item{Capítulo 7} 
		En este capítulo se sumariza el desarrollo de la tesina y se evalúan nuevamente los objetivos con los resultados obtenidos y el grado de cumplimiento. También se detallan posibles trabajos a futuro que permitirían que el proyecto crezca y se mantenga en el tiempo.		
						
\end{itemize}


			