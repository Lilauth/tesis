\chapter{Trabajo Relacionado}
Hacer una tesis implica encontrar una pregunta que valga la pena responder o un problema que valga la pena resolver y darle respuesta o solución. Es una tarea de investigación que tiene como aspecto muy importante conocer lo que ya existe alrededor de la pregunta o problema que se elige. 

Al llegar a este capitulo, el lector tiene una idea de cual es el problema. Seguramente se imagina problemas similares o soluciones al problema. El objetivo, en este momento, es convencerlo de que conocemos el problema y otros similares; que conocemos las formas en las que se lo ha intentado resolver (o a problemas similares); y que aún después de saber todo eso sigue siendo un problema importante, difícil y que nadie resolvió 8o nadie revolvió tan bien como nosotros).

Para escribir este capitulo hay que leer. Hay que buscar soluciones a problemas similares y compararlas con lo que nosotros queremos hacer. Si sabemos que la nuestra es mejor, ya podemos marcar cuales son los puntos débiles de las existentes. También se puede escribir un poco sobre otras investigaciones, que si bien no atacaron problemas parecidos, pueden ser aplicadas a resolver parte de este. 

Este capitulo es bueno ir escribiendolo en borrador cada vez que se lee algo (un articulo por ejemplo). Por lo menos hay que escribir un resumen de un párrafo de lo leído (registrando la referencia en el archivo bibliografia.bib y citando dede acá), y dar nuestra opinión al respecto en términos de su relación con el problema de nuestra tesis.