\chapter{Framework para Proyectos Android de Ciencia Ciudadana}

\section{Frameworks}

\subsection{Tipos de Frameworks}
Podemos hablar de frameworks de código privativo y de código abierto.

\subsection{Jerarquías y Composición}
De qué manera los frameworks permiten que los usuarios los configuren y les pongan comportamiento

\subsection{El Entorno de Android}
Herencia de las clases Activity y Fragment como principal herramienta para implementar código nativo Android.

\subsection{Estado del Arte}
\subsubsection{Sensr: un framework flexible para crear herramientas de recolección de datos con dispositivos móviles para ciencia ciudadana}

	Los dispositivos móviles son ideales para que las personas puedan de manera espontánea recolectar información. Sin embargo, esa simplicidad yace sobre una base que requiere fuertes conocimientos técnicos una infraestructura compleja. Por lo tanto, construir aplicaciones móviles implican una inversión que puede ser limitante para organizaciones pequeñas. Sensr es una herramienta que permite que personas que no son desarrolladoras tengan la posibilidad de crear herramientas que permitan la recolección de información para proyectos de ciencia ciudadana con dispositivos móviles. Esta herramienta aprovecha que el proceso y la estructura de la información en las actividades de recolección de datos de los proyectos de ciencia ciudadana son similares independientemente del dominio o la diversidad de los mismos. Sensr combina un ambiente de programación gráfico con una aplicación móvil para que las personas que no necesariamente poseen conocimientos técnicos puedan construir herramientas de recolección de información para dispositivos móviles y administrar la información recabada de manera colectiva.
	
	De esta manera, una persona que necesitan reunir información puede ser el autor de una campaña de ciencia ciudadana en el sitio de Sensr. La campaña es desplegada en la aplicación móvil de Sensr, y sus usuarios se pueden suscribir y contribuir a la campaña con los datos recolectados. Esta herramienta pretende simplificar de manera radical el proceso de crear una herramienta para dispositivos móviles que permita recolectar información y que sea de utilidad en una amplio conjunto de dominios de ciencia ciudadana. Los autores sólo necesitarían completar la descripción del proyecto y diseñar las plantillas o formularios que permitan el ingreso de los datos antes de incluir su proyecto en Sensr y ser distribuido de forma masiva. De esta manera, los autores se liberarían de las preocupaciones acerca de los requerimientos técnicos y las restricciones de la infraestructura.
	
	La falta de expertos técnicos y de recursos son a menudo los mayores obstáculos a la hora de desarrollar una aplicación móvil. Los grupos que quieren desarrollar una aplicación móvil de ciencia ciudadana a menudo son organizaciones sin fines de lucro o pequeñas organizaciones regionales que no poseen ni los recursos económicos ni los expertos técnicos que necesitan para desarrollar o mantener ese tipo de aplicaciones. Y además de la programación en si, la administración de los datos recolectados también representan un desafío, ya que estas mismas organizaciones tampoco poseen los servidores para almacenar o analizar el volumen de datos que puedan ser recolectados. El monitoreo participativo es un paradigma computacional que permite la recolección por parte de los voluntarios de información que se encuentra diseminada. Permite que el creciente número de usuarios de teléfonos móviles puedan compartir la información adquirida mediante los sensores de sus dispositivos en variados dominios. 
	
	Los investigadores han explorado la utilización de plataformas existentes como una alternativa para Y aunque varias de ellas son robustas y flexibles, la mayoría necesita de habilidad para programar y/o conocimiento de infraestructura en mayor o menor medida. Aunque por ejemplo el Proyecto Noah y EpiCollect son dos ejemplos claros de plataformas que soportan autoría de aplicaciones sin necesidad de programación.
	
	
	\cite{kim2013sensr}