\chapter{Samplers: Framework Android}

\section{Propuesta general}
Samplers es un framework que permite construir, de manera sencilla, aplicaciones Android para recolectar muestras en proyectos de Ciencia Ciudadana. Brinda una solución simple al problema de la recolección de la muestra usando/aprovechando las funcionalidades de los dispositivos móviles.

En un principio se pensó para que un científico pudiera crear su propia aplicación móvil de ciencia ciudadana sin tener conocimientos de programación. La idea inicial era que mediante una aplicación web el científico pudiera armar el protocolo de recolección de las muestras (el Workflow en Samplers) de manera visual e intuitiva, y se generara un archivo de configuración para Samplers. Con el archivo de configuración se pasaría a Samplers y se generaría la aplicación móvil. Pero de esta forma el alcance de la tesis era muy grande, por lo que se decidió quitar la parte de la aplicación web y suponer que que el archivo de configuración ya viene armado.

Por otro lado, abarcar los 3 sistemas operativos móviles más usados en ese momento (Android, iOS y Windows Phone) también haría muy grande el alcance de la tesis, por lo que se decidió optar por Android, que era el sistema operativo móvil más usado en ese momento. Según una estadística de Gartner sobre las ventas de smartphones a nivel mundial en el último trimestre de 2016\cite{gartner}, mas del 80\% de las mismas fueron de celulares con Android.

Para desarrollar aplicaciones móviles para la plataforma Android, el entorno de desarrollo integrado (IDE) oficial es Android Studio\cite{androidStudio}, por lo que Samplers se apoya sobre el mismo. Android Studio ha sido publicado de forma gratuita bajo Licencia Apache 2.0 y está disponible para las plataformas Microsoft Windows, MacOS y GNU/Linux.

De este modo, Samplers sería un framework que funcionaría sobre Android Studio, que recibiría un archivo de configuración y generaría una aplicación lista para ejecutarse en un dispositivo móvil con Android.

Para el archivo de configuración se eligió el formato JSON porque nos pareció más fácil de escribir/leer a mano (que XML por ejemplo, que también fue evaluado), suponiendo que el científico tuviese que armarlo a mano. Se establecieron las opciones de configuración del archivo para poder armar el workflow, que representa el protocolo de recolección de las muestras.



Samplers funciona de dos formas:
\begin{itemize}
\item Usando el generador de clases de Gradle: 
\item Extendiendo las clases: 
\end{itemize}


** hablar de FrozenSpots y HotSpots (ver la tesis de Spotters)


** Configurable o Extensible nosotros creo que hicimos los dos... :P

\section{Jerarquía de Clases}
Nuestras clases, las que permiten que se cree una app

\includegraphics[scale=0.4]{05-implementacion/Steps.png} 


\subsection{Workflow}
La clase Workflow representa el protocolo de recolección de muestras. Está formado por una colección de Steps que representan los pasos a seguir para completar dicho protocolo.


Es el encargado de llevar el estado del paso en el que se encuentra.



los steps son...
Los StepResult son el resultado obtenido de ejecutar un Step.
Una muestra esta formada por la colección de StepResult generada luego de ejecutar cada Step del Workflow. Cada ejecución del workflow puede generar una colección de StepResults diferente.
Un StepResult tiene el Id del Step en base al cual se generó. También tiene un método....

\subsubsection{Step: el Paso}

\subsubsection{StepResult: El Resultado de la Ejecución del Paso}

\subsubsection{Sample: la muestra}
La clase Sample representa una muestra tomada a partir de seguir los pasos (Steps) del protocolo del recolección (Workflow). Contiene los resultados de la ejecución de cada paso (StepResult). También guarda fecha y hora de inicio y finalización. Esto es útil para sacar una estadística de cuanto tarda un usuario en recolectar una muestra y analizar optimizaciones para la aplicación final.

\subsection{Envío de Muestras a Servidor Web}
Acá explicamos un poco la estrategia que utilizamos para el envío de muestras y las librerías externas que nos ayudaron.

Usuario afuera de la muestra porque cuando toma la muestra puede no estar logueado, y loguearse antes de enviarlas.

\section{Instalación y uso del framework}
\subsection{Instalación}
En esta sección se describe como instalar el framework en Android Studio y como instanciarlo para su uso, brindando algunos ejemplos concretos.

\subsubsection{Requerimientos mínimos:}

\begin{itemize}
\item Android Studio (Java): Si bien esta pensado para y probado en Android Studio, podria funcionar bien en otro entorno que use Java y deje importar archivos Android Archive (.aar).
\item Android SDK API17: Android 4.2 (Jelly Bean) o superior.
\end{itemize}

\subsubsection{Pasos para la Instalación:}
\begin{enumerate}
	\item Crear en Android Studio un nuevo proyecto vacío (sin ninguna Activity).
		\begin{itemize}
		\item Seleccionar API17 o superior como versión mínima de Android SDK
		\end{itemize}
	\item Importar la librería del framework en el proyecto creado
		\begin{itemize}
		\item Descargar la última versión de samplersFramework.aar desde https://github.com/cientopolis/samplers/releases/
		\item Importar la librería al proyecto: File -> New -> New Module -> Import .JAR/.AAR Package
		\end{itemize}
	\item Agregar el repositorio de Google
		\begin{itemize}
			\item En el archivo build.gradle del proyecto agregar: 
			\begin{lstlisting}[language=XML, frame=single]
allprojects {
    repositories {
        jcenter()
        google()
    }
}
			\end{lstlisting}	
		\end{itemize}
	\item Agregar las dependencias necesarias:
		\begin{itemize}
			\item En el archivo build.gradle de la aplicación agregar: 
			\begin{lstlisting}[language=XML, frame=tlb]
dependencies {
  // here the standards dependencies created by Android Studio
  // ...

  // if not added automatically, add this dependency 
  // you should use the latest version e.j. 25.+
  compile 'com.android.support:design:24.2.1' 
  compile 'com.android.support.constraint:constraint-layout:1.0.2'

  // if you will use maps and location services, add this dependencies (you should use the latest version)
  compile ('com.google.android.gms:play-services-location:12.0.1')
  compile ('com.google.android.gms:play-services-maps:12.0.1')
  
  // if you will use authentication with Google, add this dependencies (you should use the latest version)
  compile ('com.google.android.gms:play-services-auth:12.0.1')

  // the framework dependency
  compile project(":samplersFramework")
}
			\end{lstlisting}	
		\end{itemize}	

	\item Instanciar:
		\begin{itemize}
		\item La instanciación puede ser manual o usando el generador de clases en Gradle como se explica en la siguiente sección.
		\end{itemize}
\end{enumerate}	

\subsection{Instanciación}
Una vez instalado el framework...
La instanciación puede ser manual o usando el generador de clases de Gradle.

\subsubsection{Instanciación manual}
Básicamente se tiene que crear un objeto Workflow, agregarle los objetos Steps, y llamar a la activity TakeSampleActivity pasándole el workflow como parámetro.
También es necesario establecer la configuración general en el método onCreate de la activity principal (main activity).
Se puede usar una activity principal propia o se puede heredar de SamplersMainActivity. En ambos casos se debe hacer lo siguiente:
\begin{itemize}
	\item Establecer la configuración general en el método onCreate de la activity principal:
		\begin{lstlisting}[language=Java, frame=tlb]
NetworkConfiguration.setURL("http://192.168.1.10/samplers/upload.php");
NetworkConfiguration.setPARAM_NAME_SAMPLE("sample");
// Optional if you will use authentication
NetworkConfiguration.setPARAM_NAME_USER_ID("user_id");
NetworkConfiguration.setPARAM_NAME_AUTHENTICATION_TYPE("authentication_type");

// Optional if you will use authentication, set the configuration
AuthenticationManager.setAuthenticationEnabled(true);
AuthenticationManager.setAuthenticationOptional(true);
		\end{lstlisting}

	\item Crear un Workflow. Si se está heredando de SamplersMainActivity se debe hacer sobreescribiendo el método onCreate.
		\begin{lstlisting}[language=Java, frame=tlb]
@Override
protected Workflow getWorkflow() {
	Workflow workflow = new Workflow();
    	
	Step step = new InformationStep(2,"Por favor tome una foto de su gato", null);
	workflow.add(step);
    	
	step = new PhotoStep(1,"Bienvenido a la app de prueba", 2);
	workflow.add(step);
    	
	return workflow;
    	
}		
		\end{lstlisting}
Nota: en el ejemplo anterior se muestra un Workflow que tiene dos Steps. El primero muestra un mensaje de bienvenida y el segundo pide para tomar una foto. Para ver los distintos Steps que se pueden usar vea la sección de Steps.

	\item Iniciar la activity TakeSampleActivity. Si se está heredando de SamplersMainActivity esto se hace solo en el método onClick del botón "tomar muestra". De lo contrario, deberá iniciarla de la siguiente forma, en el método onClick de un botón por ejemplo:
		\begin{lstlisting}[language=Java, frame=tlb]
@Override
public void takeSampleClick(View view) {
	Workflow workflow = getWorkflow();

	Intent intent = new Intent(this, TakeSampleActivity.class);        
	intent.putExtra(TakeSampleActivity.EXTRA_WORKFLOW, workflow);
	startActivity(intent);
    	
}		
		\end{lstlisting}




\end{itemize}


\subsubsection{Instanciación usando el  generador de clases de Gradle}

Básicamente, el generador de clases de Gradle se encarga de hacer una instanciación manual a partir de un archivo de configuración (JSON). Está pensado para desarrolladores que no tienen muchos conocimientos en Android, o para servir de interfaz entre una aplicación que genere apps a través de Samplers (esto hay que escribirlo mejor).

Los pasos para usar el generador de clases de Gradle son:
\begin{enumerate}
	\item Crear un archivo JSON con el nombre SamplersConfig.json
		\begin{itemize}
			\item El formato y las opciones están explicadas en la siguiente sección.
			\item Para validar sintaxis se puede usar el validador online: jsonformatter.curiousconcept.com (Ver si se puede poner esto o hay que tener permisos?)
			\item Al final de la sección se provee un archivo de ejemplo
		\end{itemize}
		
	\item Copiar el archivo creado en el item anterior al directorio raíz del proyecto Android
	
	\item Descargar la última versión de los archivos \textbf{samplers.gradle} y \textbf{samplersclassgenerator.jar} del repositorio de Samplers (https://github.com/cientopolis/samplers/releases/) y copiarlos también al directorio raíz del proyecto Android
	
	\item Enlazar el archivo samplers.gradle en el archivo build.gradle de la aplicación
		\begin{itemize}
			\item Android Studio crea por defecto dos archivos build.gradle, uno a nivel de aplicación y otro a nivel de proyecto. Debe usarse el de aplicación
			\item Al final del archivo build.gradle de aplicación agregar:
\begin{lstlisting}[language=XML, frame=tlb]
apply from: '../samplers.gradle'
\end{lstlisting}
			\item Al guardar los cambios, Android Studio sugerirá hacer una sincronización del proyecto, hacerla. Esto generará en la aplicación una activity llamada MyMainSamplersActivity en base a las opciones configuradas en el archivo SamplersConfig.json.
			\item Si se necesita volver a generar esta activity (si se quieren modificar algunas opciones por ejemplo ) se puede eliminar la misma, hacer las modificaciones en el archivo SamplersConfig.json y volver a generar el proyecto (en el menu Build -> Make Project)
		\end{itemize}
		
	\item Eliminar o (customizar) el archivo \textbf{style.xml} que está en \textbf{res/values} en la aplicación
	
	\item Ejecutar la aplicación y listo.

\end{enumerate}


\subsection{Secciones del Archivo}

El archivo SamplersConfig.json es un archivo JSON con 3 objetos:
\begin{itemize}
	\item El objeto \textbf{project}
		
	El objeto project tiene dos campos
	\begin{itemize}
		\item \textbf{app\_path}: Un String con la ubicación del directorio de los fuentes de la aplicación, relativo al directorio del proyecto. Es donde están los archivos -java de la aplicación y donde se creará el archivo MyMainSamplersActivity.java
		\item \textbf{package\_name}: Un String con el nombre del package usado para las activities de la aplicación. Es el package donde la activity MyMainSamplersActivity será agregada.
	\end{itemize}
	
Ejemplo:
\begin{lstlisting}[language=XML, frame=tlb]	
{
  "project" : {
    "app_path" = "app/src/main/java/com/example/myApplication/"
    "package_name" : "com.example.myApplication"
  }
}
\end{lstlisting}	
	
	\item El objeto \textbf{application}
	El objeto application tiene 7 campos, de los cuales 3 son requeridos y los otros 4 opcionales (para habilitar características especiales)
	\begin{itemize}
		\item \textbf{title}: Un String con el nombre de la aplicación.
		
		 \item \textbf{welcomeMessage}: Un String con el mensaje de bienvenida que se mostrará en la activity principal (MyMainSamplersActivity)
		 
		 \item \textbf{networkConfiguration}: Un objeto con la configuración de red que se usará para enviar las muestras al servidor web. Ver mas abajo la configuración de este objeto.
		 
		 \item \textbf{googleMaps\_API\_KEY}: [Opcional] Un String con la API Key de Google. Este campo es necesario si se van a usar los servicios de ubicación y mapas (Location Step y Route Step). La API Key de google se puede obtener desde la página de google developers (https://developers.google.com/maps/documentation/android-api/signup)
		 
		 \item \textbf{mainHelpFileName}: [Opcional] Un String con el nombre del archivo HTML que contiene la ayuda principal de la aplicación. Este archivo debe estar junto con el archivo SamplersConfig.json. Ver la sección Mostrando Ayuda para mas detalles.
		 
		 \item \textbf{authenticationEnabled}: [Opcional] Un boolean que indica si se usará autenticación (true) o no (false). Si se omite este campo se asume false. Ver la sección Usando Autenticación para mas detalles.
		 
		 \item \textbf{authenticationOptional}: [Opcional] Un boolean que indica si la autenticación será opcional (true) o requerida (false). Si se omite este campo se asume true (autenticación opcional). Este campo solo tiene sentido si se usa autenticación. Ver la sección Usando Autenticación para mas detalles.
		 
	\end{itemize}
	
	
	El objeto \textbf{networkConfiguration}:
	El objeto networkConfiguration contiene la configuración de red que se usará para enviar las muestras al servidor web. Tiene 4 campos, de los cuales 2 son requeridos y los otros 2 opcionales.
	
	\begin{itemize}
	
		\item \textbf{url}: Un String con la URL del servidor web al cual se le enviaran las muestra con un mensaje HTTP POST.
		
		\item \textbf{paramName}: Un String con el nombre del parámetro dentro del mensaje HTTP POST en el que se enviará la muestra.
	
		\item \textbf{paramNameUserId}: (Opcional) Un String con el nombre del parámetro dentro del mensaje HTTP POST en el que se enviará el id del usuario que envía la muestra. Este campo solo es necesario si se usa autenticación. Ver la sección Usando Autenticación para mas detalles.
		
		\item \textbf{paramNameAuthenticationType}: (Opcional) Un String con el nombre del parámetro dentro del mensaje HTTP POST en el que se enviará el tipo de autenticación que usó el usuario que envía la muestra. Este campo solo es necesario si se usa autenticación. Ver la sección Usando Autenticación para mas detalles.
	
	\end{itemize}
	
	
Ejemplo:
\begin{lstlisting}[language=XML, frame=tlb]
{
  "application": {
    "title" : "Samplers Hello World App",
    "welcomeMessage" : "Welcome to your first Samplers App!",
    "networkConfiguration" : {
      "url" : "http://192.168.1.10/samplers/upload.php",
      "paramName" : "sample",
      "paramNameUserId" : "user_id",
      "paramNameAuthenticationType" : "authentication_type"
    },
    "authenticationEnabled" : true,
    "authenticationOptional" : true,
    "googleMaps_API_KEY" : "your_google_maps_API_KEY",
    "mainHelpFileName" : "mainhelp.html"
  } 
}
\end{lstlisting}	
	
	
	\item El objeto \textbf{workflow}
	El objeto workflow representa el protocolo para la toma de la muestra. Son los pasos que se ejecutarán para tomar la misma.
	El objeto cuenta con dos campos:
		
	\begin{itemize}
	
		\item \textbf{actionLabel}: Un String con el título que se usará para el botón que inicia la activity TakeSampleActivity, que es la encargada de tomar la muestra.
		
		\item \textbf{steps}: Un Array de Objetos Step los cuales forman el workflow. El primer objeto del array se considera como el inicio del mismo. Ver la sección Steps para mas detalles.
	
	
	\end{itemize}	
	
	
Ejemplo:
\begin{lstlisting}[language=XML, frame=tlb]	
{
  "workflow": {
    "actionLabel" : "Tomar muestra",
    "steps": [
      {
        "id" : 1,
        "type" : "Information",
        "text" : "Por favor, siga las instrucciones",
        "nextStepId" : 2
      },
      {
        "id" : 2,
        "type" : "Location",
        "text" : "Por favor posicione la muestra en el mapa",
        "nextStepId" : 3,
        "helpFileName" : "locationhelp.html"
      },
      {
        "id" : 3,
        "type": "MultipleSelect",
        "title" : "Seleecione las cosas que ve",
        "helpFileName" : "selecthelp.html",
        "options" : [
          {
            "id":1,
            "text":"Arboles"
          },
          {
            "id":2,
            "text":"Basura"
          },
          {
            "id":3,
            "text":"Agua"
          }
        ]
      }
    ]
  }
}

\end{lstlisting}	
	
\end{itemize}

Aca podriamos poner un ejemplo completo de un archivo JSON de los que estan en la wiki.

\subsection{Configuración de los Servicios de Google}

\section{Los diferentes Steps y sus resultados (StepResult)}
Esto me parece que ya esta explicado arriba... hay que ver donde lo explicamos en forma generica y aca el detalle de cada uno.
los steps son...
Los StepResult son el resultado obtenido de ejecutar un Step.
Una muestra esta formada por la colección de StepResult generada luego de ejecutar cada Step del Workflow. Cada ejecución del workflow puede generar una colección de StepResults diferente.
Un StepResult tiene el Id del Step en base al cual se generó. También tiene un método....

\subsection{InformationStep: Mostrar información}

En Android Studio (Java):
\begin{lstlisting}[language=Java, frame=tlb]	
InformationStep step = new InformationStep(1,"Texto para mostrar",2);
\end{lstlisting}

Usando el generador de clases:
\begin{lstlisting}[language=XML, frame=tlb]	
{
	"id":1,
	"type" : "Information",
	"text" : "Texto para mostrar",
	"nextStepId": 2
}
\end{lstlisting}

\subsubsection{InformationStepResult: El resultado de Mostrar información}
El resultado de mostrar información es nulo, es una clase vacía. Solo está para cerrar el circuito.

\subsection{SelectOneStep: Seleccionar una opción de un grupo de opciones}
Tiene un title
Las opciones son una lista de objetos SelectOneOption
Cada objeto SelectOneOption tiene un id, textToShow y nextStepId
Explicar la bifurcación de caminos en el workflow a partir de este Step
En forma de radio buttons

En Android Studio (Java):
\begin{lstlisting}[language=Java, frame=tlb]	
ArrayList<SelectOneOption> optionsToSelectOne = new ArrayList<SelectOneOption>();
optionsToSelectOne.add(new SelectOneOption(1,"Opcion 1", 2));
optionsToSelectOne.add(new SelectOneOption(2,"Opcion 2", 2));
optionsToSelectOne.add(new SelectOneOption(3,"Opcion 3", 3));
SelectOneStep step = new SelectOneStep(1,optionsToSelectOne,"Seleccione una opcion");

\end{lstlisting}

Usando el generador de clases:
\begin{lstlisting}[language=XML, frame=tlb]	
{
  "id" : 1,
  "type" : "SelectOne",
  "title" : "Seleccione una opcion",
  "options" : [
    {
      "id":1,
      "text":"Opcion 1",
      "nextStepId" : 2
    },
    {
      "id":2,
      "text":"Opcion 2",
      "nextStepId" : 2
    },
    {
      "id":3,
      "text":"Opcion 3",
      "nextStepId" : 3
    }
  ]
}
\end{lstlisting}

\subsubsection{SelectOneStepResult: El resultado de Seleccionar una opción de un grupo de opciones}
El resultado tiene la opción seleccionada (un objeto SelectOneOption)

\subsection{MultipleSelectStep: Seleccionar varias opciones de un grupo de opciones}
Tiene un title
Las opciones son una lista de objetos MultipleSelectOption
Cada objeto MultipleSelectOption tiene un id y textToShow
en forma de checkboxes

En Android Studio (Java):
\begin{lstlisting}[language=Java, frame=tlb]	
ArrayList<MultipleSelectOption> optionsToSelect = new ArrayList<MultipleSelectOption>();
optionsToSelect.add(new MultipleSelectOption(1,"Arboles"));
optionsToSelect.add(new MultipleSelectOption(2,"Basura"));
optionsToSelect.add(new MultipleSelectOption(3,"Agua"));
optionsToSelect.add(new MultipleSelectOption(4,"Animales"));
MultipleSelectStep step = new MultipleSelectStep(1,optionsToSelect,"Seleccione lo que ve",2); 
\end{lstlisting}

Usando el generador de clases:
\begin{lstlisting}[language=XML, frame=tlb]	
{
  "id" : 1,
  "type" : "MultipleSelect",
  "title" : "Seleccione lo que ve",
  "options" : [
    {
      "id":1,
      "text":"Arboles"
    },
    {
      "id":2,
      "text":"Basura"
    },
    {
      "id":3,
      "text":"Agua"
    },
    {
      "id":4,
      "text":"Animales"
    }
  ],
  "nextStepId" : 2
}
\end{lstlisting}

\subsubsection{MultipleSelectStepResult: El resultado de Seleccionar varias opciones de un grupo de opciones}
El resultado tiene una lista de las opciones seleccionadas (una lista de objetos MultipleSelectOption)

\subsection{PhotoStep: Tomar una foto}
Tiene instructionsToShow que se muestran a modo de instrucciones o mensaje cuando la cámara esta encendida
muestra un preview
usa una camara dependiendo de la api

En Android Studio (Java):
\begin{lstlisting}[language=Java, frame=tlb]	
PhotoStep step = new PhotoStep(1,"Por favor tome una foto de su gato",2);
\end{lstlisting}

Usando el generador de clases:
\begin{lstlisting}[language=XML, frame=tlb]	
{
  "id" : 1,
  "type" : "Photo",
  "text" : "Por favor tome una foto de su gato",
  "nextStepId" : 2
}
\end{lstlisting}

\subsubsection{PhotoStepResult: El resultado de Tomar una foto}
guarda el imageFileName de la foto. 
La foto va como archivo jpg en la carpeta de la muestra.
Puede haber varias fotos si hay varios PhotoSteps en el workflow

\subsection{SoundRecordStep: Grabar sonido}
Tiene instructionsToShow a modo de instrucciones que se muestran.

En Android Studio (Java):
\begin{lstlisting}[language=Java, frame=tlb]	
SoundRecordStep step = new SoundRecordStep(1,"Grabe el sonido de su auto",2); 
\end{lstlisting}

Usando el generador de clases:
\begin{lstlisting}[language=XML, frame=tlb]	
{
  "id" : 1,
  "type" : "Sound",
  "text" : "Grabe el sonido de su auto",
  "nextStepId" : 2
}
\end{lstlisting}

\subsubsection{SoundRecordStepResult: El resultado de Grabar sonido}
guarda el soundFileName del sonido.
El sonido va como archivo mp4 en la carpeta de la muestra.
Puede haber varios sonidos si hay varios SoundRecordSteps en el workflow


\subsection{LocationStep: Posicionar la muestra en el mapa con el GPS}
Tiene un textToShow a modo de instrucciones
Permite usar el GPS o posicionar la muestra manualmente en el mapa

En Android Studio (Java):
\begin{lstlisting}[language=Java, frame=tlb]	
LocationStep step = new LocationStep(1,"Por favor posicione la muestra en el mapa",2); 
\end{lstlisting}

Usando el generador de clases:
\begin{lstlisting}[language=XML, frame=tlb]	
{
  "id" : 1,
  "type" : "Location",
  "text" : "Por favor posicione la muestra en el mapa",
  "nextStepId" : 2
}
\end{lstlisting}

\subsubsection{LocationStepResult: El resultado de Posicionar la muestra en el mapa con el GPS}
Guarda latitude y longitude

\subsection{RouteStep: Grabar un recorrido en el mapa usando el GPS}
Tiene un textToShow a modo de instrucciones
Intervalo y mapZoom opcionales. Poner los valores por defecto

En Android Studio (Java):
\begin{lstlisting}[language=Java, frame=tlb]	
RouteStep step = new RouteStep(1,"Registre la ruta que corre",2); 
step.setInterval(10000);
step.setMapZoom(18);
\end{lstlisting}

Usando el generador de clases:
\begin{lstlisting}[language=XML, frame=tlb]	
{
  "id" : 1,
  "type" : "Route",
  "text" : "Registre la ruta que corre",
  "interval" : 10000,
  "mapZoom" : 18,
  "nextStepId" : 2
}
\end{lstlisting}

\subsubsection{RouteStepResult: El resultado de Grabar un recorrido en el mapa usando el GPS}
guarda una lista de objetos Location

\subsection{InsertTextStep: Ingresar texto}
textToShow a modo de instrucciones
sampleText a modo de ejmplo
maxLength cantidad máxima de caracteres permitida
Type Values allowed are: text, number or decimal
optional indicando si se puede dejar vacío y no ingresar ningún texto (true) o si se requiere que ingrese algo si o si (false)


En Android Studio (Java):
\begin{lstlisting}[language=Java, frame=tlb]	
InsertTextStep step = new InsertTextStep(1,"Por favor, ingrese el nombre del lago","Nombre del lago",50,InsertTextStep.InputType.TYPE_TEXT,true,2);
\end{lstlisting}

Usando el generador de clases:
\begin{lstlisting}[language=XML, frame=tlb]	
{
  "id" : 1,
  "type" : "InsertText",
  "text" : "Por favor, ingrese el nombre del lago",
  "sampleText" : "Nombre del lago",
  "inputType" : "text",
  "maxLength" : 50,
  "optional" : true,
  "nextStepId" : 2
}
\end{lstlisting}

\subsubsection{InsertTextStepResult: El resultado de Ingresar texto}
guarda el texto ingresado en insertedText

\subsection{InsertDateStep y InsertTimeStep: Ingresar fecha y hora}
ambos tienen textToShow a modo de instrucciones

En Android Studio (Java):
\begin{lstlisting}[language=Java, frame=tlb]	
InsertDateStep step = new InsertDateStep(1,"Por favor indique la fecha de la muestra",2); 
\end{lstlisting}

Usando el generador de clases:
\begin{lstlisting}[language=XML, frame=tlb]	
{
  "id" : 1,
  "type" : "InsertDate",
  "text" : "Por favor indique la fecha de la muestra",
  "nextStepId" : 2
}
\end{lstlisting}

En Android Studio (Java):
\begin{lstlisting}[language=Java, frame=tlb]	
InsertTimeStep step6 = new InsertTimeStep(1,"Por favor indique la hora de la muestra",2); 
\end{lstlisting}

Usando el generador de clases:
\begin{lstlisting}[language=XML, frame=tlb]	
{
  "id" : 1,
  "type" : "InsertTime",
  "text" : "Por favor indique la hora de la muestra",
  "nextStepId" : 2
}
\end{lstlisting}

\subsubsection{InsertDateStepResult y InsertTimeStep: El resultado de Ingresar fecha y hora}
un objeto Date que tiene la fecha o la hora según corresponda

\section{Mostrar Ayuda}

\section{Usando autenticación}
Por defecto provee autenticación con Google, porque al tener Android tiene una cuenta de Google si o si.
Esta abierto a poder agregar autenticación con otras plataformas/APIs.

Samplers provee autenticación con Google, pero ese necesario registrar la aplicación en la pagina de desarrolladores de google (https://developers.google.com/identity/sign-in/android/start-integrating). Ahí hay que seguir los pasos para [to Configure a Google API Console project]. Es necesario [proveer] el nombre de la aplicación, package name, y también el SHA-1 hash del certificado con el que se firma la aplicación.

Una vez registrada la aplicación en Google, hay que configurar Samplers para habilitar la autenticación.

Una vez configurado los valores de los parámetros de autenticación, Samplers mostrará un fragment de inicio de sesión la primera vez que el usuario intente tomar una muestra. Si la autentición es opcional, se mostrará un botón para omitir el inicio de sesión y continuar con la toma de la muestra. También se muestra un botón para iniciar sesión en la activity principal (si se está usando la que provee Samplers).

Cuando la muestra es enviada, el id de usuario y el método de autenticación (por defecto 'google') se envían junto con esta.


\subsection{Configurar autenticación con el generador de clases de Gradle}

Para usar autenticación usando el generador de clases de Gradle, es necesario configurar los siguientes parámetros en el objeto \textbf{applicaction}:

\begin{itemize}

	\item \textbf{authenticationEnabled}: poner en true para habilitar la autenticación
		
	\item \textbf{authenticationOptional}: poner en true si se desea que la autenticación sea opcional, o en false si se desea que la autenticación sea obligatoria.
	
	\item Dentro del parámetro \textbf{networkConfiguration} es necesario establecer los parámetros \textbf{paramNameUserId} y \textbf{paramNameAuthenticationType} con los nombres de los parámetros con los que irán el id de usuario y el tipo de autenticación usada respectivamente dentro del mensaje HTTP POST.
	

\end{itemize}

Ejemplo:

\begin{lstlisting}[language=XML, frame=tlb]	
{
  "application": {
    "title" : "Samplers Hello World App",
    "welcomeMessage" : "Welcome to your first Samplers App!",
    "networkConfiguration" : {
      "url" : "http://192.168.1.10/samplers/upload.php",
      "paramName" : "sample",
      "paramNameUserId" : "user_id",
      "paramNameAuthenticationType" : "authentication_type"
    },
    "authenticationEnabled" : true,
    "authenticationOptional" : true
  } 
}
\end{lstlisting}

\subsection{Configurar autenticación [con instanciación manual]}

Para usar autenticación [con instanciación manual], es necesario definir la configuración de red y de autenticación en el método \textbf{onCreate()} de la activity principal:

\begin{lstlisting}[language=Java, frame=tlb]	
@Override
protected void onCreate(Bundle savedInstanceState) {
  super.onCreate(savedInstanceState);
	
  NetworkConfiguration.setURL("http://192.168.1.10/samplers/upload.php");
  NetworkConfiguration.setPARAM_NAME_SAMPLE("sample");
  // Set the authentication params of the Network Configuration
  NetworkConfiguration.setPARAM_NAME_USER_ID("user_id");
  NetworkConfiguration.setPARAM_NAME_AUTHENTICATION_TYPE("authentication_type");

  // Set the authenticationconfiguration
  AuthenticationManager.setAuthenticationEnabled(true);
  AuthenticationManager.setAuthenticationOptional(true);
}
\end{lstlisting}

\subsection{Usando un método de autenticación propio}

Con Samplers también se puede usar un método de autenticación propio, definiendo un LoginFragment y una clase User (o varias clases si se desea proporcionar autenticación con diferentes APIs, como Facebook, Tweeter, Yahoo, etc.) y Samplers enviará junto con la muestra el id de usuario y el método de autenticación usado.


\subsubsection{Definiendo un Login Fragment [personalizado]}

Es necesario crear un fragment que herede de LoginFragment, y configurar la clase AuthenticationManager para que lo use, llamando al método setLoginFragmentClass() dentro del método onCreate() de la ativity principal:

\begin{lstlisting}[language=Java, frame=tlb]	
@Override
protected void onCreate(Bundle savedInstanceState) {
  super.onCreate(savedInstanceState);
	
  AuthenticationManager.setLoginFragmentClass(MyCustomLoginFragment.class);
}
\end{lstlisting}

El proceso de login y la interacción con las APIs responsabilidad del desarrollador, pero después de que el usuario inicia sesión en la API seleccionada, es necesario llamar al método login() en la clase AuthenticationManager y al método onLogin() en el objeto mListener heredado:

\begin{lstlisting}[language=Java, frame=tlb]	
if (loginOK) {
  AuthenticationManager.login(user, getActivity().getApplicationContext());
  mListener.onLogin(user);
}
\end{lstlisting}


\subsubsection{Definiendo una clase User [personalizada]}

Es necesario crear una clase usuario propia que implemente la interfaz User por cada método de autenticación que se use. Los objetos de dichas clases se usarán para llamar al método login() de la clase AuthenticationManager.

Ejemplo de una clase usuario propia:
\begin{lstlisting}[language=Java, frame=tlb]	
public class EMailUser implements User {

    public static final String AUTHENTICATION_TYPE = "email";

    private String userName;
    private String email;

    public GoogleUser(String userName, String email) {
        this.userName = userName;
        this.email = email;
    }

    @Override
    public String getAuthenticationType() {
        return AUTHENTICATION_TYPE;
    }

    @Override
    public String getUserName() {
        return userName;
    }

    @Override
    public String getUserId() {
        return email;
    }

}
\end{lstlisting}



\section{Personalización}

\subsection{Temas y colores}

\subsection{Idiomas}


** Ver Auth0

