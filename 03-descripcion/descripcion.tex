\chapter{Características de los Proyectos de Recolección}

\section{Proyectos de Recolección}

	
%Los proyectos de recolección generalmente tienen algunas complejidades particulares. Se necesita mucha gente para tomar las muestras, cubren grandes extensiones de territorio. Los voluntarios son imprescindibles en estos proyectos, pero también la capacitación previa o conocimiento en el área aportan. La toma de muestras sigue determinado conjunto de pasos cuyo orden debe respetarse para que la muestra sea considerada una muestra. %

\subsection{El Método Científico}
No podemos hablar mucho acá, sino más que nada dar una introducción al método científico, que es el que determina por qué una muestra es una muestra, explicar un poco cómo sería el protocolo de la toma de muestras de campo y eso. Esto podemos consultarlo, para ver de donde sacara bibliografía.

\subsection{Toma de Muestras}

	El artículo Citizen Science: Can volunteers do real research? describe el caso de un voluntario que colabora \textit{con un proyecto de ciencia ciudadana en el Appalachian Trail}. Luego de caminar varios kilómetros, se detiene en un lugar predeterminado y saca su GPS para estar seguro de que es el lugar correcto. Se aparta del sendero peatonal y busca un sendero menos transitado en el bosque más denso. Encuentra un camino lleno de huellas de ciervo y excremento que le indica que es el camino que está buscando.
	
	Este voluntario preside el Natural Bridge Appalachian Trail en Lynchburg, y camina de un lado a otro buscando una cámara digital que dejó en un árbol un mes atrás. Luego de buscar durante varios minutos la encuentra. Apaga la cámara, reemplaza la tarjeta de memoria y las pilas. Vuelve al sendero con huellas de animales, lo recorre dos tercios de milla y lo vuelve a dejar en otro árbol cuidando que el objetivo apunte al sendero. Luego se pone en la mira de la cámara y se mueve hasta escuchar el obturador activado por el sensor de movimiento. 
	
	De esta manera el y otros cientos colaboran con un proyecto de censado de mamíferos del Appalachian Trail desde el sur de Virginia hasta Pennsylvania, y manejan equipamiento, recogen información y anotan observaciones como uno de los varios proyectos que manejan en conjunto agencias gubernamentales, universidades, grupos de conservación ecológica y científicos para supervisar las tendencias ambientales en las cerca de 2175 millas del Appalachian Trail. Tanto el censo de mamíferos como el Appalachian Trail MEGA Transect dependen de voluntarios. Los científicos ciudadanos ayudan a supervisar animales salvajes, plantas u otros objetivos medioambientales y no reciben un pago por su colaboración e incluso a veces no son científicos profesionales. Son aficionados que colaboran con estos proyectos porque disfrutan de estar afuera o porque están comprometidos con los problemas ecológicos y quieren hacer algo al respecto. Generalmente no analizan la información o escriben artículos científicos, pero son esenciales para recolectar la información en la que luego se basan los estudios. 
	
	La Ciencia Ciudadana no es nueva. Lo que es nuevo es la cantidad de voluntarios que se enlistan para colaborar en los estudios, y la amplitud de la información que se les pide recoger. Los investigadores a menudo les piden a los colaboradores que utilicen equipamiento y técnicas sofisticadas para monitorear la calidad del aire y el agua; que documenten el crecimiento, la floración y la muerte de las plantas o que observen cuando las aves y otros animales migran atravesando un área o de qué manera se comportan mientras están en la misma. 
	
%En esta sección podemos ampliar el tema de las características de los proyectos de recolección. Hablar de CBC que abarca una parte muy grande de América del Norte, podemos hablar del proyecto de investigación por el cual AppEAr existe, que es el relevamiento de estuarios en Argentina. Podemos hablar de la app africana para relevar determinadas cosas que hay en terrenos súper inaccesibles, esta app es súper interesante porque es sólo pictográfica, lo que la vuelve amigable para mucha de la gente que no sabe leer ni escribir. %

\subsection{Protocolos de Recolección de Muestras}

	Los protocolos para recolección de muestras en proyectos que incluyen a científicos ciudadanos deben ser simples. Es más sencillo pedir que se identifique, documente o cuente 5 o 10 especies diferentes de plantas que sean fácilmente reconocibles y que serviría para indicar que la especie está presente en el área, en vez de pedir que se identifiquen todas las especies presentes en un área determinada. Una manera de ayudar a los voluntarios es darle libros con guías o material impreso que los ayude.
	
	Como hacemos referencia en el marco teórico, en los primeros estudios la información recolectada por los científicos ciudadanos era imprecisa como para ser utilizada. El principal problema es que la información generada por los voluntarios a veces representan valores en un rango en vez de números específicos, lo que dificulta la detección de cambios en los valores o fundamentar conclusiones. Ahora, los científicos ciudadanos son entrenados para leer instrumentos y recolectar números específicos. Esto debe seguir siendo compatible con protocolos de recolección simples. De esta manera se evitaría caer en el error de solicitarle a los voluntarios la recolección de información muy compleja o detallada. Pero algunas veces se puede solicitar este tipo de complejidad o precisión en la recolección de información. Pero esto no siempre es así. Muchos de los voluntarios que participan en los estudios tienen algún tipo de conocimiento acerca del método científico. De todas maneras debe esperarse información de calidad variada. En esos casos los científicos que dirigen los proyectos deben estar preparados para escrutar cuidadosamente la información obtenida y deben estar dispuestos a descartar información sospechosa o poco confiable.\cite{cohn2008citizen} 
		
\subsection{Diseño de Proyectos de Ciencia Ciudadana}

	El diseño y la implementación de cada proyecto requiere que se tomen decisiones acerca de los intereses de qué público se podría y  debería ser tenido en cuenta a la hora de perseguir intereses, y cómo los objetivos finales, o resultados esperados son definidos. En algunos campos donde puede participar la ciencia ciudadana, las decisiones de diseño son guiadas por teorías de participación, experiencia, o democracia.
	
	A la hora de diseñar un proyecto, una de las primeras preguntas que hay que responder es 'a los intereses de quién o de quienes sirve?'. De esta manera van a quedar un conjunto de opciones resultantes a tomar para implementarlo. Es la negociación entre los intereses científicos y los intereses públicos lo que puede influenciar un rango de resultados potenciales. Public Participation in Scientific Research \cite{shirk2012public} propone un framework para diseñar proyectos de ciencia ciudadana. Los elementos de este framework  son entradas, actividades, salidas, resultados e impacto. 
	
\begin{itemize}
	\item {Entradas}
		Los proyectos de ciencia ciudadana son, por la forma de definirse, un esfuerzo colaborativo, y es por ello que su diseño debe permitir entradas de múltiples constituyentes. Es decir, sus participantes tiene múltiples capacidades y múltiples maneras de brindar su colaboración. Por dar un ejemplo, hay varias maneras de describir una misma imagen. Algunas pueden ser más extensas y detalladas y otras descripciones pueden serlo menos. Estas entradas son los intereses (las esperanzas, deseos, objetivos y expectativas) tanto del público como de la comunidad científica en conjunto a la hora de determinar el objetivo de un proyecto. Podría haber más intereses, pero se considerarán estos dos, las entradas del framework.
		
		Para los voluntarios, sus intereses pueden ser contribuir a generar conocimiento científico, recolectar y diseminar información con respecto a peligros medioambientales, afectar la administración de recursos, proteger [livelihoods], o para satisfacer necesidades que tienen que ver con su identidad personal u objetivos de aprendizaje. Y aunque sería fácil asumir que los intereses de los científicos son principalmente conseguir resultados científicos, bien podrían estar interesados en afectar la educación, la conservación, en manejar la información surgida de sus propias observaciones,o cualquiera de los intereses atribuidos a los voluntarios. Además, estos intereses no son homogéneos incluso dentro del mismo grupo de investigadores o comunidad científica. Y, para tener en cuenta, la línea que divide a los individuos considerados `científicos' de los que son `el público' suele no estar bien definida.
		
		Estos intereses pueden ser utilizados a la hora de definir los objetivos del proyecto para conseguir resultados específicos. De esta manera, las Entradas implican una interacción entre los intereses de los científicos profesionales y los miembros del público que fueron tenidos en cuenta en el desarrollo o participación del proyecto de investigación. Cada iniciativa debería equilibrar estos intereses para identificar el foco del trabajo científico, que puede ser una pregunta de sondeo, recolección de datos o un protocolo de monitoreo.
		
		\item {Actividades}
		en las Actividades podemos encontrar el grueso del trabajo necesario para diseñar, establecer y administrar todos los aspectos del proyecto. Este trabajo generalmente lo dirige un equipo, que puede incluir científicos profesionales, miembros del público y otros participantes (como pueden ser docentes, tecnólogos, etc). En este contexto las Actividades incluyen las tareas necesarias para el desarrollo de la estructura de un proyecto, como el diseño de las estrategias y los protocolos para recolección de muestras, material de entrenamiento, tecnología de ingreso y envío de información, y el establecimiento de una red de voluntarios y los mecanismos de soporte y comunicación necesarios para mantener el nivel de participación. Las Actividades también incluyen tareas para el manejo de la implementación del proyecto, como pueden ser facilitar entrenamiento, distribución de material, organizar reuniones y eventos, y comunicarse con los colaboradores/participantes.
		
		Establecer una infraestructura para la recolección y administración de la información determina la cantidad y calidad de la información recolectada así como también la utilidad de la información que afecta los resultados de la investigación. La manera en que se manejan las actividades es lo que va a reflejar cuán bien fueron equilibrados los intereses en la etapa anterior (Entradas), ya que son los que van a influenciar las opciones relacionadas a qué medir, cuán seguido se toman esas mediciones y quien tiene el control sobre la información resultante. Conseguir colaboración de científicos en estas actividades puede incrementar la confianza en la información recolectada. Lograr que el público y la comunidad se involucren en estas actividades pueden elevar la relevancia tanto científica como local de los descubrimientos que puedan conseguirse. 
		
		\item {Salidas}
		
		El principal resultado de las actividades son las salidas. Estas incluyen observaciones, persistidas como información, y la experiencia activa de hacer, facilitar y/o analizar esas observaciones o medidas recolectadas. Pueden cuantificarse en términos de números de observaciones en una base de datos, cantidad de individuos, de visitas en un sitio web, horas-voluntario, workshops, y entrenamiento. Las diferencias en las salidas de los proyectos a menudo dependen de por qué y cómo la información se recolecta, cómo se utiliza, y el significado que se le atribuye, así también como la intensidad y el significado que se le atribuya a la experiencia vivida.
		
		La decisión de qué información va a ser recolectada, y cómo se va a poder disponer y utilizar entre las diferentes partes que constituyen el proyecto, también influencian fuertemente las salidas, incluyendo publicaciones, educación y toma de decisiones. El análisis de la información, los workshops, la visualización y diseminación de los datos por medio de la comunidad o por medio de publicaciones, el influenciamiento en los dirigentes y la reflexión personal acerca de las experiencias vividas en el proyecto, todo esto antes mencionado en conjunto posibilita la transición de información tangible y experiencias en salidas del proyecto. La prioridad y los recursos invertidos en alguno o algunos de los intereses planteados al comienzo del proyecto influyen en el tipo de observaciones y experiencias que se recolectan  y en la manera en que la información resultante es utilizada.
		
		\item {Resultados}
		
		Los resultados son habilidades, destrezas y conocimientos resultantes de las salidas del proyecto, y que son cuantificables. Como hablamos de proyectos de participación pública en el marco de la conservación y la ecología, podemos hablar de resultados que generalmente caen en tres categorías: científica, de participación individual, y de sistemas socio-ecológicos. 
		
		Consideremos los resultados en lo concerniente a ciencia. Por dar algunos ejemplos, los proyectos de ciencia ciudadana han sido la base de avances en el conocimiento y entendimiento científico en temas como: rango de tendencias, distribución, abundancia y diversidad de especies; esparcimiento de enfermedades o de especies invasivas; cambios en eventos del ciclo de vida y el impacto de esos cambios en la salud. Este tipo de proyectos, los de ciencia ciudadana, también han conseguido innovar y mejorar las técnicas para recolectar, analizar, administrar y difundir información. Uno de los puntos cruciales que brinda la ciencia ciudadana, es lograr tomar conocimiento que de otra manera sería inaccesible, mediante la compilación de información en redes de comunicación a gran escala o poder disponer de datos que tienen que ver con conocer de manera exhaustiva el entorno local. Teniendo en cuenta lo mencionado anteriormente, es importante notar que los tipos de resultados relacionados con la ciencia que un proyecto puede lograr pueden depender de las presunciones de los diseñadores del proyecto acerca de qué es lo que cuenta como conocimiento y el conocimiento y las observaciones de quiénes son relevantes.
		
		Los resultados relacionados a cada participante de manera individual incluyen el desarrollo de un nuevo conjunto de habilidades, un incremento de la comprensión del proceso científico, una mejora en la relación con el entorno o la administración, y la oportunidad de profundizar los lazos con la naturaleza así como con otras personas. Algunos participantes aumentan sus conocimientos con nuevos contenidos o incrementan la literatura científica. Otros ganan un sentido de protagonismo en su propia experticia y conocimiento en la medida en que contribuyen con la ciencia y su entorno y contexto sociales. Los científicos profesionales también obtienen resultados individuales al igual que los participantes. Un estudio documenta la mejora en el entendimiento de las condiciones regionales y una mirada más profunda acerca del conocimiento y las habilidades de los cosechadores indocumentados de frutos del bosque que lograron las agencias de personal. Otro estudio sugiere que involucrarse en trabajo de campo con compañeros puede traerles alivio en lo que respecta a sus trabajos de oficina e incluso incrementar el sentimiento de esperanza en la (de a ratos desalentadora) tarea de conservación. 
		
		Del lado de los sistemas socio-ecológicos se pueden incluir mejora en las relaciones entre comunidades y agencias de administración, mejoras secundarias del ambiente silvestre, el acceso y la utilización de la información para estudiar la degradación medioambiental, e incrementar las ganas de los participantes para involucrarse en los procesos políticos para mejorar su entorno. 
		
		La mejora en la flexibilidad de las prácticas de administración y el proceso de aprendizaje social en conjunto que se encuentra embebido en la supervisión colaborativa y participativa, puede contribuir a sistemas socio-ecológicos más resistentes. Muchos de estos resultados están basados en la profundidad de la contribución y los lazos que se formen, esta categoría puede estar influenciada mayormente por la calidad de la participación al comienzo, en la etapa de Entradas. Es importante destacar que no todos los proyectos producen resultados en todas las categorías antes mencionadas, cualquiera sean los objetivos finales, y algunos pueden tener resultados inesperados. Lograr conseguir resultados en una categoría puede influenciar resultados en otras. Y asi como los proyectos evolucionan, los resultados influencian la forma en que se manejan las subsiguientes Entradas. Por ejemplo, conseguir resultados relacionados a la ciencia refuerza el interés científico. Para que un proyecto sea sostenible es deseable que obtenga resultados en las tres categorías antes mencionadas. En proyectos con buen diseño, las Entradas pueden ser entendidas o interpretadas como objetivos, y los resultados deben ser el reflejo de esas entradas. 
		
	\item {Impacto}
	
	Comparado con los resultados, los impactos son cambios sostenidos y a largo plazo sobre los que se basan mejoras en el bienestar de la humanidad o la conservación de recursos naturales. Se considera que los resultados en el corto plazo son medidos generalmente entre los años uno y tres de la implementación de un proyecto y los resultados a largo plazo a los cuatro o seis años, los impactos evidentes pueden darse a los diez años o más de que el proyecto fue establecido. Dada esta escala de tiempos, los impactos de un proyecto casi nunca pueden ser medidos. No obstante, los programas de conservación pueden beneficiarse distinguiendo impactos de resultados para dirigir los intereses quienes están interesados operando con diferentes escalas de tiempo, como pueden los administradores de los territorios o las agencias que prestan sus fondos para financiar los proyectos.
	
	Entre los impactos deseables se encuentra la conservación y administración sustentable, una ciudadanía fortalecida y poseedora de conocimientos, ambientes humanos y naturales resistentes, y ciencia adaptativa. Por la naturaleza de los impactos, es más probable conseguirlos mediante la combinación exitosa de resultados para la investigación, para los participantes y para los sistemas socio-ecológicos.
		
\end{itemize}	

\cite{shirk2012public}


\section{Ciencia Ciudadana y Tecnología}

	Los teléfonos celulares son dispositivos que están presentes en casi todos los ámbitos y su capacidad de capturar, clasificar y transmitir imágenes, acústica, ubicación y otra información de manera autónoma o interactiva está en crecimiento.
Planteando la arquitectura adecuada, pueden actuar como una red de sensores e instrumentos de recolección de información de localización. 
Esta forma de red de sensores distribuidos puede tener aplicaciones científicas, industriales y militares. Se sabe menos acerca de su función y utilidad en la esfera pública, es decir cuando los que los operan y poseen son usuarios regulares.   Estos sensores en vez de estar en manos de un coordinador central, están siempre bajo el control de sus dueños.

	Solicitar que los dispositivos móviles que ya están [deployados] en el campo, que armen redes de sensores de manera interactiva, participativa y que le permitan a los usuarios del público general y a los profesionales reunir, analizar y compartir información regional. Los micrófonos y cámaras que están presentes en el [handset] del dispositivo pueden registrar información del entorno, mientras se siguen integrando otros sensores de manera inalámbrica. La localización brindada por las antenas de telefonía, el GPS y otras tecnologías proveen información de ubicación y [time-synchronization]. [Las radios wireless] La conexión y el procesamiento que brinda el dispositivo permiten la interacción con la información procesada tanto de manera local como en servidores remotos. 
	
	Los legisladores (creadores de políticas públicas), investigadores y la comunidad utilizan información para comprender y convencer; a mejor calidad de información se consigue una mejor comprensión y políticas significativas. Un ejemplo de ello es la preocupación ciudadana de la ciudad de Los Angeles, cuyos ciudadanos pudieron establecer una relación entre la contaminación del aire y la salud pública. El área contaminación del aire y salud pública es ampliamente estudiada en todo el país utilizando métodos de recolección de datos de manera top-down y bottom-up, y se estima que una red de recolección de datos aportaría una contribución valiosa. El artículo "Elemental Carbon and PM2.5 Levels in an Urban Community Heavily Impacted by Truck Traffic" documenta un estudio hecho de manera conjunta entre la universidad y la comunidad acerca de circulación desproporcionada de tránsito pesado y las tasas de asma registradas. De esta manera, los investigadores de la universidad local llevaron adelante el monitoreo de partículas con equipamiento especializado y con la colaboración de la comunidad para documentar el tráfico comercial de camiones, y eventualmente relacionar la densidad del tráfico con los niveles de partículas saturados de diesel y evidenciar el uso ilegal de rutas no comerciales; información que pueden influir sobre políticas públicas y de salud.
	
	Una arquitectura que permita el participatory sensing puede mejorar y sistematizar la metodología existente incrementando la cantidad, calidad y credibilidad de la información reunida por la comunidad. Implementando protocolos de recolección de información adaptativos basados en estadísticas locales o globales, el participatory sensing facilitaría datos confiables mediante geolocalización, o habilitando la subida automática de información desde equipamiento especializado que todavía no esté conectado a una red. En proyectos con diseño profesional de recolección de datos, se puede incrementar la información recabada distribuyendo observaciones previas hechas con aplicaciones que estén en distribuidas entre los participantes; por ejemplo, conteos previos de cantidad de camionetas en el tránsito del lugar.
	
	Entonces el protocolo adaptado de recolección de información, la ayuda brindada desde la aplicación, la geolocalización y la hora de la toma de la muestra incrementa la confiabilidad de la información generada. Con esta información también se podría detectar en qué lugares o momentos falta recolectar información, y podría pedirle a los participantes que toma la muestra, de ser posible, a una hora determinada o en un lugar en particular. La utilización de auriculares y micrófonos (handset) se abren las posibilidades de capturar información relativa a la exposición individual y la actividad. Además de la recolección de datos interactiva, muestras de audio tomadas en forma periódica del medio que rodea al usuario pueden ser analizadas para detectar si el dispositivo está en medio de un embotellamiento, uno de los motivos principales de la exposición a partículas de diesel. Los dispositivos móviles pueden ser utilizados para detectar patrones de actividad de las personas para establecer la correlación entre los datos recolectados por agencias gubernamentales y obras sociales (healthcare providers); dicha información podría ayudar a los médicos a analizar patrones de exposición a partículas de pacientes, y también en el análisis de actividades y exposición de comunidades. \cite{burke2006participatory}
	