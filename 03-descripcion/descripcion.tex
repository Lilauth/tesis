\chapter{Características de los Proyectos de Recolección}

\section{Proyectos de Recolección}
Los proyectos de recolección generalmente tienen algunas complejidades particulares. Se necesita mucha gente para tomar las muestras, cubren grandes extensiones de territorio. Los voluntarios son imprescindibles en estos proyectos, pero también la capacitación previa o conocimiento en el área aportan. La toma de muestras sigue determinado conjunto de pasos cuyo orden debe respetarse para que la muestra sea considerada una muestra. 
\subsubsection{Motivos para Reclutar Voluntarios}	
En esta sección podemos ampliar el tema de las características de los proyectos de recolección. Hablar de CBC que abarca una parte muy grande de América del Norte, podemos hablar del proyecto de investigación por el cual AppEAr existe, que es el relevamiento de estuarios en Argentina. Podemos hablar de la app africana para relevar determinadas cosas que hay en terrenos súper inaccesibles, esta app es súper interesante porque es sólo pictográfica, lo que la vuelve amigable para mucha de la gente que no sabe leer ni escribir. 
\subsubsection{Tecnología Móvil Aplicada a la Toma de Muestras}	
Cómo los Dispositivos móviles pueden utilizarse para recolectar muestras del medio. 

\section{Descripción 2}
Alguna reseá correspondiente a la descripción 2





