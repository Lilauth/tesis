\chapter{Características de los Proyectos de Recolección}

\section{Proyectos de Recolección}

	
%Los proyectos de recolección generalmente tienen algunas complejidades particulares. Se necesita mucha gente para tomar las muestras, cubren grandes extensiones de territorio. Los voluntarios son imprescindibles en estos proyectos, pero también la capacitación previa o conocimiento en el área aportan. La toma de muestras sigue determinado conjunto de pasos cuyo orden debe respetarse para que la muestra sea considerada una muestra. %

\subsection{El Método Científico}
No podemos hablar mucho acá, sino más que nada dar una introducción al método científico, que es el que determina por qué una muestra es una muestra, explicar un poco cómo sería el protocolo de la toma de muestras de campo y eso. Esto podemos consultarlo, para ver de donde sacara bibliografía.

\subsection{Toma de Muestras}

	El artículo Citizen Science: Can volunteers do real research? describe el caso de un voluntario que colabora \textit{con un proyecto de ciencia ciudadana en el Appalachian Trail}. Luego de caminar varios kilómetros, se detiene en un lugar predeterminado y saca su GPS para estar seguro de que es el lugar correcto. Se aparta del sendero peatonal y busca un sendero menos transitado en el bosque más denso. Encuentra un camino lleno de huellas de ciervo y excremento que le indica que es el camino que está buscando.
	
	Este voluntario preside el Natural Bridge Appalachian Trail en Lynchburg, y camina de un lado a otro buscando una cámara digital que dejó en un árbol un mes atrás. Luego de buscar durante varios minutos la encuentra. Apaga la cámara, reemplaza la tarjeta de memoria y las pilas. Vuelve al sendero con huellas de animales, lo recorre dos tercios de milla y lo vuelve a dejar en otro árbol cuidando que el objetivo apunte al sendero. Luego se pone en la mira de la cámara y se mueve hasta escuchar el obturador activado por el sensor de movimiento. 
	
	De esta manera el y otros cientos colaboran con un proyecto de censado de mamíferos del Appalachian Trail desde el sur de Virginia hasta Pennsylvania, y manejan equipamiento, recogen información y anotan observaciones como uno de los varios proyectos que manejan en conjunto agencias gubernamentales, universidades, grupos de conservación ecológica y científicos para supervisar las tendencias ambientales en las cerca de 2175 millas del Appalachian Trail. Tanto el censo de mamíferos como el Appalachian Trail MEGA Transect dependen de voluntarios. Los científicos ciudadanos ayudan a supervisar animales salvajes, plantas u otros objetivos medioambientales y no reciben un pago por su colaboración e incluso a veces no son científicos profesionales. Son aficionados que colaboran con estos proyectos porque disfrutan de estar afuera o porque están comprometidos con los problemas ecológicos y quieren hacer algo al respecto. Generalmente no analizan la información o escriben artículos científicos, pero son esenciales para recolectar la información en la que luego se basan los estudios. 
	
	La Ciencia Ciudadana no es nueva. Lo que es nuevo es la cantidad de voluntarios que se enlistan para colaborar en los estudios, y la amplitud de la información que se les pide recoger. Los investigadores a menudo les piden a los colaboradores que utilicen equipamiento y técnicas sofisticadas para monitorear la calidad del aire y el agua; que documenten el crecimiento, la floración y la muerte de las plantas o que observen cuando las aves y otros animales migran atravesando un área o de qué manera se comportan mientras están en la misma. 
	
%En esta sección podemos ampliar el tema de las características de los proyectos de recolección. Hablar de CBC que abarca una parte muy grande de América del Norte, podemos hablar del proyecto de investigación por el cual AppEAr existe, que es el relevamiento de estuarios en Argentina. Podemos hablar de la app africana para relevar determinadas cosas que hay en terrenos súper inaccesibles, esta app es súper interesante porque es sólo pictográfica, lo que la vuelve amigable para mucha de la gente que no sabe leer ni escribir. %

\subsection{Protocolos de Recolección de Muestras}

	Los protocolos para recolección de muestras en proyectos que incluyen a científicos ciudadanos deben ser simples. Es más sencillo pedir que se identifique, documente o cuente 5 o 10 especies diferentes de plantas que sean fácilmente reconocibles y que serviría para indicar que la especie está presente en el área, en vez de pedir que se identifiquen todas las especies presentes en un área determinada. Una manera de ayudar a los voluntarios es darle libros con guías o material impreso que los ayude.
	
	Como hacemos referencia en el marco teórico, en los primeros estudios la información recolectada por los científicos ciudadanos era imprecisa como para ser utilizada. El principal problema es que la información generada por los voluntarios a veces representan valores en un rango en vez de números específicos, lo que dificulta la detección de cambios en los valores o fundamentar conclusiones. Ahora, los científicos ciudadanos son entrenados para leer instrumentos y recolectar números específicos. Esto debe seguir siendo compatible con protocolos de recolección simples. De esta manera se evitaría caer en el error de solicitarle a los voluntarios la recolección de información muy compleja o detallada. Pero algunas veces se puede solicitar este tipo de complejidad o precisión en la recolección de información. Pero esto no siempre es así. Muchos de los voluntarios que participan en los estudios tienen algún tipo de conocimiento acerca del método científico. De todas maneras debe esperarse información de calidad variada. En esos casos los científicos que dirigen los proyectos deben estar preparados para escrutar cuidadosamente la información obtenida y deben estar dispuestos a descartar información sospechosa o poco confiable.
	
	
	


\section{Ciencia Ciudadana y Tecnología}

	Los teléfonos celulares son dispositivos que están presentes en casi todos los ámbitos y su capacidad de capturar, clasificar y transmitir imágenes, acústica, ubicación y otra información de manera autónoma o interactiva está en crecimiento.
Planteando la arquitectura adecuada, pueden actuar como una red de sensores e instrumentos de recolección de información de localización. 
Esta forma de red de sensores distribuidos puede tener aplicaciones científicas, industriales y militares. Se sabe menos acerca de su función y utilidad en la esfera pública, es decir cuando los que los operan y poseen son usuarios regulares.   Estos sensores en vez de estar en manos de un coordinador central, están siempre bajo el control de sus dueños.

	Solicitar que los dispositivos móviles que ya están [deployados] en el campo, que armen redes de sensores de manera interactiva, participativa y que le permitan a los usuarios del público general y a los profesionales reunir, analizar y compartir información regional. Los micrófonos y cámaras que están presentes en el [handset] del dispositivo pueden registrar información del entorno, mientras se siguen integrando otros sensores de manera inalámbrica. La localización brindada por las antenas de telefonía, el GPS y otras tecnologías proveen información de ubicación y [time-synchronization]. [Las radios wireless] La conexión y el procesamiento que brinda el dispositivo permiten la interacción con la información procesada tanto de manera local como en servidores remotos. 
	
	Los legisladores (creadores de políticas públicas), investigadores y la comunidad utilizan información para comprender y convencer; a mejor calidad de información se consigue una mejor comprensión y políticas significativas. Un ejemplo de ello es la preocupación ciudadana de la ciudad de Los Angeles, cuyos ciudadanos pudieron establecer una relación entre la contaminación del aire y la salud pública. El área contaminación del aire y salud pública es ampliamente estudiada en todo el país utilizando métodos de recolección de datos de manera top-down y bottom-up, y se estima que una red de recolección de datos aportaría una contribución valiosa. El artículo "Elemental Carbon and PM2.5 Levels in an Urban Community Heavily Impacted by Truck Traffic" documenta un estudio hecho de manera conjunta entre la universidad y la comunidad acerca de circulación desproporcionada de tránsito pesado y las tasas de asma registradas. De esta manera, los investigadores de la universidad local llevaron adelante el monitoreo de partículas con equipamiento especializado y con la colaboración de la comunidad para documentar el tráfico comercial de camiones, y eventualmente relacionar la densidad del tráfico con los niveles de partículas saturados de diesel y evidenciar el uso ilegal de rutas no comerciales; información que pueden influir sobre políticas públicas y de salud.
	
	Una arquitectura que permita el participatory sensing puede mejorar y sistematizar la metodología existente incrementando la cantidad, calidad y credibilidad de la información reunida por la comunidad. Implementando protocolos de recolección de información adaptativos basados en estadísticas locales o globales, el participatory sensing facilitaría datos confiables mediante geolocalización, o habilitando la subida automática de información desde equipamiento especializado que todavía no esté conectado a una red. En proyectos con diseño profesional de recolección de datos, se puede incrementar la información recabada distribuyendo observaciones previas hechas con aplicaciones que estén en distribuidas entre los participantes; por ejemplo, conteos previos de cantidad de camionetas en el tránsito del lugar.
	
	Entonces el protocolo adaptado de recolección de información, la ayuda brindada desde la aplicación, la geolocalización y la hora de la toma de la muestra incrementa la confiabilidad de la información generada. Con esta información también se podría detectar en qué lugares o momentos falta recolectar información, y podría pedirle a los participantes que toma la muestra, de ser posible, a una hora determinada o en un lugar en particular. La utilización de auriculares y micrófonos (handset) se abren las posibilidades de capturar información relativa a la exposición individual y la actividad. Además de la recolección de datos interactiva, muestras de audio tomadas en forma periódica del medio que rodea al usuario pueden ser analizadas para detectar si el dispositivo está en medio de un embotellamiento, uno de los motivos principales de la exposición a partículas de diesel. Los dispositivos móviles pueden ser utilizados para detectar patrones de actividad de las personas para establecer la correlación entre los datos recolectados por agencias gubernamentales y obras sociales (healthcare providers); dicha información podría ayudar a los médicos a analizar patrones de exposición a partículas de pacientes, y también en el análisis de actividades y exposición de comunidades. \cite{burke2006participatory}
	
	
	[Pasa a explicar privacidad a la hora de compartir información]