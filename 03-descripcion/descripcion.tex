\chapter{Características de los Proyectos de Recolección}

\section{Proyectos de Recolección}
Los proyectos de recolección generalmente tienen algunas complejidades particulares. Se necesita mucha gente para tomar las muestras, cubren grandes extensiones de territorio. Los voluntarios son imprescindibles en estos proyectos, pero también la capacitación previa o conocimiento en el área aportan. La toma de muestras sigue determinado conjunto de pasos cuyo orden debe respetarse para que la muestra sea considerada una muestra. 

\subsection{El Método Científico}
No podemos hablar mucho acá, sino más que nada dar una introducción al método científico, que es el que determina por qué una muestra es una muestra, explicar un poco cómo sería el protocolo de la toma de muestras de campo y eso. Esto podemos consultarlo, para ver de donde sacara bibliografía.

\subsection{Toma de Muestras}	
En esta sección podemos ampliar el tema de las características de los proyectos de recolección. Hablar de CBC que abarca una parte muy grande de América del Norte, podemos hablar del proyecto de investigación por el cual AppEAr existe, que es el relevamiento de estuarios en Argentina. Podemos hablar de la app africana para relevar determinadas cosas que hay en terrenos súper inaccesibles, esta app es súper interesante porque es sólo pictográfica, lo que la vuelve amigable para mucha de la gente que no sabe leer ni escribir. 

\section{Ciencia Ciudadana y Tecnología}

	Los teléfonos celulares son dispositivos que están presentes en casi todos los ámbitos y su capacidad de capturar, clasificar y transmitir imágenes, acústica, ubicación y otra información de manera autónoma o interactiva está en crecimiento.
Planteando la arquitectura adecuada, pueden actuar como una red de sensores e instrumentos de recolección de información de localización. 
Esta forma de red de sensores distribuidos puede tener aplicaciones científicas, industriales y militares. Se sabe menos acerca de su función y utilidad en la esfera pública, es decir cuando los que los operan y poseen son usuarios regulares.   Estos sensores en vez de estar en manos de un coordinador central, están siempre bajo el control de sus dueños.

	Solicitar que los dispositivos móviles que ya están [deployados] en el campo, que armen redes de sensores de manera interactiva, participativa y que le permitan a los usuarios del público general y a los profesionales reunir, analizar y compartir información regional. Los micrófonos y cámaras que están presentes en el [handset] del dispositivo pueden registrar información del entorno, mientras se siguen integrando otros sensores de manera inalámbrica. La localización brindada por las antenas de telefonía, el GPS y otras tecnologías proveen información de ubicación y [time-synchronization]. [Las radios wireless] La conexión y el procesamiento que brinda el dispositivo permiten la interacción con la información procesada tanto de manera local como en servidores remotos. 

	